
%MA Thesis
% Applications like firefox, chrome, mysql, postgresql or nginx are written in C/C++ 
% mostly for performance reasons or to have better control thereover, availability 
% and a vast number of third party libraries are other strong reasons. Yet using 
% these languages comes at the price of allowing code reusage attacks, as up to this
% day buffer overflows and other memory corruption exploits are haunting the various
% projects using C/C++. This is not the main focus, but only a prerequisite of the 
% attacks we are going to discuss. The language c++, which initially was built based
% on C introduced the concept of inheritance to allow for more flexible designs. 
% This modelled by storing a pointer to a table that stores the virtual functions 
% of the particular object. The relatively recently discovered COOP exploits and its
% successors leverage this pointer to change the control flow hijacking the attacked
% program. Although C does not employ the concept of virtual calls, it is still 
% attackable by modifying global code pointers as shown in the Control Flow Bending
% paper.
% 
% While there exists extensive work to protect binaries from the source level, 
% one might not have access to the the sourcecode or compilation process, 
% therefore binary based solutions must also be considered , of which there
% are near to none that can mitigate COOP exploits. In this thesis, we present
% \textit{TypeShield} a tool implemented ontop of the principles introduced by
% TypeArmor, which reportedly can mitiagte COOP attacks to a certain extent. 
% We partially verify the results of TypeArmor and implement our own matching
% schema based on the parameter wideness of callsites and calltargets. 
% Our classification schema achieves an additional reduction of the possible
% calltargets per callsite of up to 20\% with an overall reduction of about 
% 9\% when comparing to parameter count based aproaches.



%Paper abstract
High security, high performance and high availability 
applications such as the Firefox and Chrome web browsers 
are implemented in C++ for modularity, performance and 
compatibility to name just few reasons.
Virtual functions, which facilitate late binding,
are a key ingredient in facilitating run-time polymorphism
in C++ because it allows and object to use general (its own) 
or specific functions (inherited) contained in the class hierarchy.
However, because of the specific implementation of late binding,
which performs no verification in order to check where an indirect call site 
(virtual object dispatch through virtual pointers (vptrs)) is allow to
call inside the class hierarchy, this opens a large attack surface which
was successfully exploited by the COOP attack.
Since manipulation (changing or inserting new vptrs) violates the 
programmer initial pointer semantics and allows an attacker to
redirect the control flow of the program as he desires, vptrs corruption
has serious security consequences similar to those of other 
data-only corruption vulnerabilities.
Despite the alarmingly high number of vptr corruption
vulnerabilities, the vptr corruption problem has not
been sufficiently addressed by the researchers.

In this paper, we present \textit{TypeShild}, a run-time vptr corruption
detection tool. It is based on executable instrumentation at load time
and uses a novel run-time type and function parameter counter technique
in order to overcome the limitations of current approaches and efficiently
verify dynamic dispatching during run-time.
In particular, \textit{TypeShild} can be automatically and easily used
in conjunction with legacy applications or where source code is missing.
It achieves higher caller/caller matching (precision) and with reasonable
run-time overhead.
We have applied \textit{TypeShild} to real life software such as
web servers, JavaScript engines, FTP servers and large-scale software
including Chrome and Firefox browsers, and were able to efficiently
and with low performance overhead to protect this applications from 
vptr corruptions vulnerabilities.
Our evaluation shows that our target reductin schema achieves an additional
reduction of the possible calltargets per callsite of up to 
20\% with an overall reduction of about 9\% when comparing 
to parameter count based aproaches.
