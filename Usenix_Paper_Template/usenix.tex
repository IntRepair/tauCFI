% TEMPLATE for Usenix papers, specifically to meet requirements of
%  USENIX '05
% originally a template for producing IEEE-format articles using LaTeX.
%   written by Matthew Ward, CS Department, Worcester Polytechnic Institute.
% adapted by David Beazley for his excellent SWIG paper in Proceedings,
%   Tcl 96
% turned into a smartass generic template by De Clarke, with thanks to
%   both the above pioneers
% use at your own risk.  Complaints to /dev/null.
% make it two column with no page numbering, default is 10 point

% Munged by Fred Douglis <douglis@research.att.com> 10/97 to separate
% the .sty file from the LaTeX source template, so that people can
% more easily include the .sty file into an existing document.  Also
% changed to more closely follow the style guidelines as represented
% by the Word sample file. 

% Note that since 2010, USENIX does not require endnotes. If you want
% foot of page notes, don't include the endnotes package in the 
% usepackage command, below.

\documentclass[letterpaper,twocolumn,10pt]{article}
\usepackage{usenix,epsfig,endnotes}
\usepackage{algorithmic}
\usepackage[resetcount, ruled, linesnumbered, plain]{algorithm2e}
\usepackage{stackengine}
\usepackage{tikz}
\lstset{
  language=C++,
  xleftmargin=.05\textwidth,      % <-- change this to a suitable length
%   xrightmargin=.05\textwidth,  % <-- change this to a suitable negative length
  numbersep=2pt,        % default 10pt
  numbers=left,
  numberstyle=\ttfamily\scriptsize,
  basicstyle=\ttfamily\scriptsize,
  breaklines=true,
  keywordstyle=\bfseries,
  showstringspaces=false,
  numbers=left, 
  numbersymbol=$.$,
  firstnumber=0,
  keepspaces=true,
  moredelim=[is][\underbar]{+}{+},
  commentstyle=\color{ForestGreen},
  morekeywords={try, return, include, printf, extern, true, false, logic}
}
\begin{document}

%don't want date printed
\date{}

%make title bold and 14 pt font (Latex default is non-bold, 16 pt)
\title{\Large \bf Wonderful : A Terrific Application and Fascinating Paper}

\author{
{\rm Your N.\ Here}\\
Your Institution
\and
{\rm Second Name}\\
Second Institution
}

\maketitle

% Use the following at camera-ready time to suppress page numbers.
% Comment it out when you first submit the paper for review.
\thispagestyle{empty}


\subsection*{Abstract}
Your Abstract Text Goes Here.  Just a few facts.
Whet our appetites.

\section{Introduction}

A paragraph of text goes here.  Lots of text.  Plenty of interesting
text. \\

More fascinating text. Features\endnote{Remember to use endnotes, not footnotes!} galore, plethora of promises.\\

\section{Alg.}

\section{Alg. example}
\IncMargin{1.3em} 
\begin{algorithm}[t!]
\caption{Quick fix locations searching and patches generation}
\label{alg:one}
\algsetup{linenosize=\tiny}
  \scriptsize
\SetAlgoLined
\KwIn{Satisfiable program execution paths set $S_{Paths}:=$ \{$s_k\arrowvert$ 0 $\leq$ k $\leq$n, $\forall$ n $\geq$ 0 \}}

\KwOut{Refactorings set $R_{set}:=$ \{$r_j\arrowvert$ 0 $\leq$ j $\textless$ 2 \}} 

$W_{set}:=$ \{$w_k\arrowvert$ 0 $\leq$ k $\leq$ n,$\forall$ n $\geq$ 0 \}; \textcolor{ForestGreen}{\textbf{\fontfamily{cmtt}\selectfont{// set of working lists, k'th list}}}\\
$N_{set}:=$ \{$n_t\arrowvert$ 0 $\leq$ t $\leq$ n,$\forall$ n $\geq$ 0 \}; \textcolor{ForestGreen}{\textbf{\fontfamily{cmtt}\selectfont{// set of nodes}}}\\
 
$N_{set}:=\emptyset;  W_{set}:=\emptyset;$ \textcolor{ForestGreen}{\textbf{\fontfamily{cmtt}\selectfont{// initializing both nodes set and working list set to empty set}}}\\

$countBP:$=0; $countGQF:$=0; \textcolor{ForestGreen}{\textbf{\fontfamily{cmtt}\selectfont{// init. counters, count buggy paths and generated fixes}}}\

$R_{set}:$=$\emptyset$; 

 \While{(($Sat_{paths}$.hasNext))}{
     
     \If{(hasBug($s_k$)}{ 
        
        $countBP$ := $countBP$ + 1; \textcolor{ForestGreen}{\textbf{\fontfamily{cmtt}\selectfont{// count the buggy paths}}}\\
       
        $i:= $ \textit{startIndex($s_k$)}; \textcolor{ForestGreen}{\textbf{\fontfamily{cmtt}\selectfont{// set the start index of the path}}}\\
       
        $w_k:= $ \textit{setWorkList($s_k$)}; \textcolor{ForestGreen}{\textbf{\fontfamily{cmtt}\selectfont{// set the detected buggy path into the work list}}}\\
        
        $NLocs:=1;$ \textcolor{ForestGreen}{\textbf{\fontfamily{cmtt}\selectfont{// number of quick fix locations}}}\\
        
        $C:=0; $ \textcolor{ForestGreen}{\textbf{\fontfamily{cmtt}\selectfont{// quick fix locations counter}}}\\
        
        \ \textcolor{ForestGreen}{\textbf{\fontfamily{cmtt}\selectfont{// if the work list length greater than 0 else skip path}}}\\
             
             \If{(getLength($w_k$) \textgreater \ 0) }{  
               
               $n_t$ := \textit{initNode($w_k$)}; \textcolor{ForestGreen}{\textbf{\fontfamily{cmtt}\selectfont{// the node at which the bug was detected}}}\
               
               $N_{set}:$=$N_{set}$ $\cup$ \{$n_t$\}; \textcolor{ForestGreen}{\textbf{\fontfamily{cmtt}\selectfont{// add a node for the in-place fix}}}\
		
		$r_j$ := \textit{refact($n_t$)}; \textcolor{ForestGreen}{\textbf{\fontfamily{cmtt}\selectfont{// create a new bug refactoring}}}\
		
		$R_{set}:$=$R_{set}$ $\cup$ \{$r_j$\}; \textcolor{ForestGreen}{\textbf{\fontfamily{cmtt}\selectfont{// add new refactoring to the set R}}}\
		 
		 \While{($i$ \textgreater 0 $\wedge$ $C$ \textless $NLocs$)}{
		   
		   $fNode$ := \{$w_k,_i$\};  \textcolor{ForestGreen}{\textbf{\fontfamily{cmtt}\selectfont{// get next node from work list located at index i}}}\ 
		
		  \If{(isQuickFixNode($fNode$))}{
		
		   $n_{t+1}$ := $fNode$; \textcolor{ForestGreen}{\textbf{\fontfamily{cmtt}\selectfont{// store current node}}}\       
                   
                   $N_{set}$:=$N_{set}$ $\cup$ \{$n_{t+1}$\};  \textcolor{ForestGreen}{\textbf{\fontfamily{cmtt}\selectfont{// add the node for a not in-place fix}}}\  
		     
		     \textit{setConsObject($w_k$)}; \textcolor{ForestGreen}{\textbf{\fontfamily{cmtt}\selectfont{// store constraint}}}\       
			
			  \If{(notAffectedPaths($S_{Paths}$, $n_{t+1}$))}{ 
                             $pLoc$ := \textit{probLoc($n_{t+1}$)};  \\
			      \textit{putMarker($pLoc$)}; \textcolor{ForestGreen}{\textbf{\fontfamily{cmtt}\selectfont{// put new marker}}}\ \\
			      $r_{j+1}$ := \textit{refact($n_{t+1}$)}; \textcolor{ForestGreen}{\textbf{\fontfamily{cmtt}\selectfont{// create a new bug refactoring}}}\  \\
			      $R_{set}:$=$R_{set}$ $\cup$ \{$r_{j+1}$\}; \textcolor{ForestGreen}{\textbf{\fontfamily{cmtt}\selectfont{// add refactoring}}}\  \\
                              $countGQF$ := $countGQF$ + 1; \textcolor{ForestGreen}{\textbf{\fontfamily{cmtt}\selectfont{// count the generated fixes}}}\\

                          }

			$C$ := $C$ + 1; \textcolor{ForestGreen}{\textbf{\fontfamily{cmtt}\selectfont{// increase not in-place quick fix locations counter}}}\
	         }
		 $i$ := $i$ - 1; \textcolor{ForestGreen}{\textbf{\fontfamily{cmtt}\selectfont{// go one step backwards on the path}}}\
		}
	      }
	$k$ := $k$ + 1; \textcolor{ForestGreen}{\textbf{\fontfamily{cmtt}\selectfont{// get next satisfiable program execution path}}}\ \\
     }
  }
\end{algorithm}


\section{This is Another Section}

Some embedded literal typset code might 
look like the following :

{\tt \small
\begin{verbatim}
#include <iostream>
using namespace std;
main()
{
cout << "Hello world \n";
return 0;
}

\end{verbatim}
}

Now we're going to cite somebody.  Watch for the cite tag.
Here it comes~\cite{Einstein}.  

Lorem ipsum dolor sit amet, consectetur adipiscing elit, sed do eiusmod tempor incididunt ut labore et dolore magna aliqua. Ut enim ad minim veniam, quis nostrud exercitation ullamco laboris nisi ut aliquip ex ea commodo consequat. Duis aute irure dolor in reprehenderit in voluptate velit esse cillum dolore eu fugiat nulla pariatur. Excepteur sint occaecat cupidatat non proident, sunt in culpa qui officia deserunt mollit anim id est laborum.

Lorem ipsum dolor sit amet, consectetur adipiscing elit, sed do eiusmod tempor incididunt ut labore et dolore magna aliqua. Ut enim ad minim veniam, quis nostrud exercitation ullamco laboris nisi ut aliquip ex ea commodo consequat. Duis aute irure dolor in reprehenderit in voluptate velit esse cillum dolore eu fugiat nulla pariatur. Excepteur sint occaecat cupidatat non proident, sunt in culpa qui officia deserunt mollit anim id est laborum.

Lorem ipsum dolor sit amet, consectetur adipiscing elit, sed do eiusmod tempor incididunt ut labore et dolore magna aliqua. Ut enim ad minim veniam, quis nostrud exercitation ullamco laboris nisi ut aliquip ex ea commodo consequat. Duis aute irure dolor in reprehenderit in voluptate velit esse cillum dolore eu fugiat nulla pariatur. Excepteur sint occaecat cupidatat non proident, sunt in culpa qui officia deserunt mollit anim id est laborum.

Lorem ipsum dolor sit amet, consectetur adipiscing elit, sed do eiusmod tempor incididunt ut labore et dolore magna aliqua. Ut enim ad minim veniam, quis nostrud exercitation ullamco laboris nisi ut aliquip ex ea commodo consequat. Duis aute irure dolor in reprehenderit in voluptate velit esse cillum dolore eu fugiat nulla pariatur. Excepteur sint occaecat cupidatat non proident, sunt in culpa qui officia deserunt mollit anim id est laborum.

{\footnotesize \bibliographystyle{acm}
\bibliography{sample}}


\theendnotes

\end{document}







