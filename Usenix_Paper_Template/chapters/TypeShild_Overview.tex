\section{\textit{TypeShield} Overview}
\label{chapter:TypeShild Overview}

\textbf{Adversary Model and Assumptions.}
\label{Adversary Model}
We largely use the same threat model and the same basic assumptions as described in the TypeArmor 
paper \cite{veen:typearmor}, meaning that our attacker has read and write access to the data 
sections of the attacked binary.  We also assume that the protected binary does not contain 
self modifying code, handcrafted assembly or any kind of obfuscation. We also consider pages 
to be either writable or executable but not both at the same time. Furthermore we assume 
that our attacker has the ability to execute a memory corruption to hijack the programs 
control flow. As our schema targets only forward control flow, namely indirect function 
calls, we assume that a solution for backward CFI edges is in place.

\textbf{Invariants for Targets and Callsites.}
\label{Invariants for Targets and Callsites}
Advanced code reuse attacks attempt to change the calltargets that are invoked within indirect 
call-sites, standard CFI solutions cannot defend against this and TypeArmor proposed the approach
of creating two sets of invariants. 

\begin{enumerate}
\item Indirect call-sites provide a number of parameters (possibly overestimated compared to source)
\item Call-targets require a minimum number of parameters (possibly underestimated compared to source)
\end{enumerate}

The main idea is now that a call-site might not call any function in the binary but only call-targets
that do not require more parameters than provided by the call-site itself. To achieve this classification
of call-targets and call-sites, TypeArmor proposed to use a modified version of forward liveness analysis
for call-targets and backward reaching definitions analysis for call-sites.


\textbf{\textit{TypeShield} Impact on COOP.}
\label{TypeShild Impact on COOP}
The problem with relying solely on the parameter count is that a call-site being classified as using
6 or more parameters can use basically all address taken functions within the binary. This is however
counterproductive and we attempt a possible solution, by extending the classification schema to the
single parameters themselves:

\begin{enumerate}
\item Indirect call-sites provides a maximum wideness of value to each parameter (possibly overestimated compared to source)
\item Call-targets require a minimum wideness of value for each parameter(possibly underestimated compared to source)
\end{enumerate}

Basically the principle stays the same, but instead of just requiring that the call-site parameter count
is not lower than the call-target parameter count we now require the same also for the wideness of each
parameter.

\begin{figure}[h!]
\centering
\resizebox{0.25\textwidth}{!}{
\begin{tikzpicture}

\draw[thick] (0,0) rectangle node[anchor=center]  (320)  {\LARGE{(32, 0)}}  (2,-1);

\draw[thick] (3,2) rectangle node[anchor=center]  (640)  {\LARGE{(64, 0)}}  (5,1);

\draw[thick] (0,3) rectangle node[anchor=center]  (3232) {\LARGE{(32, 32)}} (2,4);

\draw[thick] (3,5) rectangle node[anchor=center]  (6432) {\LARGE{(64, 32)}} (5,6);
\draw[thick] (-1,5) rectangle node[anchor=center] (3264) {\LARGE{(32, 64)}} (-3,6);

\draw[thick] (0,7) rectangle node[anchor=center]  (6464) {\LARGE{(64, 64)}} (2,8);

  \draw[draw, -triangle 45, thick, arrowhead=1cm] (3264.north) -- (6464.south);
  \draw[draw, -triangle 45, thick, arrowhead=1cm] (6432.north) -- (6464.south);
  \draw[draw, -triangle 45, thick, arrowhead=1cm] (3232.north) -- (3264.south);
  \draw[draw, -triangle 45, thick, arrowhead=1cm] (3232.north) -- (6432.south);
  
  \draw[draw, -triangle 45, thick, arrowhead=1cm] (640.north) -- (6432.south);
  \draw[draw, -triangle 45, thick, arrowhead=1cm] (320.north) -- (3232.south);
  \draw[draw, -triangle 45, thick, arrowhead=1cm] (320.north) -- (640.south);
  
\end{tikzpicture}
}
\caption{Example for the wideness based schema when only using a parameter wideness of 64, 32 and 0 bits.}
\label{fig:lattice3264}
\end{figure}

While there are still occurrences where call-sites may target all call-targets, we split the buckets up
into smaller ones, as shown in Figure \ref{fig:lattice3264}. For example in the parameter-count oriented
schema a call-site classified as (32,32) would be able to call functions classified as (64,0), however
in the parameter wideness oriented schema that is not possible.

