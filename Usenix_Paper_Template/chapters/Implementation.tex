\section{Implementation}
\label{chapter:Implementation}

We implemented \textsc{TypeShield} as a module pass for the \textit{di-opt} environment pass provided by the 
DynInst~\cite{bernat:dyninst} instrumentation framework (v. 9.2.0). 
However, converting the pass to a standalone executable is also possible, as we do not rely on an extended set of DynInst features
except for the pass abstraction.

We currently restricted our analysis and instrumentation to x86-64 bit elf 
binaries using the SystemV call convention, because the DynInst library does not yet support the
Windows platform. However, there is currently work going on in order to allow DynInst to work with
Windows binaries as well. We focused on the SystemV call convention as most C/C++ compilers
on Linux implement this ABI, however we encapsulated most ABI dependent behavior, so it should 
be possible to implement other ABIs with relative ease. Therefore, we deem it possible to implement
\textsc{TypeShield} for the Windows platform in the near future, as we do not use any other 
platform-dependent API's. 

We developed the core part of our pass in an instruction analyzer, which relies on the DynamoRIO~\cite{dynamorio:drmemory} library 
(v. 6.6.1) to decode single instructions and provide access to its information. The analyzer is then
used to implement our version of the reaching and liveness analysis (similar to PathArmor~\cite{veen:typearmor}), which can
be customized with relative ease, as we allow for arbitrary path merging functions. However, we implemented 
the three basic versions as follows: destructive, intersection and union.
In order to accomplish this we patched the DynInst library in order to allow for local annotation of call-targets with arbitrary
information, leveraging its relocation schema, which relies on the basic block abstraction.

We implemented a Clang/LLVM (v. 4.0.0, trunk 283889) pass used for 
collecting ground truth data in order to measure the quality and performance of our tool. 
The ground truth data is then used to verify the output of our tool for several test targets. 
This is accomplished with the help of our python based evaluation and test environment.

In total we implemented \textsc{TypeShield} in 5123 source code lines (SLoC) of C++ code, our Clang/LLVM pass in 200 SLoC
of C++ code and our test environment in 2674 SLoC of Python code. 


