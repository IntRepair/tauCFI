\section{Implementation}
\label{chapter:Implementation}

We implemented \textsc{TypeShield} using the DynInst~\cite{bernat:dyninst} (v.9.2.0) instrumentation framework. We currently restricted our analysis and instrumentation to x86-64 
bit elf binaries using the Itanium C++ ABI call convention.
We focused on the Itanium C++ ABI call 
convention as most C/C++ compilers on Linux implement this ABI, however, we encapsulated most ABI-dependent behavior, so it should be 
possible to support other ABIs as well. 
We developed the main part of our binary analysis pass in an instruction analyzer, which relies on the DynamoRIO~\cite{dynamorio:drmemory} library (v.6.6.1)
to decode single instructions and provide access to
its information. 
The analyzer is then used to implement our version of the reaching and liveness analysis, 
which can be customized with relative ease, as we allow for arbitrary path merging functions. 
Next, we implemented a 
Clang/LLVM (v.4.0.0, trunk 283889) back-end pass used for collecting ground truth data in order to measure the quality and performance of our tool. 
The ground truth data is then used to verify 
the output of our tool for several test targets. This is accomplished with the help of our Python-based evaluation and test environment. 
In total, we implemented \textsc{TypeShield} in 5501 lines of code (LOC) of C++ code, our Clang/LLVM pass in 416 LOC
of C++ code and our test environment in 3239 Python LOC. 


