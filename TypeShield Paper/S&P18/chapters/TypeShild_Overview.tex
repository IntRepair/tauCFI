\section{System Overview}
\label{chapter:TypeShild Overview}
In~\cref{Overview} we present the main steps performed by \textsc{TypeShield} in order to harden a program binary
% we present in~\cref{TypeShield Policy Mechanism} our type based policy and approach overview 
% In~\cref{Backward Edge CFI Policy} we highlight our backward edge protection policy, 
while  
% in~\cref{Invariants for Targets and Callsites} we present the invariants for calltargets and callsites, while 
in~\cref{Adversary Model} we introduce the threat model used in this paper.
% Finally, in~\cref{TypeShild Impact on COOP} we talk about parameter count vs. parameter type policies.
\subsection{Approach Overview}
\label{Overview}
Figure~\ref{System overview.} depicts the overview our approach.
From right to left the program binary is analyzed (see left hand side in Figure~\ref{System overview.}) by \textsc{TypeShield} and the calltargets
and callsite analysis are performed for determining 
how many parameters are provided, 
how many are consumed and their register wideness.
After this step labels are inserted at each previously identified callsite and at each calltarget. 
The enforced policy is schematically represented by the black highlighted dots (addresses) in Figure~\ref{System overview.} which are allowed to call only legitimate red highlighted dots (addresses).
Next for each function return address the address set determined by each address located after each legitimate (is allowed to call the function) callsite is collected.
This information is obtained by using the previously determined callsite forward-edge mapping to derive a function return backward map containing function returns as key and return targets as values.
In this way \textsc{TypeShield} has for each function return site a set of legitimate addresses where the function return site is allowed to transfer the program control flow.
Finally, range or compare checks are inserted before each function return site. This checks are used to check during runtime if the 
address where the function return wants to jump to is contained in the legitimate set for each particular return site.
This is represented in Figure~\ref{System overview.} by green highlighted dots (addresses) that are allowed to call only legitimate blue highlighted dots (addresses).
Finally, the result is a hardened program binary (see right hand side in Figure~\ref{System overview.}).



% \subsection{Invariants}
% % \textbf{Invariants for Targets and Callsites.}
% \label{Invariants for Targets and Callsites}
% 
% \subsubsection{Calltargets and Callsites}
% We propose the following invariants for the function calltargets and callsites.
% (1) indirect callsites provide a number of parameters (\textit{i.e.,} possibly overestimated compared to program source code), 
% (2) calltargets require a minimum number of parameters (\textit{i.e.,} possibly underestimated compared to program source code), and
% (3) the wideness of the callsite parameters has to be bigger or equal to the wideness of the parameters registers expected at the calltarget.
% In a nutshell the idea is that a callsite might only call functions that do not require more parameters than provided by the callsite and
% where the parameter register wideness of each parameter of the callsite is higher or equal to that parameter registers used at the calltarget.
% Figure~\ref{System overview.} depicts this requirements by the forward indirect edges pointing from the black dots to the legitimate
% red dots.
% 
% \subsubsection{Callertarget Returns}
% We propose the following invariant for the calltargets returns.
% (1) we enforce the caller caller convention between the calltarget return instruction and the address next to callsite which was used in first place to 
% call that calltarget.
% Figure~\ref{System overview.} depicts this analogy by the backward indirect edges pointing from the green dots to the legitimate
% blue dots.
% \vspace{-.99cm}
% \subsection{Type Based Policy}
% \label{TypeShield Policy Mechanism}
% 
% \begin{figure}[H]
% % \captionsetup{labelformat=empty}
% \vspace{-.3cm}
% \hspace{-.3cm}
% \resizebox{0.49\textwidth}{!}{
% \begin{tikzpicture}[shorten >=1pt,node distance=2cm,on grid,auto]
%     \begin{scope}
%             % labels
% %     \foreach \i in {0,...,5}
% %       \path[blue] (0,-0.25) node{0} (-0.25,0) node{0};
%       \path[blue] (0.25,-0.25) node{1}  (-0.25, 0.25) node{0};
%       \path[blue] (0.75,-0.25) node{2}  (-0.25, 0.75) node{1};
%       \path[blue] (1.25,-0.25) node{3}  (-0.25, 1.25) node{2};
%       \path[blue] (1.75,-0.25) node{4}  (-0.25, 1.75) node{4};
%       \path[blue] (2.25,-0.25) node{5}  (-0.25, 2.25) node{8};
%       \path[blue] (2.75,-0.25) node{6} (-0.25, 2.75) node{};
% %       \path[blue] (7/2,-0.25) node{64} (-0.25,7/2) node{64};
%     % loop over the lattice points
%     \foreach \i in {0,...,6}
%       \foreach \j in {0,...,6}{
%        \ifnum \j < 6
%         \draw (\i / 2,\j / 2) rectangle (1,1);
%         \fi
%         % check if (\i,\j) > (2,2)
% %         \ifnum \i < 3
% %           \ifnum \j < 3
% %             \fill[red] (\i + 0.25,\j + 0.25) circle(3pt);
% %           \fi
% %         \fi
%       };
%       
%       \fill (0   + 0.25, 0   + 0.25) circle(4pt);
%       \fill (0.5 + 0.25, 0.5 + 0.25) circle(4pt);
%       \fill (0 + 0.25, 0.5 + 0.25) circle(4pt);
%       \fill (0.5 + 0.25, 0 + 0.25) circle(4pt);
%       \fill (0.5 + 0.25, 1 + 0.25) circle(4pt);
%       \fill (0 + 0.25, 1 + 0.25) circle(4pt);
%       %       \draw [line width=0.5mm, green] (1 + 0.25, 1 + 0.25) circle (6pt);
%       
% %       \draw (0.5 + 0.25, 3 + 0.25) circle(4pt);
% %       \draw (0 + 0.25, 3 + 0.25) circle(4pt);
% %       \draw (1 + 0.25, 3 + 0.25) circle(4pt);
% %       \draw (1 + 0.75, 3 + 0.25) circle(4pt);
% %       \draw (2.25, 3 + 0.25) circle(4pt);
% %       \draw (2.75, 3 + 0.25) circle(4pt);
% %       \draw (3.25, 3 + 0.25) circle(4pt);
%       
% %       \draw (0.5 + 0.25, 2.5 + 0.25) circle(4pt);
% %       \draw (0 + 0.25, 2.5 + 0.25) circle(4pt);
% %       \draw (1 + 0.25, 2.5 + 0.25) circle(4pt);
% %       \draw (1 + 0.75, 2.5 + 0.25) circle(4pt);
% %       \draw (2.25, 2.5 + 0.25) circle(4pt);
% %       \draw (2.75, 2.5 + 0.25) circle(4pt);
% %       \draw (3.25, 2.5 + 0.25) circle(4pt);
%       
%       \draw (0.5 + 0.25, 2 + 0.25) circle(4pt);
%       \draw (0 + 0.25, 2 + 0.25) circle(4pt);
%       \draw (1 + 0.25, 2 + 0.25) circle(4pt);
%       \draw (1 + 0.75, 2 + 0.25) circle(4pt);
%       \draw (2.25, 2 + 0.25) circle(4pt);
%       \draw (2.75, 2 + 0.25) circle(4pt);
% %       \draw (3.25, 2 + 0.25) circle(4pt);
%       
%       \fill (0.5 + 0.25, 1.5 + 0.25) circle(4pt);
%       \fill (0 + 0.25, 1.5 + 0.25) circle(4pt);
%       \fill (1 + 0.25, 1.5 + 0.25) circle(4pt);
%       \fill (1 + 0.75, 1.5 + 0.25) circle(4pt);
%       \draw (2.25, 1.5 + 0.25) circle(4pt);
%       \draw (2.75, 1.5 + 0.25) circle(4pt);
% %       \draw (3.25, 1.5 + 0.25) circle(4pt);
%       
%       \fill (1 + 0.25, 1 + 0.25) circle(4pt);
%       \fill (1 + 0.75, 1 + 0.25) circle(4pt);
%       \fill (2.25, 1 + 0.25) circle(4pt);
%       \fill (2.75, 1 + 0.25) circle(4pt);
% %       \draw (3.25, 1 + 0.25) circle(4pt);
%       
%       \fill (1 + 0.25, 0.5 + 0.25) circle(4pt);
%       \fill (1 + 0.75, 0.5 + 0.25) circle(4pt);
%       \fill (2.25, 0.5 + 0.25) circle(4pt);
%       \fill (2.75, 0.5 + 0.25) circle(4pt);
% %       \fill (3.25, 0.5 + 0.25) circle(4pt);
%       
%       \fill (1 + 0.25, 0 + 0.25) circle(4pt);
%       \fill (1 + 0.75, 0 + 0.25) circle(4pt);
%       \fill (2.25, 0 + 0.25) circle(4pt);
%       \fill (2.75, 0 + 0.25) circle(4pt);
% %       \fill (3.25, 0 + 0.25) circle(4pt);
%     \end{scope}
%     
%     \begin{scope}[xshift=1.5cm, yshift=2.75cm]
%      \node[]    (q_1) {	\large{$M1:callsite$}};
%     \end{scope}
%     
%     \begin{scope}[xshift=3.25cm, yshift=1.2cm]
%      \node[]    (q_1) {	\large{$\wedge$}};
%     \end{scope}
%     
%     \begin{scope}[xshift=5.3cm, yshift=2.75cm]
%      \node[]    (q_1) {	\large{$M2:calltarget$}};
%     \end{scope}
%     
%     \begin{scope}[xshift=9cm, yshift=2.75cm]
%      \node[]    (q_1) {\large{$M3:policy \ result$}};
%     \end{scope}
% 
%     \begin{scope}[xshift=3.8cm]
%        % labels
% %     \foreach \i in {0,...,5}
% %       \path[blue] (0,-0.25) node{0} (-0.25,0) node{0};
%       \path[blue] (0.25,-0.25) node{1}  (-0.25, 0.25) node{0};
%       \path[blue] (0.75,-0.25) node{2}  (-0.25, 0.75) node{1};
%       \path[blue] (1.25,-0.25) node{3}  (-0.25, 1.25) node{2};
%       \path[blue] (1.75,-0.25) node{4}  (-0.25, 1.75) node{4};
%       \path[blue] (2.25,-0.25) node{5}  (-0.25, 2.25) node{8};
%       \path[blue] (2.75,-0.25) node{6} (-0.25, 2.75) node{};
% %       \path[blue] (7/2,-0.25) node{64} (-0.25,7/2) node{64};
%     % loop over the lattice points
%     \foreach \i in {0,...,6}
%       \foreach \j in {0,...,6}{
%       \ifnum \j < 6
%        \draw (\i / 2,\j / 2) rectangle (1,1);
%       \fi
%         
%         % check if (\i,\j) > (2,2)
% %         \ifnum \i < 3
% %           \ifnum \j < 3
% %             \fill[red] (\i + 0.25,\j + 0.25) circle(3pt);
% %           \fi
% %         \fi
%       };
%       
%       \draw (0   + 0.25, 0   + 0.25) circle(4pt);
%       \draw (0.5 + 0.25, 0.5 + 0.25) circle(4pt);
%       \draw (0 + 0.25, 0.5 + 0.25) circle(4pt);
%       \draw (0.5 + 0.25, 0 + 0.25) circle(4pt);
%       \draw (0.5 + 0.25, 1 + 0.25) circle(4pt);
%       \draw (0 + 0.25, 1 + 0.25) circle(4pt);
%       
% %       \fill (0.5 + 0.25, 3 + 0.25) circle(4pt);
% %       \fill (0 + 0.25, 3 + 0.25) circle(4pt);
% %       \fill (1 + 0.25, 3 + 0.25) circle(4pt);
% %       \fill (1 + 0.75, 3 + 0.25) circle(4pt);
% %       \fill (2.25, 3 + 0.25) circle(4pt);
% %       \fill (2.75, 3 + 0.25) circle(4pt);
% %       \fill (3.25, 3 + 0.25) circle(4pt);
%       
% %       \fill (0.5 + 0.25, 2.5 + 0.25) circle(4pt);
% %       \fill (0 + 0.25, 2.5 + 0.25) circle(4pt);
% %       \fill (1 + 0.25, 2.5 + 0.25) circle(4pt);
% %       \fill (1 + 0.75, 2.5 + 0.25) circle(4pt);
% %       \fill (2.25, 2.5 + 0.25) circle(4pt);
% %       \fill (2.75, 2.5 + 0.25) circle(4pt);
% %       \fill (3.25, 2.5 + 0.25) circle(4pt);
%       
%       \fill (0.5 + 0.25, 2 + 0.25) circle(4pt);
%       \fill (0 + 0.25, 2 + 0.25) circle(4pt);
%       \fill (1 + 0.25, 2 + 0.25) circle(4pt);
%       \fill (1 + 0.75, 2 + 0.25) circle(4pt);
%       \fill (2.25, 2 + 0.25) circle(4pt);
%       \fill (2.75, 2 + 0.25) circle(4pt);
% %       \fill (3.25, 2 + 0.25) circle(4pt);
%       
%       \fill (0.5 + 0.25, 1.5 + 0.25) circle(4pt);
%       \fill (0 + 0.25, 1.5 + 0.25) circle(4pt);
%       \fill (1 + 0.25, 1.5 + 0.25) circle(4pt);
%       \fill (1 + 0.75, 1.5 + 0.25) circle(4pt);
%       \fill (2.25, 1.5 + 0.25) circle(4pt);
%       \fill (2.75, 1.5 + 0.25) circle(4pt);
% %       \fill (3.25, 1.5 + 0.25) circle(4pt);
%       
%       \fill (1 + 0.25, 1 + 0.25) circle(4pt);
%       \fill (1 + 0.75, 1 + 0.25) circle(4pt);
%       \fill (2.25, 1 + 0.25) circle(4pt);
%       \fill (2.75, 1 + 0.25) circle(4pt);
% %       \fill (3.25, 1 + 0.25) circle(4pt);
%       
%       \fill (1 + 0.25, 0.5 + 0.25) circle(4pt);
%       \fill (1 + 0.75, 0.5 + 0.25) circle(4pt);
%       \fill (2.25, 0.5 + 0.25) circle(4pt);
%       \fill (2.75, 0.5 + 0.25) circle(4pt);
% %       \fill (3.25, 0.5 + 0.25) circle(4pt);
%       
%       \fill (1 + 0.25, 0 + 0.25) circle(4pt);
%       \fill (1 + 0.75, 0 + 0.25) circle(4pt);
%       \fill (2.25, 0 + 0.25) circle(4pt);
%       \fill (2.75, 0 + 0.25) circle(4pt);
% %       \fill (3.25, 0 + 0.25) circle(4pt);
% %       requires
%     \end{scope}
%     
%     \begin{scope}[xshift=7cm, yshift=1.2cm]
%      \node[]    (q_1) {\Large{$=$}};
%     \end{scope}
%     
%      \begin{scope}[xshift=7.5cm]
%             % labels
% %     \foreach \i in {0,...,5}
% %       \path[blue] (0,-0.25) node{0} (-0.25,0) node{0};
%       \path[blue] (0.25,-0.25) node{1}  (-0.25, 0.25) node{0};
%       \path[blue] (0.75,-0.25) node{2}  (-0.25, 0.75) node{1};
%       \path[blue] (1.25,-0.25) node{3}  (-0.25, 1.25) node{2};
%       \path[blue] (1.75,-0.25) node{4}  (-0.25, 1.75) node{4};
%       \path[blue] (2.25,-0.25) node{5}  (-0.25, 2.25) node{8};
%       \path[blue] (2.75,-0.25) node{6} (-0.25, 2.75) node{};
% %       \path[blue] (7/2,-0.25) node{64} (-0.25,7/2) node{64};
%     % loop over the lattice points
%     \foreach \i in {0,...,6}
%       \foreach \j in {0,...,6}{
%       \ifnum \j < 6
%        \draw (\i / 2,\j / 2) rectangle (1,1);
%       \fi
%         
%         % check if (\i,\j) > (2,2)
% %         \ifnum \i < 3
% %           \ifnum \j < 3
% %             \fill[red] (\i + 0.25,\j + 0.25) circle(3pt);
% %           \fi
% %         \fi
%       };
%       
%       \fill (0   + 0.25, 0   + 0.25) circle(4pt);
%       \draw [line width=0.5mm, red] (0   + 0.25, 0   + 0.25) circle (6pt);
%       \fill (0.5 + 0.25, 0.5 + 0.25) circle(4pt);
%       \draw [line width=0.5mm, red] (0.5 + 0.25, 0.5 + 0.25) circle (6pt);
%       \fill (0 + 0.25, 0.5 + 0.25) circle(4pt);
%       \draw [line width=0.5mm, red] (0 + 0.25, 0.5 + 0.25) circle (6pt);
%       \fill (0.5 + 0.25, 0 + 0.25) circle(4pt);
%       \draw [line width=0.5mm, red] (0.5 + 0.25, 0 + 0.25) circle (6pt);
%       \fill (0.5 + 0.25, 1 + 0.25) circle(4pt);
%       \draw [line width=0.5mm, red] (0.5 + 0.25, 1 + 0.25) circle (6pt);
%       \fill (0 + 0.25, 1 + 0.25) circle(4pt);
%       \draw [line width=0.5mm, red] (0 + 0.25, 1 + 0.25) circle (6pt);
%       
% %       \draw (0.5 + 0.25, 3 + 0.25) circle(4pt);
% %       \draw [line width=0.5mm, red] (0.5 + 0.25, 3 + 0.25) circle (6pt);
% %       \draw (0 + 0.25, 3 + 0.25) circle(4pt);
% %       \draw [line width=0.5mm, red] (0 + 0.25, 3 + 0.25) circle (6pt);
% %       \draw (1 + 0.25, 3 + 0.25) circle(4pt);
% %       \draw [line width=0.5mm, red] (1 + 0.25, 3 + 0.25) circle (6pt);
% %       \draw (1 + 0.75, 3 + 0.25) circle(4pt);
% %       \draw [line width=0.5mm, red] (1 + 0.75, 3 + 0.25) circle (6pt);
% %       \draw (2.25, 3 + 0.25) circle(4pt);
% %       \draw [line width=0.5mm, red] (2.25, 3 + 0.25) circle (6pt);
% %       \draw (2.75, 3 + 0.25) circle(4pt);
% %       \draw [line width=0.5mm, red] (2.75, 3 + 0.25) circle (6pt);
% %       \draw (3.25, 3 + 0.25) circle(4pt);
% %       \draw [line width=0.5mm, red] (3.25, 3 + 0.25) circle (6pt);
%       
% %       \draw (0.5 + 0.25, 2.5 + 0.25) circle(4pt);
% %       \draw [line width=0.5mm, red] (0.5 + 0.25, 2.5 + 0.25) circle (6pt);
% %       \draw (0 + 0.25, 2.5 + 0.25) circle(4pt);
% %       \draw [line width=0.5mm, red] (0 + 0.25, 2.5 + 0.25) circle (6pt);
% %       \draw (1 + 0.25, 2.5 + 0.25) circle(4pt);
% %       \draw [line width=0.5mm, red] (1 + 0.25, 2.5 + 0.25) circle (6pt);
% %       \draw (1 + 0.75, 2.5 + 0.25) circle(4pt);
% %       \draw [line width=0.5mm, red] (1 + 0.75, 2.5 + 0.25) circle (6pt);
% %       \draw (2.25, 2.5 + 0.25) circle(4pt);
% %       \draw [line width=0.5mm, red] (2.25, 2.5 + 0.25) circle (6pt);
% %       \draw (2.75, 2.5 + 0.25) circle(4pt);
% %       \draw [line width=0.5mm, red] (2.75, 2.5 + 0.25) circle (6pt);
% %       \draw (3.25, 2.5 + 0.25) circle(4pt);
% %       \draw [line width=0.5mm, red] (3.25, 2.5 + 0.25) circle (6pt);
%       
%       \draw (0.5 + 0.25, 2 + 0.25) circle(4pt);
%       \draw [line width=0.5mm, red] (0.5 + 0.25, 2 + 0.25) circle (6pt);
%       \draw (0 + 0.25, 2 + 0.25) circle(4pt);
%       \draw [line width=0.5mm, red] (0 + 0.25, 2 + 0.25) circle (6pt);
%       \draw (1 + 0.25, 2 + 0.25) circle(4pt);
%       \draw [line width=0.5mm, red] (1 + 0.25, 2 + 0.25) circle (6pt);
%       \draw (1 + 0.75, 2 + 0.25) circle(4pt);
%       \draw [line width=0.5mm, red] (1 + 0.75, 2 + 0.25) circle (6pt);
%       \draw (2.25, 2 + 0.25) circle(4pt);
%       \draw [line width=0.5mm, red] (2.25, 2 + 0.25) circle (6pt);
%       \draw (2.75, 2 + 0.25) circle(4pt);
%       \draw [line width=0.5mm, red] (2.75, 2 + 0.25) circle (6pt);
% %       \draw (3.25, 2 + 0.25) circle(4pt);
% %       \draw [line width=0.5mm, red] (3.25, 2 + 0.25) circle (6pt);
%       
%       \fill (0.5 + 0.25, 1.5 + 0.25) circle(4pt);
%       \draw [line width=0.5mm, green] (0.5 + 0.25, 1.5 + 0.25) circle (6pt);
%       \fill (0 + 0.25, 1.5 + 0.25) circle(4pt);
%       \draw [line width=0.5mm, green] (0 + 0.25, 1.5 + 0.25) circle (6pt);
%       \fill (1 + 0.25, 1.5 + 0.25) circle(4pt);
%       \draw [line width=0.5mm, green] (1 + 0.25, 1.5 + 0.25) circle (6pt);
%       \fill (1 + 0.75, 1.5 + 0.25) circle(4pt);
%       \draw [line width=0.5mm, green] (1 + 0.75, 1.5 + 0.25) circle (6pt);
%       \draw (2.25, 1.5 + 0.25) circle(4pt);
%       \draw [line width=0.5mm, red] (2.25, 1.5 + 0.25) circle (6pt);
%       \draw (2.75, 1.5 + 0.25) circle(4pt);
%       \draw [line width=0.5mm, red] (2.75, 1.5 + 0.25) circle (6pt);
% %       \draw (3.25, 1.5 + 0.25) circle(4pt);
% %       \draw [line width=0.5mm, red] (3.25, 1.5 + 0.25) circle (6pt);
%       
%       \fill (1 + 0.25, 1 + 0.25) circle(4pt);
%       \draw [line width=0.5mm, green] (1 + 0.25, 1 + 0.25) circle (6pt);
%       \fill (1 + 0.75, 1 + 0.25) circle(4pt);
%       \draw [line width=0.5mm, green] (1 + 0.75, 1 + 0.25) circle (6pt);
%       
%       \fill (2.25, 1 + 0.25) circle(4pt);
%       \draw [line width=0.5mm, green] (2.25, 1 + 0.25) circle (6pt);
%       \fill (2.75, 1 + 0.25) circle(4pt);
%       \draw [line width=0.5mm, green] (2.75, 1 + 0.25) circle (6pt);
% %       \draw (3.25, 1 + 0.25) circle(4pt);
% %       \draw [line width=0.5mm, red] (3.25, 1 + 0.25) circle (6pt);
%       
%       \fill (1 + 0.25, 0.5 + 0.25) circle(4pt);
%       \draw [line width=0.5mm, green] (1 + 0.25, 0.5 + 0.25) circle (6pt);
%       \fill (1 + 0.75, 0.5 + 0.25) circle(4pt);
%       \draw [line width=0.5mm, green] (1 + 0.75, 0.5 + 0.25) circle (6pt);
%       \fill (2.25, 0.5 + 0.25) circle(4pt);
%       \draw [line width=0.5mm, green] (2.25, 0.5 + 0.25) circle (6pt);
%       \fill (2.75, 0.5 + 0.25) circle(4pt);
%       \draw [line width=0.5mm, green] (2.75, 0.5 + 0.25) circle (6pt);
% %       \draw (3.25, 0.5 + 0.25) circle(4pt);
% %       \draw [line width=0.5mm, red] (3.25, 0.5 + 0.25) circle (6pt);
%       
%       \fill (1 + 0.25, 0 + 0.25) circle(4pt);
%       \draw [line width=0.5mm, green] (1 + 0.25, 0 + 0.25) circle (6pt);
%       \fill (1 + 0.75, 0 + 0.25) circle(4pt);
%       \draw [line width=0.5mm, green] (1 + 0.75, 0 + 0.25) circle (6pt);
%       \fill (2.25, 0 + 0.25) circle(4pt);
%       \draw [line width=0.5mm, green] (2.25, 0 + 0.25) circle (6pt);
%       \fill (2.75, 0 + 0.25) circle(4pt);
%       \draw [line width=0.5mm, green] (2.75, 0 + 0.25) circle (6pt);
% %       \draw (3.25, 0 + 0.25) circle(4pt);
% %       \draw [line width=0.5mm, red] (3.25, 0 + 0.25) circle (6pt);
%     \end{scope}
%     
% \end{tikzpicture}}
% \caption{Forward indirect edges parameter type \& count policy.}
% \label{Type and parameter count policy.}
% \vspace{-.5cm}
% \end{figure}
% 
% Figure~\ref{Type and parameter count policy.} depicts
% on the $X$ axis (parameter count) and $Y$ axis (register wideness) of matrices $M1$, $M2$ and $M3$ which represent function parameter count 
% and bit-widths in bytes, respectively.
% Note that our type policy performs an $\wedge$ (\textit{i.e.,} logical and) operation
% between each entry in $M1_{i,j}$ and $M2_{i,j}$ where $i$ and $j$ are column and row indexes. 
% If two black filled circles located in $M1$ $\wedge$ $M2$ overlap on positions $M1_{i} = M2_{i} \wedge M1_{j} = M2_{j}$ than we have a match.
% Green circles indicate a match whereas red circles indicate a mismatch in $M3$.
% Only if at least one match (green circle) is present in each of the matrix columns of 
% $M3$ than the indirect call transfer will be allowed.
% Further, Figure~\ref{Type and parameter count policy.} highlights
% the behavior of our type based policy
% when the callsite provides 6 parameters $\langle pcs1, ..., pcs6 \rangle$ having following bit 
% wideness \textit{pcs}1: 4-byte, \textit{pcs}2: 4-byte, \textit{pcs}3: 4-byte, \textit{pcs}4: 8-byte, \textit{pcs}5: 2-byte, 
% \textit{pcs}6: 2-byte, and the calltarget is expecting 6 parameters $pct1, ..., pct6$ having following bit 
% wideness \textit{pct}1: 4-byte, \textit{pct}2: 4-byte, \textit{pct}3: 0-byte, \textit{pct}4: 0-byte, \textit{pct}5: 0-byte, 
% \textit{pct}6: 0-byte of the expected parameters. 
% TypeArmor's parameter count policy is the following.
% 
% \begin{definition}
%  \label{eqn:2}Let $A$ be a calltarget $ct_{A}$ and $B$ a callsite $cs_{B}$ than: 
% $ct_{A} \subseteq cs_{B} \iff \forall i \subseteq [1, 6],$ 
% $count(parameter($A$))$ $\leq$ $count(parameter($B$))$.
% \end{definition}
% The forward-edge policy of \textsc{TypeShield} is as follows.
% \begin{definition}
% \label{eqn:1} Let $A$ be a calltarget $ct_{A}$ and $B$ a callsite $cs_{B}$ than: 
% $ct_{A}$ $\subseteq$ $cs_{B}$ $\iff$ $\forall$ $i$ $\subseteq$ $[1, 6],$
% $wideness$ $(parameter($A$)[i])$ $\leq$ \ $wideness (parameter($B$)[i])$.
% \end{definition}
% 
% However, one can observe that the first policy (Definition~\ref{eqn:2}) offers less precision w.r.t. forward edge mapping on the legitimate target set
% than the second policy (Definition~\ref{eqn:1}). Note that the first policy performs 
% only parameter count checks whereas the second policy checks for each parameter index in part separately w.r.t. count and register wideness.
% 
% \subsection{Backward Edge Policy}
% \label{Backward Edge CFI Policy}
% \textsc{TypeShield} uses a backward edge (\textit{i.e.,} \texttt{retn}) fine-grained CFI protection policy which 
% relies on enforcing the legitimate forward edge addresses after each callsite
% to each calltarget return address (\textit{i.e.,} function return address). 
% This corresponds to the caller-callee calling convention 
% which basically enforces that each function should return to the next address after the callsite which 
% was used to call that function before. \textsc{TypeShield} provides three modes of operation for
% protecting the backward-edge policy. This modes of operation will be presented in section~\cref{chapter:Design}.

% \subsection{Parameter Count vs. Parameter Type}
% % \textbf{\textsc{TypeShield} Impact on COOP.}
% \label{TypeShild Impact on COOP}
% % \vspace{-.5cm}
% \begin{figure}[h!]
% \centering
% \resizebox{0.28\textwidth}{!}{
% \begin{tikzpicture}
% \draw[thick] (0,-3) [blue] rectangle node[anchor=center]  (0)  {\HUGE{$\bot$ (0, 0)}}   (2,-4);
% \draw[thick] (0,0) rectangle node[anchor=center]  (320)  {\HUGE{(a, 0)}}  (2,-1);
% \draw[thick] (3,2) rectangle node[anchor=center]  (640)  {\HUGE{(b, 0)}}  (5,1);
% \draw[thick] (0,3) rectangle node[anchor=center]  (3232) {\HUGE{(a, a)}} (2,4);
% \draw[thick] (3,5) rectangle node[anchor=center]  (6432) {\HUGE{(b, a)}} (5,6);
% \draw[thick] (-1,5) rectangle node[anchor=center] (3264) {\HUGE{(a, b)}} (-3,6);
% \draw[thick] (0,7) [blue] rectangle node[anchor=center]  (6464) {\HUGE{$\top$ (b, b)}} (2, 8);
% 
%   %%TA and TS
%   \draw[draw, -triangle 45, thick] (3264.north) -- (6464.south);
%   \draw[draw, -triangle 45, thick] (6432.north) -- (6464.south);
%   \draw[draw, -triangle 45, thick] (3232.north) -- (3264.south);
%   \draw[draw, -triangle 45, thick] (3232.north) -- (6432.south);
%   
%   \draw[draw, -triangle 45, thick] (640.north) -- (6432.south);
%   \draw[draw, -triangle 45, thick] (320.north) -- (3232.south);
%   \draw[draw, -triangle 45, thick] (320.north) -- (640.south);
%   \draw[draw, -triangle 45, thick] (0.north) -- (320.south);
%   
%   %%only TA
%   \draw[draw, color=red, dotted, -triangle 45, thick] (640.south) -- (320.east);
%   \draw[draw, color=red, by applying dotted, -triangle 45, thick] (640.north) -- (3232.south);
%   
%   \draw[draw, color=red, dotted, -triangle 45, thick]  (6464.south) -- (3264.east);
%   \draw[draw, color=red, dotted, -triangle 45, thick]  (6464.south) -- (6432.west);
%   \draw[draw, color=red, dotted, -triangle 45, thick]  (3264.east) -- (3232.north);
%   \draw[draw, color=red, dotted, -triangle 45, thick]  (6432.west) -- (3232.north);
%   
%   %%draw boxes arround
%   \draw[thick,dashed] (2.1,0.6) ellipse (4.5cm and 2cm);
%    \begin{scope}[xshift=6cm, yshift=-.3cm]
%      \node[]    (q_1) {\HUGE{1 parameter}};
%    \end{scope}
%    
%    \draw[thick,dashed] (1,5.5) ellipse (4.8cm and 2.8cm);
%    \begin{scope}[xshift=6cm, yshift=6.8cm]
%      \node[]    (q_2) {\HUGE{2 parameters}};
%    \end{scope}
%    
%    
%    
%    \begin{scope}[xshift=5cm, yshift=-1.9cm]
%      \node[]    (q_4) {\HUGE{\textit{Legend}}};
%    \end{scope}
%    
%    \begin{scope}[xshift=5.1cm, yshift=-2.7cm]
%      \node[]    (q_4) {\HUGE{allowed}};
%      \draw[draw, -triangle 45, thick]  (1.2,-.02) -- (2.2,-.02);
%    \end{scope}
%    
%    \begin{scope}[xshift=5.1cm, yshift=-3.4cm]
%      \node[]    (q_3) {\HUGE{foridden}};
%      \draw[draw, color=red, dotted, -triangle 45, thick]  (1.2,-.05) -- (2.2,-.05);
%    \end{scope}
%    
%    \draw (3.8,-2.3) -- (7.4,-2.3) -- (7.4,-3.7) -- (3.8,-3.7) -- (3.8,-2.3);
%    
% \end{tikzpicture}
% 
% }
% \caption{Calltargets \& callsites transition lattice with 2 params.}
% \label{fig:lattice3264}
% \end{figure}
% Figure~\ref{fig:lattice3264} depicts with $a$ $\wedge$ $b$ $\in \{0 \ byte, 8 \ byte, 16 \ byte, $ $32 \ byte, 64 \ byte\}$ two 
% function parameters that have $\{0 \ byte, 1 \ byte, 2 \ byte, 4 \ byte, 8 \ byte\}$ register wideness. 
% \textsc{TypeShield} allows a transition from $a \rightarrow b \ iff$ $a_{i} \le b_{i}$ where $i \in [1, 2]$.
% Note that $\top$ and $\bot$ represent the top and bottom elements of the lattice, respectively.
% An arrow represents an indirect control flow transfer from a callsite to a calltarget. 
% The given lattice contains in total 8 black colored arrows (legal) and 6 red colored arrows (illegal) indirect control flow transitions. 
% \textsc{TypeShield} allows only the legal transfers whereas~\cite{veen:typearmor} allows all of them.
% Further, Figure~\ref{fig:lattice3264} depicts
% a subset of the total indirect control flow transfer space in any given C/C++ program represented as a lattice. 
% In case a CFI policy schema is based on function parameter count with callsite overestimation and calltarget subestimation 
% it is possible that a callsite can use any calltarget as long as the number of 
% parameters provided and required are fulfilling the policy, even if the parameter types do not match 
% (\textit{i.e.,} consider a 8-bit value provided by the callsite but a 64-bit values required by the calltarget). 
% Such a parameter count based policy is not precise~\cite{vci:asiaccs} and would allow any call transfer 
% inside the lattice space presented in Figure~\ref{fig:lattice3264} and as such the calltarget set per 
% callsite would be too permissive.
% 
% In order to effectively deal with this limitation we extend the above presented parameter count based policy 
% to parameters types (\textit{i.e.,} register wideness) as well. We introduce the following policy rules: 
% (1) indirect callsites provide a maximum wideness to each parameter, and
% (2) calltargets require a minimum wideness for each parameter. 
% Note that for both rules the minimum and maximum wideness for each function 
% parameter is possibly underestimated compared to the source code of the program with which we 
% also compare in \cref{chapter:Evaluation}.
% Also note that the number of provided parameters must be not lower than the required number of consumed parameters. 
% Finally, our approach is more fine-grained by considering parameter wideness and as such the allowed calltarget lattice 
% space is considerably reduced since only the black arrows are allowed.



% The result is that we split the buckets of TypeArmor up into smaller ones as shown 
% in the limited example depicted in Figure \ref{fig:lattice3264}.
% There we can see that while in a parameter-count oriented schema a callsite classified as (32,32) would be able to 
% call functions classified as (64,0), however in our parameter wideness oriented schema that is not possible.


