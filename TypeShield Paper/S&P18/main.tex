
% \documentclass[10pt]{IEEEtran}
\documentclass[twocolumn]{IEEEtran} %!PN
% \documentclass[conference, 10pt, times]{IEEEtran}
% \usepackage{usenix,epsfig,endnotes}
% \usepackage{flushend}
\usepackage{tikz}
\usepackage{pgfplots}
\newcommand*\circled[1]{\tikz[baseline=(char.base)]{
            \node[shape=circle,draw,inner sep=1pt] (char) {#1};}}
\usetikzlibrary{arrows,positioning} 
\tikzset{
    %Define standard arrow tip
    >=stealth',
    %Define style for boxes
    punkt/.style={
           rectangle,
           rounded corners,
           draw=black, very thick,
           text width=6.5em,
           minimum height=2em,
           text centered},
    % Define arrow style
    pil/.style={
           ->,
           thick,
           shorten <=2pt,\textsc{}
           shorten >=2pt,}
}
\usepackage{verbatim}
\usepackage{amsmath}
\usepackage[ruled,vlined,linesnumbered,resetcount]{algorithm2e}
\SetAlFnt{\scriptsize\sffamily}
\SetKwComment{Comment}{$\triangleright$\ }{}
\SetEndCharOfAlgoLine{}

\makeatletter
\newcommand\thefontsize[1]{{#1 The current font size is: \f@size pt\par}}
\makeatother

% \SetEndCharOfAlgoLine{}
\newcommand\mycommfont[1]{\scriptsize\ttfamily\textcolor{blue}{#1}}
\SetCommentSty{mycommfont}

\usepackage{array}
\usepackage[]{xcolor}
\usepackage{ifthen}
\usepackage{makecell}
\usepackage{booktabs}
\usepackage[disable]{todonotes}
\usepackage{graphicx}
% \usepackage{todonotes}
\usepackage[T1]{fontenc}
\usepackage{tikz}
\usepackage[utf8]{inputenc}
\usepackage{svg}
\usepackage{etex}
\usepackage{svg}
\usepackage{amsmath,amssymb}
\usetikzlibrary{matrix,arrows}
\usetikzlibrary{positioning,arrows}
\usetikzlibrary{shapes,arrows,fit,calc,positioning,automata}
\PassOptionsToPackage{hyphens}{url}\usepackage{hyperref}

\usepackage{fix-cm}    
\usepackage[flushleft]{threeparttable} % http://ctan.org/pkg/threeparttable
\let\labelindent\relax
\usepackage{enumitem}

\makeatletter
\newcommand\HUGE{\@setfontsize\Huge{19}{19}}
\makeatother 

\usepackage{xcolor}
\hypersetup{
    colorlinks=true,
    linkcolor=red,
    urlcolor=blue,
    citecolor=blue
}

%use to check the not used references
%and remove them
% \usepackage{refcheck}

%draft watermark
% \usepackage{draftwatermark}
% \SetWatermarkText{DRAFT}
% \SetWatermarkScale{1}

\inputencoding{latin1}
\inputencoding{utf8}

\usepackage{fancyhdr,import}
% \usepackage[colorlinks=true,linkcolor=blue, citecolor=blue]{hyperref}
\usepackage{cleveref}
\crefname{section}{§}{§§}
\Crefname{section}{§}{§§}
\crefformat{section}{§#2#1#3}

\usepackage{balance}
\usepackage{setspace}
\usepackage{adjustbox}
\usepackage{lipsum}% dummy text
\usepackage{longtable,csvsimple}
% \usepackage[margin=.75in]{geometry}
% \usepackage[toc,page]{appendix}
\renewcommand{\UrlBreaks}{\do.\do\/\do\a\do\b\do\c\do\d\do\e\do\f\do\g\do\h\do\i\do\j\do\k\do\l\do\m\do\n\do\o\do\p\do\q\do\r\do\s\do\t\do\u\do\v\do\w\do\x\do\y\do\z\do\A\do\B\do\C\do\D\do\E\do\F\do\G\do\H\do\I\do\J\do\K\do\L\do\M\do\N\do\O\do\P\do\Q\do\R\do\S\do\T\do\U\do\V\do\W\do\X\do\Y\do\Z}
\def\UrlBreaks{\do\/\do-}
 
\makeatletter
\g@addto@macro{\UrlBreaks}{%
  \do\/\do\a\do\b\do\c\do\d\do\e\do\f%
  \do\g\do\h\do\i\do\j\do\k\do\l\do\m%
  \do\n\do\o\do\p\do\q\do\r\do\s\do\t%
  \do\u\do\v\do\w\do\x\do\y\do\z%
  \do\A\do\B\do\C\do\D\do\E\do\F\do\G%
  \do\H\do\I\do\J\do\K\do\L\do\M\do\N%
  \do\O\do\P\do\Q\do\R\do\S\do\T\do\U%
  \do\V\do\W\do\X\do\Y\do\Z}
\g@addto@macro\UrlBreaks{\do\-\.}
\makeatother
  
%%algorithm stuff
\makeatletter
\providecommand{\bigsqcap}{%
  \mathop{%
    \mathpalette\@updown\bigsqcup
  }%
}
\newcommand*{\@updown}[2]{%
  \rotatebox[origin=c]{180}{$\m@th#1#2$}%
}
\makeatother

\makeatletter
\let\oldnl\nl% Store \nl in \oldnl
\newcommand{\nonl}{\renewcommand{\nl}{\let\nl\oldnl}}% Remove line number for one line
\makeatother

\usepackage{graphicx} % http://ctan.org/pkg/graphicx
\usepackage{booktabs} % http://ctan.org/pkg/booktabs
\usepackage{xparse}   % http://ctan.org/pkg/xparse
% Rotation: \rot[<angle>][<width>]{<stuff>}
\NewDocumentCommand{\rot}{O{45} O{2em} m}{\makebox[#2][l]{\rotatebox{#1}{#3}}}%
\selectlanguage{english}
% \let\oldalgorithm\algorithm
% \let\oldendalgorithm\endalgorithm
% 
% \let\algorithm\figure
% \let\endalgorithm\endfigure

%%% Allow more line breaks in URLs.
\usepackage{url}
\makeatletter
\g@addto@macro{\UrlBreaks}{\UrlOrds}
\makeatother

%%my own dings
\usepackage{tikz}
\newcommand\encircle[1]{%
\tikz[yshift=-2pt] 
   \node (X) [draw, shape=circle, inner sep=0, scale=0.5, fill=black, text=white] {\strut #1};}
   
%% add -shell-escape as flag for pdflatex in order to work
\usepackage{minted}
\usepackage{placeins}
\usepackage{float}
% \usepackage[outputdir=build]{minted}

\usetikzlibrary{arrows,positioning} 
\tikzset{
    %Define standard arrow tip
    >=stealth',
    %Define style for boxes
    punkt/.style={
           rectangle,
           rounded corners,
           draw=black, very thick,
           text width=6.5em,
           minimum height=2em,
           text centered},
    % Define arrow style
    pil/.style={
           ->,
           thick,
           shorten <=2pt,
           shorten >=2pt,}
}

\usepackage{listings}
\lstdefinelanguage
   [x64]{Assembler}     % add a "x64" dialect of Assembler
   [x86masm]{Assembler} % based on the "x86masm" dialect
   % with these extra keywords:
   {morekeywords={CDQE,CQO,CMPSQ,CMPXCHG16B,JRCXZ,LODSQ,MOVSXD, %
                  POPFQ,PUSHFQ,SCASQ,STOSQ,IRETQ,RDTSCP,SWAPGS, %
                  movaps,
                  rax,rdx,rcx,rbx,rsi,rdi,rsp,rbp, %
                  r8, r14, r15, r15d, r8d,r8w,r8b,r9,r9d,r9w,r9b}} % etc.

\lstset{language=[x64]Assembler}


\usepackage{amsthm}
\usepackage[htt]{hyphenat}
\usepackage{svg}
\usepackage{pifont}
\theoremstyle{definition}
\newtheorem{definition}{Definition}
% \newtheorem*{remark}{Remark}
% \newtheorem{theorem}{Theorem}[section]
% \newtheorem{proposition}{Theorem}[section]
% \newtheorem{corollary}{Corollary}[theorem]
% \newtheorem{lemma}[theorem]{Lemma}

\hyphenation{op-tical net-works semi-conduc-tor}
\bibliographystyle{plain}
\fancypagestyle{myplain}
{
  \fancyhf{}
  \renewcommand\headrulewidth{0pt}
  \renewcommand\footrulewidth{0pt}
  \fancyfoot[C]{\thepage}
}
\fancypagestyle{myfancy}{
  \fancyhf{}
  \fancyhead[CO]{\nouppercase\leftmark}
  \fancyhead[CE]{\hdrtitle}
  \fancyhead[LE,RO]{\thepage}
  \renewcommand\headrulewidth{0.4pt}
  \pagestyle{fancy}
  \renewcommand\sectionmark[1]{\markboth{##1}{}}%don't move this
}
\definecolor{bostonuniversityred}{rgb}{0.8, 0.0, 0.0}
\definecolor{cadmiumgreen}{rgb}{0.0, 0.42, 0.24}
\definecolor{lightgray}{rgb}{0.83, 0.83, 0.83}

\begin{document}
%%used to avoid using --- to replace author names of similar entries
% \bstctlcite{IEEEexample:BSTcontrol}
%don't want date printed
% \date{}

%make title bold and 14 pt font (Latex default is non-bold, 16 pt)
% \title{\Large \bf \textsc{TypeShield}: Precise Protection of Forward Indirect Calls in Binaries}
% \title{\Large \bf \textsc{TypeShield}: Protecting Forward Indirect Calls in C++ Binaries for Real}
% \title{\Large \bf \textsc{TypeShield}: Practical, Precise and Effective Protection of Forward Indirect Calls in C++ Binaries}
% \title{\Large \bf \textsc{TypeShield}: Practical, Precise \& Effective Protection of Forward Indirect Calls}
% \title{\textsc{TypeShield}: Practical Forward \& Backward Edge Defense Against Code Reuse Attacks using Binary Type Information}
\title{\textsc{TypeShield}: Practical Defense Against Code Reuse Attacks using Binary Type Information}


% author names and affiliations
% use a multiple column layout for up to three different
% affiliations

% \authorinfo{tba.}
% \author{Paul Muntean, Matthias Fischer, Andre Rein, Jens Grossklags, Claudia Eckert \\ Technical University of Munich}
\author{tba.}

% Use the following at camera-ready time to suppress page numbers.
% Comment it out when you first submit the paper for review.

\maketitle
%page nymbering option
\thispagestyle{myplain}
\pagestyle{myplain}
\pagenumbering{arabic}
\begin{abstract}
%long version
% Applications aiming for high performance and availability draw on several features in the C/C++ programming language. 
% A key building block are virtual functions, which facilitate late binding, and thereby support runtime polymorphism. 
% However, practice-driven and academic research have identified an alarmingly high number of virtual pointer corruption 
% vulnerabilities which undercut security in significant ways and are still in need of a thorough solution approach.
% 
% We contribute to this research area by proposing \textsc{TypeShield}, a binary runtime virtual pointer protection tool 
% which is based on instrumentation of program executables at load time. \textsc{TypeShield} applies a novel runtime 
% type and function parameter counter control-flow integrity (CFI) policy in order to overcome the limitations of available approaches and to 
% efficiently verify dynamic dispatches during runtime. To enhance practical applicability, \textsc{TypeShield} can 
% be automatically and easily used in conjunction with legacy applications or where source code is missing to harden 
% binaries.
% We have applied \textsc{TypeShield} to web servers, FTP servers and the SPEC CPU2006 benchmark and were able to 
% efficiently and with low performance overhead protect these applications from forward indirect edge corruptions 
% based on virtual pointers. Further, in a direct comparison with the state-of-the-art tool, \textsc{TypeShield} 
% achieves higher caller/callee matching (\textit{i.e.,} precision), while maintaining a more favorable 
% runtime performance overhead ($\approx$ 4\%) than other state-of-the-art tools.
% Focusing the evaluation on target reduction techniques, we can demonstrate that our approach achieves a notable 
% additional reduction of the possible calltargets per callsite of up to 35\% associated with an overall reduction
% of about 13\% and a comparable runtime performance overhead as other state-of-the-art parameter-only count-based tools.
% Finally, we want to particularly emphasize that in this paper we provide for each experiment a precise description
% w.r.t. setup, results, tool misses, mean, median and geomean values which clearly increases the reproducibility.

%short version
% Applications aiming for high performance and availability draw on several object oriented 
% features available in the C/C++ programming language such dynamic object dispatch.
% However, there is an alarmingly high number of object dispatch (\textit{i.e.,} forward-edge) corruption vulnerabilities which undercut security 
% in significant ways and are in need of a thorough solution.

We propose, \textsc{TypeShield}, a binary runtime forward-edge and backward-edge protection tool 
which instruments program executables at load time. \textsc{TypeShield} enforces a novel runtime 
control-flow integrity (CFI) policy based on function parameter type and count in order to overcome the limitations of available approaches and to 
efficiently verify dynamic object dispatches and function returns during runtime. To enhance practical applicability, \textsc{TypeShield} can 
be automatically and easily used in conjunction with legacy applications or where source code is missing to harden binaries.
We evaluated \textsc{TypeShield} on highly relevant open source programs and the SPEC CPU2006 benchmark and were able to 
efficiently and with low performance overhead protect these applications from forward-edge and backward-edge corruptions.
Finally, in a direct comparison with state-of-the-art tools, \textsc{TypeShield}
achieves higher caller/callee matching precision, while maintaining a
low runtime overhead.


%  % Abstract for the TUM report document
% Included by MAIN.TEX


\clearemptydoublepage
\phantomsection
\addcontentsline{toc}{chapter}{Abstract}	

\vspace*{2cm}
\begin{center}
{\Large \bf Abstract}
\end{center}
\vspace{1cm}

An abstracts abstracts the thesis!

This is just an example that shows how long the abstract should be
and which parts an abstract should contain. \\


Many applications such as the Chrome and Firefox
browsers are largely implemented in C++ for its perfor-
mance and modularity. Type casting, which converts one
type of an object to another, plays an essential role in en-
abling polymorphism in C++ because it allows a program
to utilize certain general or specific implementations in
the class hierarchies. However, if not correctly used, it
may return unsafe and incorrectly casted values, leading
to so-called bad-casting or type-confusion vulnerabili-
ties. Since a bad-casted pointer violates a programmer’s
intended pointer semantics and enables an attacker to
corrupt memory, bad-casting has critical security implica-
tions similar to those of other memory corruption vulner-
abilities. Despite the increasing number of bad-casting
vulnerabilities, the bad-casting detection problem has not
been addressed by the security community.

In this paper, we present C A V ER , a runtime bad-casting
detection tool. It performs program instrumentation
at compile time and uses a new runtime type tracing
mechanism—the type hierarchy table—to overcome the
limitation of existing approaches and efficiently verify
type casting dynamically. In particular, C A V ER can be
easily and automatically adopted to target applications,
achieves broader detection coverage, and incurs reason-
able runtime overhead. We have applied C A V ER to large-
scale software including Chrome and Firefox browsers,
and discovered 11 previously unknown security vulnera-
bilities: nine in GNU libstdc++ and two in Firefox, all
of which have been confirmed and subsequently fixed by
vendors. Our evaluation showed that C A V ER imposes up
to 7.6\% and 64.6\% overhead for performance-intensive
benchmarks on the Chromium and Firefox browsers, re-
spectively.
\end{abstract}

% \keywords{C++ object dispatch, indirect call, forward edge, code reuse attack} % TODO: replace with your keywords


%contents
\section{Introduction}
\label{chapter:Introduction}

% \textbf{Big Picture.}
The object-oriented programming (OOP) paradigm is \textit{de facto standard} concept for developing large, complex and efficient
systems because it facilitates inheritance for objects which seem to be at first not related to each other; this in turn 
facilitates better code reuse, software maintenance and software design. There are many programing languages
which support OOP concepts, however the C++ programing language is the most used programming language for systems where 
runtime performance and reliability are the main goal.

An important C++ OOP convention is the object calling convention when for instance virtual functions are called.
Virtual functions are an important concept which facilitates late binding and allows the programmer to overwrite a
virtual function of the base-class with his own implementation. In fact, in order to implement virtual functions, 
the compiler needs to generate a table (\textit{i.e.,} virtual table meta-data structure) of all
virtual functions for each class containing them and provide to each instance of such a class (\textit{i.e.,} object) a pointer (\textit{i.e.,} virtual pointer)
to the aforementioned table. While this allows more flexible code to be built---through late binding---it is simplistically implemented with no security
mechanisms at in place.

\begin{figure}[t!]
\centering
\begin{tikzpicture}
\hspace{-.5cm}
  \node (img)  {\includegraphics[scale=0.75]{figures/distribution2.pdf}};
%   \node[below=of img, node distance=0cm, yshift=1.2cm, xshift=.25cm, font=\color{black}] {Year};
%   \node[left=of img, node distance=0cm, rotate=90, anchor=center,yshift=-.9cm,font=\color{black}] {\# Bad Type Casts Reported};
 \end{tikzpicture}
\vspace{-.5cm}
\caption{\# of type confusions reported by the NIST NVD for the last years. 
As of May'17, 124 type confusion bugs where reported in applications such as Google's Chrome and V8 JS engine; Mozilla Firefox; Microsoft's IE 10, Edge and Chakra JS engine; Adobe Flash Player;
\& iOS/MacOS apps.}
\label{typeconfusion:nvd:statistics}
\vspace{-.7cm}
\end{figure}

% \textbf{Problem.}
The performance benefit of late binding comes with high security implications. 
First, dangling object pointers lead to undefined behavior---as specified by the C++ language standard N4618~\cite{N4618}---which 
can be exploited to point into illegal (\textit{i.e.,} not previously intended) virtual tables.
Second, trough memory corruptions (\textit{e.g.,} buffer/integer overflows) the virtual pointer of an object making a 
call to a virtual function can be corrupted to point into:
\textit{1)} illegal virtual tables, 
\textit{2)} newly inserted virtual tables, or
\textit{3)} overwritten virtual table entries
such that advanced Code-Reuse Attacks (CRAs) as the advanced COOP~\cite{schuster:coop} attack
and its extensions~\cite{crane:readactor++, crane:readactor++, subversive-c:lettner, ctf:coop, loop:oriented} 
become easily doable. This type of attack can bypass most of the to date CFI-based enforcement policies, since:
\textit{1)} it does not exploit indirect backward edges (\textit{i.e.,} return edges) but rather
\textit{2)} it exploits the forward indirect control flow transfers imprecision which can not be statically upfront 
determined since alias analysis is undecidable~\cite{alias:undecidable} in program binaries.

% \textbf{Available Tools.}
To avoid object dispatch corruptions Control-Flow Integrity (CFI)~\cite{abadi:cfi2, abadi:cfi} can be successfully used.
CFI is one of the most used techniques for securing indirect control flow transfers inside programs
by usually adding runtime checks before each indirect call site.

Source code based tools usually insert runtime checks during the compilation of 
the program such as SafeDispatch~\cite{safedispatch:jang}, ShrinkWrap \cite{haller:shrinkwrap} and IFCC/VTV~\cite{vtv:tice}.
Other tools modify and reorder the contents of the virtual table layout such as VTI~\cite{bounov:interleaving} 
in order to derive efficient range checks on each object dispatch during runtime. Nevertheless, to the best of our knowledge 
due to runtime performance issues only IFCC/VTV~\cite{vtv:tice} is currently in production available.

Binary based tools typically enforce imprecise forward-edge CFI 
policies, often allowing control transfers from any valid call site 
to any valid referenced entry point \textit{e.g.,} binCFI~\cite{ccfir:zhang, zhang:usenix}. 
In the best case, existing policies only reduce the target set by
removing all entry points of other modules unless they were
explicitly exported or observed at runtime~\cite{payer:dimva}. 

TypeArmor~\cite{veen:typearmor} implements a fine grained forward edge CFI 
policy based on parameter count for binaries. It calculates invariants for call targets and indirect call sites based on
the number of parameters they use by leveraging static analysis of the binary, which then is
patched to enforce those invariants during runtime. 
The main shortcoming of TypeArmor is that it has low precision 
w.r.t. to the number of call targets allowed per call site 
(see \cref{{Too Permissive Parameter-Based Policies}} for more details).

VCI~\cite{vci:asiaccs} is a binary rewriting tool that can protect C++ binaries against 
vtable attacks. VCI strives to reconstruct several language semantics from the binary with limited success.
These will be later on used for a CFI policy based on resolving pairings of virtual table calls (vcall)
with precise sets of target classes. The policy is enforced similarly to TypeArmor by inserting the 
needed checks before each virtual call. VCI performs a restricted type of alias analysis during type propagation.
Also, it fails in some situations to identify the class types used by a vcall. Additionally, VCI can not deal with 
virtual-dispatch-like C calls and it fails to find any
constructor that defines the \textit{this} pointer. Overall, VCI tries to recuperate many high level 
semantics without focusing on one of them from the binary. As result the call target set per call site is too 
permissive.

% \textbf{Tool Limitations.}
These source code tools offer a certain degree of protection when code is provided, however 
the above mentioned binary tools offer limited or no protection due to an in first place
imprecise calltarget set per callsite.

% \textbf{Our Idea.}
In this paper, we present \textsc{TypeShield}, a runtime binary-level illegitimate forward calls 
filtering tool that is based on an improved forward-edge fine-grained CFI policy compared 
to previous work~\cite{veen:typearmor, crane:readactor++}.
\textsc{TypeShield} analyzes only 64-bit binaries and only function parameters 
which are passed with the help of registers. This means that based on the 
used ABI, \textsc{TypeShield} is able to track 4 or 6 arguments for the Microsoft's x64-bit calling convention
or System V ABI, respectively. Similarly to TypeArmor we do not take into consideration floating-point 
arguments passed via xmm registers; which we want to address in future work. However, as we will 
demonstrate in the evaluation section, this will provide us enough information to 
more be precisely than TypeArmor when stopping several state-of-the-art CRAs.

More precisely, the analysis performed by \textsc{TypeShield}:
\textit{1)} uses for each function parameter its register wideness (\textit{i.e.,} ABI dependent) in order to map calltargets per callsites,  
\textit{2)} uses an address taken (AT) analysis similar to~\cite{veen:typearmor} for all calltargets, and 
\textit{3)} compares individually parameters of callsites and calltargets in order to check if an indirect call transfer is acceptable or not, 
thus this providing a more fine-grained calltarget set per callsite than other state-of-the-art tools.
\textsc{TypeShield} is based on a use-def callees analysis to approximated the function prototypes, 
and liveness analysis at indirect callsites to approximate callsite signatures. This 
efficiently leads to a more precise CFG of the binary program in question, 
which can be used also by other systems in order to gain a more precise CFG on which to 
enforce other types of CFI related policies.
\textsc{TypeShield} incorporates an improved protection policy which is
based on the insight that if the binary adheres to the standard calling convention
for indirect calls, undefined arguments at the call site are not used by any callee by design. 
This further helps to reduce the possible target set of callees for each callsite.
\textsc{TypeShield} relies on a more precise than TypeArmor construction of both the callee parameter types and call site signatures.
\textsc{TypeShield} uses automatically inferred parameter types which are later used into the classification of matching call sites and call targets.
This helps to obtain more precise callee target sets for each caller as the TypeArmor.
\textsc{TypeShield} compared to TypeArmor uses different analysis strategies for basic block merging.
Furthermore, \textsc{TypeShield} disallows an indirect call transfer that prepares
fewer arguments than the target callee consumes and where the types of the 
arguments provided are not super types of the arguments expected at the target.
It then uses this information to enforce that each call site targets only a strict call target set.
\textsc{TypeShield} takes the binary of a program as input and it automatically instrument it in order
to detect illegitimate indirect calls at runtime. 
More precisely, \textsc{TypeShield} achieves three goals.

\textbf{Precision.} \textsc{TypeShield} employs a more precise analysis than TypeArmor in order to reduce the call target set for each 
               call site. Our evaluation shows that \textsc{TypeShield} incurs X\% precision w.r.t. TypeArmor on the same programs.

\textbf{Performance.} \textsc{TypeShield} employs runtime policy optimization techniques to further reduce the runtime overheads
              Our evaluation shows that \textsc{TypeShield} imposes up to X\% and X\% overheads for performance-intensive benchmarks on the
              SPEC CPU2006 benchamrks and the webser applications, respectively. On
              the contrary, TypeArmor is X\% slower than \textsc{TypeShield}
              on the x Program.
              
\textbf{Scope.} \textsc{TypeShield} can detect forbidden indirect calls and as such it can protect similarly as vTrust~\cite{zhang:vtrust} against
              virtual table injection, corruption and reuse attacks. As such \textsc{TypeShield} can serve as a platform for developing other types of defenses for different 
              types of attacks.

% \textbf{Contributions.} 
In summary, we make the following contributions:
\label{Contribution}
\begin{itemize}
 \item \textbf{Security analysis of forward indirect calls.} 
 We analyzed the usage of illegitimate indirect forward calls in detail,
 thus providing security researchers and practitioners a better understanding of this emerging
 threat.

 \item \textbf{Illegitimate indirect calls detection tool.}
 We designed and implemented \textsc{TypeShield}, a general, automated, and easy to deploy tool
 that can be applied to C/C++ binaries in order to detect and mitigate illegitimate forward indirect calls 
 during runtime. 
 
 \item \textbf{Experiments.} We demonstrate trough extensive experiments that our precise
 binary-level CFI strategy can mitigate advanced code reuse attacks in absence of C++ semantics.
 For example \textsc{TypeShield} can protect against the COOP attack and its variations.
\end{itemize}

\newsavebox{\firstlisting}
\begin{lrbox}{\firstlisting}
\begin{minipage}[c]{\linewidth}
\begin{minted}[
% frame=lines,
framesep=2mm,
linenos,
frame=none,
firstnumber=1,
framesep = 1.0cm,
linenos,
numbersep=5pt,
%gobble=2,
%frame=lines,
framesep=2mm,
%fontsize=\tiny        
% baselinestretch=1.2,
% bgcolor=LightGray,
fontsize=\small,
]{C++}
class nsMultiplexInputStream final 
 :public nsIMultiplexInputStream //A0
 ,public nsISeekableStream //A1
 ,public nsIIPCSerializableInputStream //A2
 ,public nsICloneableInputStream{ //A3
nsTArray<nsCOMPtr<nsIInputStream>> mStreams;
NS_IMETHODIMP nsMultiplexInputStream::Close(){
  MutexAutoLock lock(mLock);
  mStatus = NS_BASE_STREAM_CLOSED;
  //set NS_OK flag
  nsresult rv = NS_OK;
  //get array length
  uint32_t len = mStreams.Length();
  //array-based main loop gadget (ML-G)
 for (uint32_t i = 0; i<len; ++i){
  //(1) hijacked indirect call
  nsresult rv2=mStreams[i]->Close();
  if (NS_FAILED(rv2)) {
      rv = rv2;
  }
 }
  return rv;
}
\end{minted}
\end{minipage}
\end{lrbox}

% \begin{figure}
%  \begin{minipage}[!t]{.40\linewidth}
%   \usebox{\firstlisting}
%  \end{minipage}%%
% \hfill
% \hspace{1.2cm}
% \begin{minipage}[!b]{.5\linewidth}
%    \includegraphics[width=1.3\textwidth]{figures/loop.pdf}
% \end{minipage}
% \caption{Code example used to illustrate how a COOP loop gadget works.}
% \label{Code example used to illustrate how a COOP loop gadget works}
% \end{figure}

%%%%%%%%%%%%%%%%%
 \begin{figure*}[!t]
    \centering
   \setlength{\unitlength}{0.1\textwidth}
   \begin{picture}(10,4)
%    \centering
     \put(4.81,0){\includegraphics[width=.43\textwidth]{figures/loop.pdf}}
     \put(1.5,2){\usebox{\firstlisting}}
   \end{picture}
\caption{Description of how a counterfeit object-oriented programming main loop gadget (ML-G) works.}
\label{Code example used to illustrate how a COOP loop gadget works}
\end{figure*}

%most probably not needed at this time.
% \label{Outline}
% \textbf{Outline.} 
The remainder of this paper is organized as follows.
\cref{C++ Bad Forward Indirect Calls} explains forbidden forward indirect calls issues and their security implications, and 
\cref{chapter:TypeShild Overview} contains an overview of \textsc{TypeShield}.
\cref{chapter:Design} describes the theory used and decisions made during the design of \textsc{TypeShield}, and
\cref{chapter:Implementation} briefly presents the implementation details of \textsc{TypeShield}, while
\cref{chapter:Evaluation} evaluates several properties of \textsc{TypeShield}.
\cref{chapter:Discussion} contains the discussion, and
\cref{chapter:Related_Work} surveys related work, while
\cref{chapter:Future_Work} highlights future research venues. 
Finally, \cref{chapter:Conclusion} concludes this paper.



\section{Background}
\label{C++ Bad Forward Indirect Calls}
% In this section,
% we present a brief overview of the concept of C++-based polymorphism in~\cref{Polymorphism in C++}
% and how indirect calls can be checked in practice in~\cref{C++ Indirect Calls in Practice}.
% In~\cref{section:countpolicy} we present a forward edge function parameter count-based policy (\cite{veen:typearmor}), and
% in~\cref{Security Implications of Forbidden Forward Indirect Calls} we highlight security implications of indirect calls ,while
% in~\cref{{Too Permissive Parameter-Based Policies}} we show that the state-of-the-art parameter count-based policy
% (\cref{section:countpolicy}) is imprecise w.r.t. to the enforced calltarget set per callsite. 
% Finally, in~\cref{Running Example: CVE X} we present in detail a real COOP attack.

\subsection{Polymorphism in C++ Programs}
% \textbf{Polymorphism in C++.}
\label{Polymorphism in C++}
Polymorphism, along inheritance and encapsulation, are the most used modern object-oriented concepts in C++. In C++, polymorphism allows accessing different types of objects 
through a common base class. A pointer of the type of the base object can be used to point to object(s) which are derived from the base class. In C++, there are 
several types of polymorphism:
\textit{a)} compile-time (or static, usually is implemented with templates), 
\textit{b)} runtime (dynamic, is implemented with inheritance and virtual functions), 
\textit{c)} ad-hoc (\textit{e.g.,} if the range of actual types that can be used is finite and the combinations must be individually specified prior to use), and
\textit{d)} parametric (\textit{e.g.,} if code is written without mention of any specific type and thus can be used transparently with any number of new types). 
The first two are implemented through early and late binding, respectively. In C++, overloading concepts fall under the category of \textit{c)} and virtual functions, 
templates or parametric classes fall under the category of pure polymorphism. However, C++ provides polymorphism through: 
\textit{i)} virtual functions,
\textit{ii)} function name overloading, and 
\textit{iii)} operator overloading. 
In this paper, we are concerned with dynamic polymorphism, based on virtual functions (see ISO/IEC N3690~\cite{iso:iecN3690}), because it can be exploited to call: 
\textit{x)} illegitimate virtual table entries (not) contained in the class hierarchy by varying or not the number of parameters and types,
\textit{y)} legitimate virtual table entries (not) contained in the class hierarchy by varying or not the number of parameters and types, and 
\textit{z)} fake virtual tables entries not contained in the class hierarchy by varying or not the number of parameters and types.
By legitimate and illegitimate virtual table entries we mean those virtual table entries which for a single indirect callsite lie in the virtual table hierarchy. More 
precisely, a virtual table entry is legitimate for a callsite if from the callsite to the virtual table containing the entry there is an inheritance path (see~\cite{haller:shrinkwrap}). 
Virtual functions have several uses and issues associated, but for the scope of this paper we will look at the indirect callsites which are exploited by calling illegitimate virtual 
table entries (\textit{i.e.,} functions) with varying number and type of parameters, \textit{x)}. More precisely, 
\textit{1) load-time enforcement:} as calling each indirect callsite (\textit{i.e.,} callee) requires a fixed number of parameters which are passed each time the caller is calling, 
we enforce a fine-grained CFI policy by statically determining the number and types of all function parameter that belong to an indirect callsite, and
\textit{2) runtime verification:} as differentiating during runtime legitimate from illegitimate indirect caller/callee pairs requires parameter type (along parameter number), we 
check during run-time before each indirect callsite if the caller matches with the callee based on the previously added checks.


\newsavebox{\firstlisting}
\begin{lrbox}{\firstlisting}
\begin{minipage}[c]{\linewidth}
\begin{minted}[
% frame=lines,
framesep=2mm,
linenos,
%highlightlines={10},
highlightcolor={lightgray},
frame=none,
firstnumber=1,
framesep = 1.0cm,
linenos,
numbersep=2pt,
%gobble=2,
%frame=lines,
framesep=2mm,
fontsize=\tiny,      
% fontsize=\scriptsize,
% baselinestretch=1.2,
% bgcolor=LightGray,
% fontsize=\footnotesize,
]{C++}
class nsMultiplexInputStream final 
 :public nsIMultiplexInputStream //A0
 ,public nsISeekableStream //A1
 ,public nsIIPCSerializableInputStream //A2
 ,public nsICloneableInputStream{ //A3
nsTArray<nsCOMPtr<nsIInputStream>> mStreams;
NS_IMETHODIMP nsMultiplexInputStream::Close(){
  MutexAutoLock lock(mLock);
  mStatus = NS_BASE_STREAM_CLOSED;
  //set NS_OK flag
  nsresult rv = NS_OK;
  //get array length
  uint32_t len = mStreams.Length();
 //array-based main loop gadget (ML-G)
 for (uint32_t i = 0; i<len; ++i){
  //(0)hijacked object dispatch
  nsresult rv2=mStreams[i]->Close();
  if (NS_FAILED(rv2)) {
      rv = rv2;
  }
 }
  return rv;
}
\end{minted}
\end{minipage}
\end{lrbox}

% \subsubsection{Exploiting Object Dispatches in C++}
% % \textbf{Exploiting Object Dispatches.}
% \label{Exploiting Polymorphism Weaknesses}
%  \begin{figure}[!h]
%    \vspace{-.37cm}
%    \centering
%    \resizebox{2.3\linewidth}{!}{
%    \setlength{\unitlength}{0.1\textwidth}
%    \begin{picture}(10,4)
% %    \centering
%      \put(2.1, 1.5){\includegraphics[width=.22\textwidth]{figures/loop.pdf}}
%      \put(.031, 2.5){\usebox{\firstlisting}}
%    \end{picture}}
% \vspace{-2.6cm}
% % \caption{Description of how a counterfeit object-oriented programming main loop gadget (ML-G) works.}
% \caption{COOP loop gadget (ML-G, REC-G, UNR-G) at work.}
% \label{Code example used to illustrate how a COOP loop gadget works}
% \end{figure}
% 
% Figure~\ref{Code example used to illustrate how a COOP loop gadget works}
% depicts a C++ code example (left) and how a COOP main-loop gadget (right) 
% (\textit{i.e.,} based either on ML-G (main-loop) or REC-G (recursive-gadget) or UNR-G (unrolled COOP gadget), 
% see~\cite{crane:readactor++} for more details) is used to sequentially call COOP gadgets by iterating trough 
% a loop controlled by the attacker.
% 
% First, the object dispatch (see Figure~\ref{Code example used to illustrate how a COOP loop gadget works} line 17) is exploited by the attacker
% in order to call different functions in the whole program by iterating on an array of fake objects previously inserted in the array.
% 
% Second, in order to achieve this the attacker previously exploits an existing program memory corruption (\textit{e.g.,} buffer overflow) 
% which is further used to corrupt an object dispatch, \ding{182}, by inserting fake objects in the array and by changing the number of initial loop iterations.
% Next she invokes gadgets, \ding{182} and \ding{184} up to {\tiny\encircle{\Large{M}}}, 
% through the calls, \ding{183} and \ding{185} up to {\tiny\encircle{\Large{N}}}, contained in the loop. 
% As it can be observed in Figure~\ref{Code example used to illustrate how a COOP loop gadget works} she 
% can invoke from the same callsite legitimate functions (in total {\tiny\encircle{\Large{N}}}) residing in the virtual table (vTable) inheritance path
% (\textit{i.e.,} at the time of writing this paper this type of information is particularly hard to recuperate from program binaries)
% for this particular callsite, indicated with green color vTable entries. 
% However, a real COOP attack invokes illegitimate vTable entries residing in the whole initial program hierarchy (or the extended one)
% with less or no relationship to the initial callsite,
% indicated with red-color vTable entries.
% 
% Third, in this way different addresses contained in the program (1) (vTable) hierarchy (contains only virtual members), 
% (2) class hierarchy (contains both virtual and non-virtual members) and (or) the whole program address space can be called. 
% For example the attacker can call in any entry in the:
% (1) class hierarchy of the whole program,
% (2) class hierarchy containing only legitimate targets for this callsite,
% (3) virtual table hierarchy of the whole program,
% (4) virtual table hierarchy containing only legitimate targets for this callsite,
% (5) virtual table hierarchy and class hierarchy containing only legitimate targets for this callsite, and
% (6) virtual table hierarchy and class hierarchy of the whole program.
% 
% Finally, because there are no intrinsic language semantics---such as object cast checks---in the C++ programming language for object dispatches
% the loop gadget indicated in Figure~\ref{Code example used to illustrate how a COOP loop gadget works} can be unconstrained used to call 
% any possible entry in the whole program. Thus, making any program address a gadget part.
% 
% \subsubsection{Security Implications of Indirect Calls}
% % \textbf{Security Implications of Forbidden Indirect Calls.}
% \label{Security Implications of Forbidden Forward Indirect Calls}
% The C++ language standard (N3690~\cite{iso:iecN3690}) does not specify what happens when calling different virtual table entries from an indirect callsite. 
% The standard says that we have a virtual function-related undefined behavior when: \textit{a virtual function call uses an explicit class member access and 
% the object expression refers to the complete object of x or one of that object's base class sub-objects but not x or one of its base class sub-objects}. As 
% undefined behavior is not a clearly defined concept, we argue that in order to be able to deal with undefined behavior or unspecified behavior related to 
% virtual function calls one needs to know how these language-dependent concepts are implemented inside the used compilers.
% 
% Forbidden forward-edge indirect calls are the result of a vPointer corruption. A vPointer corruption is not a vulnerability, but rather a capability which
% can be the result of a spatial or temporal memory corruption triggered by: 
% (1) bad-casting~\cite{byoungyoung:typecasting} of C++ objects, 
% (2) buffer overflow in a buffer adjacent to a C++ object or a use-after-free condition~\cite{schuster:coop}.
% A vPointer corruption can be exploited in several ways. A manipulated vPointer can be exploited by pointing it in any existing or added program virtual 
% table entry or into a fake virtual table which was added by an attacker. For example in case a vPointer
% was corrupted than the attacker could highjack the control flow of the program and start a COOP attack~\cite{schuster:coop}.
% 
% vPointer corruptions are a real security threat which can be exploited if there is a memory corruption (\textit{e.g.,} buffer overflow) which is adjacent 
% to the C++ object or a use-after-free condition. As a consequence, each corruption which can reach an object (\textit{e.g.,} bad object casts) is a potential
% exploit vector for a vPointer corruption. Interestingly to notice in this context is that through:
% (1) memory layout analysis (through highly configurable compiler tool chains) of source code based locations which are highly prone to memory corruptions such 
% as declarations and uses of buffers, integers or pointer deallocations one can obtain the internal machine code layout representation.
% (2) analysis of a code corruption which is adjacent (based on (1)) to a C++ object based on application class hierarchy, the virtual table hierarchy and each
% location in source code where an object is declared and used (\textit{e.g.,} modern compiler tool chains can spill out this information for free), one can 
% derive an analysis which can determine---up to a certain extent---if a memory corruption can influence (\textit{e.g.,} is adjacent) to a C++ object.
% 
% Finally, tools based on these two concepts (\textit{i.e.,} (1) and (2)) can be used by attackers, \textit{e.g.,} to find new vulnerabilities, and by defenders
% to harden the source code only at the places which are most exposed to such vulnerabilities (\textit{i.e.,} targeted security hardening).
% 
% \subsection{Real COOP Attack Example}
% \label{Real COOP Attack Example}
% % \textbf{Real COOP Attack Example.}
% The bug CVE-2014-3176 was exploited by Crane \textit{et al.}~\cite{crane:readactor++} in order to perform a
% COOP attack, on the Google Chromium Web browser. The details of this attack are highly complex involving not properly 
% handled interaction of browser extensions between the IPC, the sync API, and Google V8 engine and for this reason we briefly present a better
% documented COOP exploit which is in principle similar with this attack.
% 
% \label{Running Example: CVE X}
% %%second pic
% \begin{figure}[h!]
%     \centering
%     \includegraphics[width=0.35\textwidth]{figures/class_hierarchy.pdf}
% \caption{Class hierarchy of classes used in the COOP attack.}
% \label{Class exploit}
% \vspace{-.29cm}
% \end{figure}
% Figure~\ref{Class exploit} depicts~\footnote{The class inheritance hierarchy of the classes involved in the COOP attack against the Firefox browser. Red letters 
% indicate forbidden virtual table entries and green letters indicate allowed virtual table entries for the given indirect callsite
% contained in the main loop gadget.} a turing complete COOP attack~\cite{schuster:coop} which was used to attack the Mozilla Firefox Web browser. 
% By exploiting an existing buffer overflow bug the attacker was able to call into existing virtual table entries by having a main loop gadget at his disposal.
% 
% First, the attacker uses the C++ class \texttt{nsMultiplexInputStream} (see Figure~\ref{Class exploit}) which contains a 
% main loop gadget (ML-G) inside the \texttt{nsMultiplexInputStream::Close(void)} 
% function in order to perform indirect calls by dispatching calls on the fake objects contained in the array. The objects 
% contained in the array during normal execution are of \texttt{nsInputStream} type and each of the objects will call the 
% \texttt{Close(void)} function in order to close each of the previously opened streams. 
% 
% Second, for performing the COOP attack, the 
% attacker crafts a C++ program containing an array buffer holding six fake objects. These fake objects can call inside (and outside) 
% the initial class and virtual table hierarchies with no constraints. During the attack a buffer is created in order to hold the 
% fake objects. The crafted buffer will be used instead of the real code in order to call different functions available in the program code. 
% For example, the attacker calls a function contained in the class \texttt{xpcAccessibleGeneric} which is not in the class 
% hierarchy or virtual table hierarchy of the initially intended type of objects used inside the array. Moreover, the header 
% file of this class (\texttt{xpcAccessibleGeneric}) is not included in the class \texttt{nsMultiplex-InputStream}. 
% 
% Third, in total six fake objects are used to call into functions residing in unrelated class hierarchies with varying number of parameters 
% and return types. The final goal of this attack is to prepare the program memory such that a Unix shell can be opened at 
% the end of this attack.
% 
% 
% Finally, this example illustrates why detecting vPointer corruptions is not trivial for real-world applications. As depicted in 
% Figure~\ref{Class exploit}, the class \texttt{nsInputStream} has 11 classes which inherit directly or indirectly from 
% this class. The classes \texttt{nsSeekableStream}, \texttt{nsIPCSerializableInputStream} and \texttt{nsCloneableInputStream}
% provide additional inherited virtual tables which represent illegitimate calltargets for the initial \texttt{nsInputStream} 
% objects and legitimate calltargets for the six fake objects which were added during the attack. Furthermore, declaration and
% usage of the objects can be widely spread out in the source code. This makes detection of the object types 
% (\textit{i.e.,} base class), range of virtual tables (\textit{i.e.,} longest virtual table inheritance path for a
% particular callsite) and parameter types of the virtual table entries (\textit{i.e.,} functions) in which it is 
% allowed to call a trivial task for source code applications, but a hard task when one wants to apply similar 
% security policies (\textit{e.g.,} which rely on parameter types of virtual table entries) to binary executables.

\subsection{Checking Indirect Calls in Practice}
% \textbf{Checking Indirect Forward-Edge Calls in Practice.}
\label{C++ Indirect Calls in Practice}
To the best of our knowledge, only the IFCC/VTV~\cite{vtv:tice} compiler based tools (up to 8.7\% performance overhead) are deployed in practice
and can be used to check legitimate from illegitimate indirect forward-edge calls during runtime. Virtual pointers are checked based on the class hierarchy. 
Furthermore, ShrinkWrap~\cite{haller:shrinkwrap} (to the best of our knowledge not deployed in practice) is a tool which further reduces the legitimate 
virtual table ranges for a given indirect callsite through precise analysis of the program class hierarchy and virtual table hierarchy. Evaluation results
show similar performance overhead but more precision with respect to legitimate virtual table entries per callsite. We noticed by analyzing the previous 
research results that the overhead incurred by these security checks can be very high due to the fact that for each callsite many range checks have to be
performed during runtime. Therefore, in our opinion, despite its security benefit these types of checks cannot be applied to high performance applications.

A number of other highly promising tools (albeit also not deployed in practice) can overcome some of the drawbacks of the previously described tools. 
Bounov \textit{et al.}~\cite{bounov:interleaving} presented a tool ($\approx$ 1\% runtime overhead)
for indirect forward-edge callsite checking based on virtual table layout interleaving. The tool has better performance than VTV and better precision with
respect to allowed virtual tables per indirect callsite. Its precision (selecting legitimate virtual tables for each callsite) compared to ShrinkWrap is
lower since it does not consider virtual table inheritance paths. vTrust~\cite{zhang:vtrust} (average runtime overhead 2.2\%) enforces two layers of defense
(virtual function type enforcement and virtual table pointer sanitization) against virtual table corruption, injection and reuse. TypeArmor~\cite{veen:typearmor}
($\le$ than 3 \% runtime overhead) enforces a CFI-policy based on runtime checking of caller/callee pairs and function parameter count matching. It is important to note 
that there are no C++ language semantics which can be used to enforce type and parameter count matching for indirect caller/callee pairs, this could be addressed
with specifically intended language constructs in the future.





\section{Threat Model}
\label{Adversary Model}

We align our threat model with the same basic assumptions as described in~\cite{veen:typearmor}. 
More precisely, we assume a resourceful attacker that has read and write access to the data 
sections of the attacked program binary. We also assume that the protected binary does not contain 
self-modifying code, handcrafted assembly or any kind of obfuscation. We also consider pages 
to be either writable or executable but not both at the same time. Further, we assume 
that the attacker has the ability to execute a memory corruption to hijack the program
control flow. Finally, the analyzed program binary is not hand-crafted and the compiler
which was used to generate the binary adheres to one of the 
standard calling conventions mentioned in \cref{chapter:Introduction}.
\chapter{TypeShild Overview}
\label{chapter:TypeShild Overview}

after the Design and Implementation section is done
we pick the most important points of TypeShild design and Implementation and describe them here.
The goal of this section is to be an appetizer for the whole design and Implementation section.
Which are usually dry (trocken).m

\section{1. Select Important Point from Design Chapter}
\section{2. Select Important Point from Design Chapter}
\section{3. Select Important Point from Design Chapter}
\section{System Design}
\label{chapter:Design}

In this section, we present 
% in~\cref{Parameter Count Based Policy} our function parameter count based policy,
in \cref{section:typepolicy}, the details of our type policy, and 
in~\cref{section:instructionanalysis} we introduce the definitions for our instructions analysis based on register states, 
while in~\cref{section:calltargetanalysis} we present the design of our calltarget analysis.
In~\cref{section:callsiteanalysis} we depict the design of our callsite analysis\footnote{Callsites detection in the 
binary is based on the capabilities of DynInst.}, and  
in~\cref{Enforcing The Forward Edge Policy} we present our forward-edge policy instrumentation strategy, while 
in~\cref{Backward Edge Analysis} we highlight our function backward-edge analysis and policy instrumentation strategy.

\subsection{Parameter Register Wideness Based Policy}
\label{section:typepolicy}

% \begin{figure}[h!]
% \center
% \resizebox{.65\columnwidth}{!}{
% \begin{tikzpicture}
% 
% \fill[black!10!white] (0,7.5) rectangle (12,6);
% \fill[black!20!white] (0,6) rectangle (12,4.5);
% \fill[black!30!white] (0,4.5) rectangle (12,3);
% \fill[black!40!white] (0,3) rectangle (12,1.5);
% \fill[black!50!white] (0,1.5) rectangle (12,0);
% 
% \draw[-triangle 45, thick] (-0.5,0) -- node[sloped, anchor=center, above] {\LARGE{growing required bits/parameter}} (-0.5,7.5);
% \draw[-triangle 45, thick] (12.5,7.5) -- node[sloped, anchor=center, above] {\LARGE{growing provided bits/parameter}} (12.5,0);
% 
% \draw (0,7.5)  --node[anchor=south] {\rot{\LARGE{p. 6}}} (2,7.5);
% \draw (2,7.5)  --node[anchor=south] {\rot{\LARGE{p. 5}}} (4,7.5);
% \draw (4,7.5)  --node[anchor=south] {\rot{\LARGE{p. 4}}} (6,7.5);
% \draw (6,7.5)  --node[anchor=south] {\rot{\LARGE{p. 3}}} (8,7.5);
% \draw (8,7.5)  --node[anchor=south] {\rot{\LARGE{p. 2}}} (10,7.5);
% \draw (10,7.5) --node[anchor=south] {\rot{\LARGE{p. 1}}} (12,7.5);
% 
% \draw (0,7.5)  rectangle node[anchor=center] {\LARGE{0-bits}} (2,6);
% \draw (2,7.5)  rectangle node[anchor=center] {\LARGE{0-bits}} (4,6);
% \draw (4,7.5)  rectangle node[anchor=center] {\LARGE{0-bits}} (6,6);
% \draw (6,7.5)  rectangle node[anchor=center] {\LARGE{0-bits}} (8,6);
% \draw (8,7.5)  rectangle node[anchor=center] {\LARGE{0-bits}} (10,6);
% \draw (10,7.5) rectangle node[anchor=center] {\LARGE{0-bits}} (12,6);
% 
% \draw (0,6)  rectangle node[anchor=center] {\LARGE{8-bits}} (2,4.5);
% \draw (2,6)  rectangle node[anchor=center] {\LARGE{8-bits}} (4,4.5);
% \draw (4,6)  rectangle node[anchor=center] {\LARGE{8-bits}} (6,4.5);
% \draw (6,6)  rectangle node[anchor=center] {\LARGE{8-bits}} (8,4.5);
% \draw (8,6)  rectangle node[anchor=center] {\LARGE{8-bits}} (10,4.5);
% \draw (10,6) rectangle node[anchor=center] {\LARGE{8-bits}} (12,4.5);
% 
% \draw (0,4.5)  rectangle node[anchor=center] {\LARGE{16-bits}} (2,3);
% \draw (2,4.5)  rectangle node[anchor=center] {\LARGE{16-bits}} (4,3);
% \draw (4,4.5)  rectangle node[anchor=center] {\LARGE{16-bits}} (6,3);
% \draw (6,4.5)  rectangle node[anchor=center] {\LARGE{16-bits}} (8,3);
% \draw (8,4.5)  rectangle node[anchor=center] {\LARGE{16-bits}} (10,3);
% \draw (10,4.5) rectangle node[anchor=center] {\LARGE{16-bits}} (12,3);
% 
% \draw (0,3)  rectangle node[anchor=center] {\LARGE{32-bits}} (2,1.5);
% \draw (2,3)  rectangle node[anchor=center] {\LARGE{32-bits}} (4,1.5);
% \draw (4,3)  rectangle node[anchor=center] {\LARGE{32-bits}} (6,1.5);
% \draw (6,3)  rectangle node[anchor=center] {\LARGE{32-bits}} (8,1.5);
% \draw (8,3)  rectangle node[anchor=center] {\LARGE{32-bits}} (10,1.5);
% \draw (10,3) rectangle node[anchor=center] {\LARGE{32-bits}} (12,1.5);
% 
% \draw (0,1.5)  rectangle node[anchor=center] {\LARGE{64-bits}} (2,0);
% \draw (2,1.5)  rectangle node[anchor=center] {\LARGE{64-bits}} (4,0);
% \draw (4,1.5)  rectangle node[anchor=center] {\LARGE{64-bits}} (6,0);
% \draw (6,1.5)  rectangle node[anchor=center] {\LARGE{64-bits}} (8,0);
% \draw (8,1.5)  rectangle node[anchor=center] {\LARGE{64-bits}} (10,0);
% \draw (10,1.5) rectangle node[anchor=center] {\LARGE{64-bits}} (12,0);
% \end{tikzpicture}
% }
% \caption{Type policy schema for callsites and calltargets.}
% \label{fig:TYPEschema}
% \vspace{-.5cm}
% \end{figure}
% 
% Figure~\ref{fig:TYPEschema} depicts the basic principle of our function parameter type policy.
% Note that p. means parameter. As it is depicted in this example, when requiring parameter width, one starts at the 
% bottom of the above matrix and grows to the top, as it is always possible to accept more parameters than required.
% Also, the reverse is true for providing parameters, as it is possible to accept less parameters than provided.
% Note that accepting more parameters than provided is not allowed.

We use the register width of the function parameter in order to infer the type information. As previously mentioned, there are 4 types 
of reading and writing accesses. Therefore, our set of possible types for parameters is $\texttt{TYPE} = \{64, 32, 16, 8, 0\}$; where 0 models the absence of a 
parameter. Since Itanium C++ ABI specifies 6 registers (\textit{i.e.,} \texttt{rdi}, \texttt{rsi}, \texttt{rdx}, \texttt{rcx}, \texttt{r8}, and \texttt{r9}) as 
parameter passing registers during function calls, we classify our callsites and calltargets into $\texttt{TYPE}^6$. Similar to
our count policy, we allow overestimations of callsites and underestimations of calltargets, on the parameter types as well. Therefore, for a 
callsite $cs$ it is possible to call a calltarget $ct$, only if for each parameter of $ct$ the corresponding parameter of $cs$ is not smaller w.r.t. the register width.
This results in a finer-grained policy which is further restricting the possible set of calltargets for each callsite.

%What we call the \emph{type} policy is the idea of not only relying on the parameter count but also on the parameter type. However, due to complexity reasons,
%we are restricting ourselves to the general purpose registers, which the Intanium C++ ABI designates as parameter registers. Furthermore, we are not inferring 
%the actual type of the data but the wideness of the data stored in the register. The schema again is that we have calltargets requiring wideness and the
%callsite providing it as depicted in Figure \ref{fig:TYPEschema}.

%e are currently interested in x86-64 binaries, the registers we are looking at are 64-bit registers that can be accessed in four different ways:
%\textit{1)} the whole 64-bit of the register, meaning a wideness of 64,
%\textit{2)} the lower 32-bit of the register, meaning a wideness of 32,
%\textit{3)} the lower 16-bit of the register, meaning a wideness of 16, and
%\textit{4)} the lower 8-bit of the register, meaning a wideness of 8.

%Four of those registers can also directly access the higher 8-bit of the lower 16-bit of the register. For our purpose we register this access as a 16-bit access. 

%Based on this information,we can assign a register one of 5 possible types $\mathcal{T} = \{64, 32, 16, 8, 0\}$. We also included the type 0 to model the absence of data within a register. 
%Similar to the \emph{count} policy, we allow overestimation of types in callsites and underestimation of types in calltargets. However, the matching idea is different, 
%because as can we depict in Figure \ref{fig:TYPEschema}, the type of a calltarget and a callsite no longer depends solely on its parameter count, 
%each callsite and calltarget has its type from the set of $\mathcal{T}^6$, with the following comparison operator:
%$
%	u \leq_{type} v :\Longleftrightarrow  
%	\forall_{i = 0}^{5} {u_i \leq v_i} , \text {with } u, v \in \mathcal{T}^6
%$.

%Again we allow any callsite $cs$ call any calltarget $ct$, when it fulfills the requirement $ct \leq cs$. 
%The way we represent this is by letting the type for a calltarget parameter progress from 64-bit to 0-bit---if a calltarget requires a 32-bit value in its 1st 
%parameter, it also should accept a 64-bit value from its callsite---and similarly we let the type for a callsite progress from 0-bit to 64-bit - If a 
%callsite provides a 32-bit value in its 1st parameter it also provides a 16-bit, 8-bit and 0-bit to a calltarget. Now the advantage of the \emph{type} policy
%in comparison to the \emph{count} policy is that while our type comparison implies the count comparison, the other direction does not hold.
%Meaning, just having an equal or lesser number of parameters than a callsite, does no longer allow a calltarget being called there, thus restricting the number of calltargets per 
%callsite even further. A function that requires 64-bit in its first parameter, and 0-bit in all other parameters, would have been callable by a callsite providing 8-bit 
%in its first and second parameter when using the \emph{count} policy, however in the \emph{type} policy this is no longer possible. Thus, it should decrease the number of targets per bucket.

% \subsection{Parameter Count Based Policy}
% \label{Parameter Count Based Policy}
% \textbf{\emph{Count} Policy.}
% \label{section:countpolicy}
% \vspace{-1.1em}
% \begin{figure}[!h]
% \centering
% \resizebox{0.2\textwidth}{!}{
% \begin{tikzpicture}
% 
% %\fill[black!40!white] (0,0) rectangle (9,9);
% %\fill[black!30!white] (0,0) rectangle (7,7);
% %\fill[black!20!white] (0,0) rectangle (4,4);
% %\fill[black!10!white] (0,0) rectangle (2,2);
% %
% %\draw (0,0) --node[anchor=south] {0 params}  (2,0)  -- (2,2) -- (0,2) -- (0,0) ;
% %\draw (0,0) -- (2,0) --node[anchor=south] {1 param} (4,0) -- (4,4) -- (0,4) -- (0,0);
% %\draw (0,0) --(4,0) --node[anchor=south] {2 ... 5 params} (7,0) -- (7,7) -- (0,7) -- (0,0);
% %\draw (0,0) --(7,0) --node[anchor=south] {6 params} (9,0) -- (9,9) -- (0,9) -- (0,0);
% %\draw[dashed] (4,4) -- (7,7);
% 
% 
% \fill[black!00!white] (0,7) rectangle (6,6);
% \fill[black!10!white] (0,6) rectangle (6,5);
% \fill[black!20!white] (0,5) rectangle (6,4);
% \fill[black!30!white] (0,4) rectangle (6,3);
% \fill[black!40!white] (0,3) rectangle (6,2);
% \fill[black!50!white] (0,2) rectangle (6,1);
% \fill[black!60!white] (0,1) rectangle (6,0);
% 
% 
% \draw[-triangle 45, thick] (-0.5,0) -- node[sloped, anchor=center, above] {\Large{growing required \# of bytes per param.}} (-0.5,7);
% \draw[-triangle 45, thick] (6.5,7) -- node[sloped, anchor=center, above]  {\Large{growing provided \# of bytes per param.}} (6.5,0);
% 
% \draw (0,7) rectangle node[anchor=center] {\LARGE{0 parameters}} (6,6);
% 
% \draw (0,6) rectangle node[anchor=center] {\LARGE{1 parameter}}  (6,5);
% 
% \draw (0,5) rectangle node[anchor=center] {\LARGE{2 parameters}} (6,4);
% 
% \draw (0,4) rectangle node[anchor=center] {\LARGE{3 parameters}} (6,3);
% 
% \draw (0,3) rectangle node[anchor=center] {\LARGE{4 parameters}} (6,2);
% 
% \draw (0,2) rectangle node[anchor=center] {\LARGE{5 parameters}} (6,1);
% 
% \draw (0,1) rectangle node[anchor=center] {\LARGE{6 parameters}} (6,0);
% 
% 
% \end{tikzpicture}
% }
% \caption{Callsite \& calltargets {count} policy classification.}
% \label{fig:COUNTschema}
% \vspace{-.5cm}
% \end{figure}
% 
% Figure~\ref{fig:COUNTschema} depicts the used matching schema which shows that calltargets require
% parameters whereas callsites provide these parameters. 

Further, we built a function parameter count-based policy similar to~\cite{veen:typearmor}. Calltargets are classified based on the number of parameters 
that these provide and callsites are classified by the number of parameters that these require. 
Further, we consider the generation of high precision measurements for such classification with binaries as the only source of information rather difficult. 
Therefore, over-estimations of parameter count for callsites and underestimations of the parameter count for calltargets is deemed acceptable. 
This classification is based on the general purpose registers that the call convention of the current ABI---in this case the 
Itanium C++ ABI~\cite{itanium:abi}---designates as parameter registers. Furthermore, we do not consider floating point registers or multi-integer registers for simplicity
reasons.
The \emph{count} policy is based on allowing any callsite $cs$, which provides $c_{cs}$ parameters, to call any calltarget $ct$, 
which requires $c_{ct}$ parameters, iff $c_{ct} \leq c_{cs}$ holds. However, the main problem is that while there is a significant 
restriction of calltargets for the lower callsites, the restriction capability drops rather rapidly when reaching higher parameter 
counts, with callsites that use 6 or more parameters being able to call all possible calltargets.
This is more precisely expressed as 
$\forall$ $cs_1$, $cs_2$; $c_{cs_1}$ $\leq$ $c_{cs_2}$ $\rightarrow$  $\|$ $\{ct \in \mathcal{F}$ $|$ $c_{ct}$ $\leq$ $c_{cs_1}$ $\} \| \leq$ $\|$ $\{ct \in \mathcal{F} | c_{ct} \leq c_{cs_2}  \} \|$.

One possible remedy would be the ability to introduce an upper bound for the classification deviation of parameter counts, 
however, as of now, this does not seem feasible with current technology. Another possibility would be the overall reduction
of callsites, which can access the same set of calltargets, a route which we will explore within this work.

\subsection{Analysis of Register States}
\label{section:instructionanalysis}

Our register state analysis is register state based, another alternative would be to do symbol-based data-flow analysis which we will leave as future work.
In order for the reader to understand our analysis we will first give some definitions.
The set $\texttt{INSTR}$ describes all possible instructions that can occur within the executable section of a program binary. In our case,
this is based on the x86-64 instruction set. An instruction $i \in \texttt{INSTR}$ can non-exclusively perform two kinds of operations on any number of existing 
registers. Note that there are registers that can directly access the higher 
8-bit of the lower 16-bit. For our purpose, we register this access as a 16-bit access.
(1) Read $n$-bit from the register with $n \in \{ 64, 32, 16, 8 \}$, and 
(2) Write $n$-bit to the register with $n \in \{ 64, 32, 16, 8 \}$.

Next, we describe the possible change within one register as $\delta \in \Delta$ with $\Delta = \{ w64, w32, w16, w8, 0 \} \times \{r64, r32, r16, r8, 0 \}$. 
Note that 0 represents 
the absence of either a write or read access and $(0, 0)$ represents the absence of both. Furthermore, $wn$ or $rn$ with $n \in \{64,32,16,8\}$ implies all $wm$ or $rm$ with $m \in 
\{64,32,16,8\}$ and $m < n$ (\textit{e.g.,} $r64$ implies $r32$). Note that we exclude 0, as it means the absence of any access.
Intanium C++ ABI specifies 16 general purpose integer registers. Therefore, we represent the change occurring at the processor level as $\delta_p \in \Delta^{16}$. 
In our analysis, we calculate this change for each instruction $i \in \texttt{INSTR}$ via the function $decode : \texttt{INSTR} \mapsto \Delta^{16}$.

% Finally, in \cref{section:addresstakenanalysis} we introduce a version of 
% address taken analysis based on \cite{mingwei:sekar} to restrict the number of available calltargets even more. 
%At last we introduce a patching schema for callsites and calltargets to enforce the invariants we inferred.

%Usually data-flow analysis algorithms are based on set of variable or sets of definitions, which both are basically unbounded. However, we are analyzing the state of registers, 
%which are baked into hardware and therefore their number is given, thus requiring us to adapt the data-flow theory to work on tuples.
%
%The set $\mathcal{I}$ describes all possible instructions that can occur within the executable section of a binary. In our case this is based on the instruction set for x86-64 processors.
%
%An instruction $i \in \mathcal{I}$ can non-exclusively perform two kinds of operations on any number of existing registers:
%\textit{1)} Read $n$-bit from the register with $n \in \{ 64, 32, 16, 8 \}$, and
%\textit{2)} Write $n$-bit to the register with $n \in \{ 64, 32, 16, 8 \}$.
%
%Thus, we describe the possible change that occurs in one register with the set $S = \{ w64, w32, w16, w8, 0 \} \times \{r64, r32, r16, r8, 0 \}$. Note that 0 signals the absence
%of either a write or read access and $(0, 0)$ signals the absence of both. Furthermore, $wn$ or $rn$ with $n \in \{64,32,16,8\}$ implies all $wm$ or $rm$ with $m \in \{64,32,16,8\}$ 
%and $m < n$ (\textit{e.g.,} $r64$ implies $r32$). Note that we exclude 0, as it means the absence of any access.

%Intanium C++ ABI specifies 16 general purpose integer registers, thus for our purpose we represent the change occurring at the processor level as $\mathcal{S} = S^{16}$.

%At last we declare a function, which calculates the change occurring in the processor state, when executing an instruction from $\mathcal{I}$:
%$decode : \mathcal{I} \mapsto \mathcal{S}$.

%However, we do not go into detail how this function actually calculates this sate, because we rely on external libraries to perform this task. Implementing this function our self 
%is out of scope due to the lengthy work required, as the x86-64 instruction set is quite large.

\subsection{Calltarget Analysis}
\label{section:calltargetanalysis}
Our calltarget analysis classifies calltargets according to the parameters they expect. Underestimations are allowed, however, overestimations 
are not permitted. For this purpose, we employ a customizable modified liveness analysis 
algorithm, which iterates over address-taken\footnote{A program function is defined to have its address taken if there is at least one binary instruction
which loads the function entry point into memory. Note that by definition, indirect calls can only target AT functions.} functions 
with the goal of analyzing register state information in order to determine if these registers are used for arguments passing.
% Furthermore, we will present certain corner cases we encountered, at the end of this section.

% \subsubsection{Liveness Analysis}
% A variable is alive before the execution of an instruction, if at least one of the originating paths performs a read access before any write access on that variable. 
% If applied to a function, this calculates the variables that need to be alive at the beginning, as these are its parameters. 
% 
% \begin{algorithm}[h!]
% %         \scriptsize
%         \footnotesize
%  	\SetAlgoLined
% 	\SetKwInOut{Input}{Input}
%         \SetKwInOut{Output}{Output}
%         \Input{The basic block to be analyzed - block : $\texttt{INSTR}^*$}
%         \Output{The liveness state - $\mathcal{S}^\mathcal{L}$}
%         \BlankLine
% 	\SetKwProg{Fn}{Function }{ is}{end}
% 	\Fn{\texttt{analyze} (block : $\texttt{INSTR}^*$) : $\mathcal{S}^\mathcal{L}$}
% 	{
%  	$state$ = Bl \Comment*[r]{Initialize the state with first block}
%  	
%  	\ForEach{inst $\in$ block}{
%  	
%  		$state' = \texttt{analyze\_instr}(inst)$                     \Comment*[l]{Calc. changes}
% 
% 		$state = \texttt{merge\_h}(state, state')$                     \Comment*[r]{Merge changes}
% 	}
% 
% 	$states$ = $\emptyset$                                                 \Comment*[r]{Set of succ. states}
% 	
% 	$blocks$ = $\texttt{successor(block)}$                                   \Comment*[r]{Get succ. blocks}
% 	
% 	\ForEach{block' $\in$ blocks} {
% 	
%  		$state'$ = $\texttt{ analyze}(block')$   \Comment*[r]{Analyze succ. block}
%  		
% 		$states$ = $states$ $\cup$ \{$state'$\}  \Comment*[r]{Add succ. states}
% 	}
% 
% 	$state'$ = $\texttt{merge\_h }(states)$   \Comment*[r]{Merge succ. states}
% 
% 	\Return $merge\_v(state, state')$  \Comment*[r]{Merge to final state}
% 
% 	}
% \caption{Basic block liveness analysis.}
% \label{alg:liveness}
% \end{algorithm}
% % \vspace{-.5cm}
% Algorithm~\ref{alg:liveness} is based on the liveness analysis algorithm presented in~\cite{khedker2009data}, which consists of a depth-first traversal of basic blocks. 
% For customization, we rely on the implementation of several functions which we will present next. $\mathcal{S}^\mathcal{L}$ is the set of possible register states which depends on the specific 
% implementations of the following operations.
% 
% $\bullet$ $\texttt{merge\_v} : \mathcal{S}^\mathcal{L} \times \mathcal{S}^\mathcal{L} \mapsto \mathcal{S}^\mathcal{L}$, (merge vertically block states) describes how to merge the current state with the following state change.
% 
% $\bullet$ $\texttt{merge\_h} : \mathcal{P}(\mathcal{S}^\mathcal{L}) \mapsto \mathcal{S}^\mathcal{L}$, (merge horizontally block states) describes how to merge a set of states resulting from several paths.
% 
% $\bullet$ $\texttt{analyze\_instr} : \texttt{INSTR} \mapsto \mathcal{S}^\mathcal{L}$, (analyze instruction) calculates the state change that occurs due to the given instruction.
% 
% $\bullet$ $\texttt{succ} : \texttt{INSTR}^* \mapsto \mathcal{P}(\texttt{INSTR}^*)$, (successor of a basic block) calculates the successors of the given block.
% 
% In our specific case, the function \texttt{analyze\_instr} needs to also to handle non-jump and non-fall-through successors, as these are not handled by DynInst. 
% Essentially, there are three relevant cases.
% First, if the current instruction is an indirect call or a direct call and the analysis algorithm is set not to follow calls, 
% then our analysis will return a state where all registers are considered to be written before read. Second, if the current instruction is
% a direct call and the analysis algorithm is set not to follow calls, 
% then we start an analysis of the target function an return its result.
% If the instruction is a constant write (\textit{e.g.,} xor of two registers), 
% then we remove the read portion before we return the decoded state.
% Finally, in any other case, we simply return the decoded state.
% This leaves us with the two undefined merge functions and the undefined liveness state $\mathcal{S}^\mathcal{L}$. 

\subsubsection{Required Parameter Wideness}
\label{subsection:requiredparamwideness}
For our type policy, we need a finer representation of the state of one register as follows.
(1) $W$ represents write before read access,
(2) $r8, r16, r32, r64$ represents read before write access with 8-, 16-, 32-, 64-bit width, and
(3) $C$ represents the absence of access.
%\begin{enumerate}
%\item Was the register written to before its value could be read ? \\ We represent this with the state $W$.
%\item How much was read from the register before its value was overwritten? \\ We represent this with the states $\{ r8, r16, r32, r64 \}$ 
%using $R$ as a placeholder for arbitrary reads.
%\item Did neither read nor write access occur for the register ? \\ We represent this with the state $C$.
%\end{enumerate}
This gives us the following $S^\mathcal{L} = \{ C, r8, r16, r32, r64, W \}$ register state which translates to the register super state 
$\mathcal{S}^\mathcal{L} = (S^\mathcal{L})^{16}$.
%Now, we assume that unless the instructions we are looking at does discard the value it is reading (\texttt{xor rax rax} would be such 
%an instruction that we call const\_write) that reading does precede the writing withing one instruction.
As there could be more than one read of a register before it is written, we might be interested in more than just the first occurrence of a write or read on a path. 
To permit this, we allow our merge operations to also return the value $RW$, which represents the existence of both read and write access and then can use $W$ with the functionality of an end marker.
% We arrive therefore at three possible vertical merge functions:
%\begin{itemize}
%	\item The same vertical merge operator as used in the \emph{count} policy, which only gives us the first non $C$ state ($merge\_v^{r}$).
%	\item A vertical merge operator that conceptually intersects all read accesses along a path until the first write occurs ($merge\_v^{i}$).
%	\item A vertical merge operator that conceptually calculates the union of all read accesses along a path until the first write occurs ($merge\_v^{u}$).
%\end{itemize}
Therefore, our vertical merge operator conceptually intersects all read accesses along a path until the first write 
occurs $merge\_v^{i}$. In any other case, it behaves like the previously mentioned vertical merge function.
%Our horizontal merge function is a simple pairwise combination of the given set of states:
%\begin{align}
%merge\_h(\{s\}) &= s\\
%merge\_h(\{s\} \cup s') &= s \circ merge\_h(s')
%\end{align}
Our horizontal merge $merge\_h$ function is a pairwise combination of the given set of states, which are then combined with an union-like operator 
with $W$ preceding $WR$ and $WR$ preceding $R$ and $R$ preceding $C$. Unless one side is $W$, read accesses are combined in such a way that always the higher one is selected.

%The results of our experiments with the implementation of calltarget classification gave presented us with essentially one possible candidate
%that we can base our horizontal merge function on, namely the union operator with an analysis function that follows into direct calls. The 
%basic schema of the merging is depicted in \ref{tbl:TYPECTunion} and it essentially behaves as if it was the union operator (when both states
%are set, the higher one is chosen). However, we have to account for W being used as an end marker, which is why we added mapping for RW, 
%which is essentially that. 
%
%\begin{table}[h]
%\centering
%% \resizebox{\columnwidth}{!}{%
%\begin{tabular}{c?c|c|c|c}
%$\bigcup^{\mathcal{L}}$  & C & R & W & RW\\
%\Xhline{1pt}
%C & C & R & W & RW\\
%\hline
%R & R & $\text{R}^{\cup}$ & W & $\text{R}^{\cup}$W\\
%\hline
%W & W & W & W & W\\Liveness analysis of a 
%\hline
%RW & RW & $\text{R}^{\cup}$W & W & RW\\
%\end{tabular}
%% }
%\caption{The union mapping operator for liveness in the \emph{type} policy.}
%\label{tbl:TYPECTunion}
%\end{table}

\subsubsection{Required Parameter Count} 
For our {count} policy, we need a coarse representation of the state of one register, thus we use the following representation.
(1) $W$ represents write before read access, 
(2) $R$ represents read before write access, and 
(3) $C$ represents the absence of access. 
Further, this gives us the $S^\mathcal{L} = \{ C, R, W \}$ as register state, which translates to the register super state $\mathcal{S}^\mathcal{L} = (S^\mathcal{L})^{16}$. 
We implement \texttt{merge\_v} in such a way that a state within a superstate is only updated if the corresponding register was not accessed, as represented by $C$. 
Our reasoning is that the first access is the relevant one in order to determine read before write.
Our horizontal \textit{merge($merge\_h$)} function is a simple pairwise combination of the given set of states, which are then combined with an union like operator with $W$ preceding $R$ and $R$ preceding $C$.
The index of highest parameter register based on the used call convention that has the state R considered to be the number of parameters a function at least requires to be prepared by a callsite.

%
%This leaves us with the two merge functions remaining undefined and we will leave the implementation of these and the interpretation of the  liveness state $\mathcal{S}^\mathcal{L}$ into parameters up to the following subsections.
%
%\textbf{Required Parameter Count.}
%\label{subsection:requiredparamcount}
%To implement the \emph{count} policy, we only need a coarse representation of the state of one register, thus we use the same representation as TypeArmor:
%\textit{1)} $W$ represents write before read access,
%\textit{2)} $R$ represents read before write access, and
%\textit{3)} $C$ represents the absence of access.
%
%This gives us the following register state $S^\mathcal{L} = \{ C, R, W \}$ which translates to the register super state $\mathcal{S}^\mathcal{L} = (S^\mathcal{L})^{16}$.
%We are only interested in the first occurrence of a $R$ or $W$ within one path, as following reads or writes do not give us more information. 
%Therefore, our vertical merge function ($merge\_v$) behaves in the following way that only when the first given state is $C$, 
%then the return value is represented by the second state. In all other cases it will return the first state.
%
%\begin{align}
%merge\_v^{r} (cur, delta) &= \left\{
%  \begin{array}{lr}
%     delta & cur = C \\
%     cur & otherwise
%  \end{array}
%\right. \\Liveness analysis of a 
%merge\_v (cur, delta) &= (s'_0, ... s'_15) \text { with } s'_j = merge\_v^{r}(cur_j, delta_j)
%\end{align}

%Our horizontal merge($merge\_h$) function is a simple pairwise combination of the given set of states, 
%which are then combined with an union like operator with $W$ preceding $R$ preceding $C$.
%\begin{align}
%merge\_h(\{s\}) &= s\\
%merge\_h(\{s\} \cup s') &= s \circ merge\_h(s')
%\end{align}

%
%We have three viable possibilities for our combination operator $\circ$, depicted in Table \ref{fig:COUNTlivenessmapping}, which all give priority to $W$:
%\begin{itemize}
%\item [$\bigsqcap^{\mathcal{L}}$] is what we call the destructive combination operator, as it returns W on any mismatch.
%\item [$\bigcap^{\mathcal{L}}$] is what we call the intersection operator, as it returns C, when combining C and R, similar to an intersection.
%\item [$\bigcup^{\mathcal{L}}$] is what we call the union operator, as it returns R, when combining C and R similar to an union.
%\end{itemize}


\newcolumntype{?}{!{\vrule width 1pt}}
%
%\begin{table}
%
%% \centering
%\resizebox{\columnwidth}{!}{%Liveness analysis of a 
%\begin{tabular}{c?c|c|c}
%$\bigsqcap^{\mathcal{L}}$ & C & R & W\\
%\Xhline{1pt}
%C & C & W & W\\
%\hline
%R & W & R & W\\
%\hline
%W & W & W & W
%\end{tabular}
%\begin{tabular}{c?c|c|c}
%$\bigcap^{\mathcal{L}}$  & C & R & W\\(merge vertically block states)
%\Xhline{1pt}
%C & C & C & W\\
%\hline
%R & C & R{todo} & W\\
%\hline
%W & W & W & W
%\end{tabular}
%\begin{tabular}{c?c|c|c}
%$\bigcup^{\mathcal{L}}$  & C & R & W\\
%\Xhline{1pt}
%C & C & R & W\\
%\hline
%R & R & R & W\\
%\hline
%W & W & W & W
%\end{tabular}}
%\caption{Different mappings for combining two liveness state values in horizontal matching for the \emph{count} policy.}
%
%\label{fig:COUNTlivenessmapping}
%\end{table}
%The index of highest parameter register based on the used call convention that has the state R is considered to be the number of parameters a function at least requires to be prepared by a callsite.

\subsubsection{Void/Non-Void Calltarget}
In order to determine if a calltarget is a void or non-void return function
\textsc{TypeShield} traverses backwards the basic blocks from the return instruction of the function an looks for the \texttt{RAX} register.
In case  there is a write operation on the \texttt{RAX} register then \textsc{TypeShield}
infers that the function return is non-void and thus provides a pointer value back.

% \subsubsection{Encountered Analysis Issues}
% \paragraph{Variadic Functions}
% \label{subsection:variadicfunctions}
% Variadic functions are a special type of C/C++ functions that have a basic set of parameters, 
% which they always require and a variadic set of parameters, which 
% may vary. A prominent example of this would be the $printf$ function, which is used 
% to output text to \texttt{stdout}.
% 
% % \begin{figure}[thp] % the figure provides the caption
% % \centering          % which should be centered
% % \begin{tabular}{c}  % the tabular makes the listing as small as possible and centers it
% % \footnotesize
% % \begin{lstlisting}
% % 00000000004222f0 <make_cmd>:
% %  4222f0:push   %r15
% %  4222f2:push   %r14
% %  4222f4:push   %rbx
% %  4222f5:sub    $0xd0,%rsp
% %  4222fc:mov    %esi,%r15d
% %  4222ff:mov    %rdi,%\begin{figure}[!h]
% %  422302:test   %al,%al
% %  422304:je     42233d <make_cmd+0x4d>
% %  422306:movaps %xmm0,0x50(%rsp)
% %  42230b:movaps %xmm1,0x60(%rsp)
% %  422310:movaps %xmm2,0x70(%rsp)
% %  422315:movaps %xmm3,0x80(%rsp)
% %  42231d:movaps %xmm4,0x90(%rsp)
% %  422325:movaps %xmm5,0xa0(%rsp)
% %  42232d:movaps %xmm6,0xb0(%rsp)
% %  422335:movaps %xmm7,0xc0(%rsp)
% %  42233d:mov    %r9,0x48(%rsp)
% %  422342:mov    %r8,0x40(%rsp)
% %  422347:mov    %rcx,0x38(%rsp)
% %  42234c:mov    %rdx,0x30(%rsp)
% %  422351:mov    $0x50,%esi
% %  422356:mov    %r14,%rdi
% %  422359:callq  409430 <pcalloc>
% % \end{lstlisting}
% % \end{tabular}
% % \caption{ASM code of the \texttt{make\_cmd} function with optimize level O2, which has a variadic parameter list.}
% % \label{fig:asmvariadic}
% % \end{figure}
% 
% \newsavebox{\firstlistingA}
% \begin{lrbox}{\firstlistingA}
% \begin{minipage}[c]{0.52\linewidth}
% \begin{minted}[
% % frame=lines,
% framesep=2mm,
% linenos,
% frame=none,
% firstnumber=1,
% linenos,
% highlightlines={10-17},
% highlightcolor={lightgray},
% numbersep=2pt,
% %gobble=2,
% %frame=lines,
% framesep=2mm,
% % fontsize=\tiny       
% fontsize=\scriptsize      
% % baselinestretch=1.2,
% % bgcolor=LightGray,
% % fontsize=\footnotesize,
% %use in terminal: pygmentize -L lexers to see all code highlighting options
% ]{asm}
% 00000000004222f0 <make_cmd>:
%  4222f0:push   %r15
%  4222f2:push   %r14
%  4222f4:push   %rbx
%  4222f5:sub    $0xd0,%rsp
%  4222fc:mov    %esi,%r15d
%  4222ff:mov    %rdi,%\begin{figure}[!h]
%  422302:test   %al,%al
%  422304:je     42233d <make_cmd+0x4d>
%  422306:movaps %xmm0,0x50(%rsp)
%  42230b:movaps %xmm1,0x60(%rsp)
%  422310:movaps %xmm2,0x70(%rsp)
%  422315:movaps %xmm3,0x80(%rsp)
%  42231d:movaps %xmm4,0x90(%rsp)
%  422325:movaps %xmm5,0xa0(%rsp)
%  42232d:movaps %xmm6,0xb0(%rsp)
%  422335:movaps %xmm7,0xc0(%rsp)
%  42233d:mov    %r9,0x48(%rsp)
%  422342:mov    %r8,0x40(%rsp)
%  422347:mov    %rcx,0x38(%rsp)
%  42234c:mov    %rdx,0x30(%rsp)
%  422351:mov    $0x50,%esi
%  422356:mov    %r14,%rdi
%  422359:callq  409430 <pcalloc>
% \end{minted}
% \end{minipage}
% \end{lrbox}
% \begin{figure}[H]
% \centering
% % \subfloat[]
% {\usebox{\firstlistingA}} 
% \caption{Assembly code of the \texttt{make\_cmd} function which was compiled with Clang -O2 flag, and has a variadic parameter list which is shaded gray above.}
% \label{fig:asmvariadic}
% \end{figure}
% 
% Figure~\ref{fig:asmvariadic} depicts the binary code of a variadic function which allows an easier processing of parameters
% due to the fact that all potential variadic parameters are moved into a contiguous block of memory.
% Our analysis interprets this functions as a read access on all parameters and thus, we arrive at a potentially problematic overestimation. 
% In our solution we opted to find these spurious reads and ignore them for now. A compiler will implement this type of operation very 
% similar for all cases, thus we can achieve our desired outcome using the following steps:
% (1) we search for (what we call) the xmm-passthrough block, which entirely consists of moving values of registers \texttt{xmm0} to \texttt{xmm7} into
% contiguous memory, (in our case basic block [\texttt{0x422306}, \texttt{0x42233d} [ );
% (2) we look at the predecessor of the xmm-passthrough block, which we call the entry block, next we %[\texttt{0x4222f0},\texttt{0x4222f2} [ )
% check if the successors of the entry block consist of the xmm-passthrough block and the successor of the xmm-passthrough block (we call the param-passthrough block), and
% (in our case basic block [\texttt{0x42233d} Liveness analysis of a, \texttt{0x42235e} [ );
% (3) we look at the param-passthrough block and set all instructions that move the value of a parameter register into memory to be ignored. 
% (in our case the instructions \texttt{0x42233d}, \texttt{0x422342}, \texttt{0x422347} and \texttt{0x42234c})
% 
% \paragraph{Ignoring Reads} When one instruction writes and reads a register at the same time, we give the read access precedence, however, there 
% are exceptions (also mentioned in TypeArmor). However, we expand slightly on that as follows:
% (1) \texttt{xor \%rax, \%rax} is the first scenario, as it will always result in \texttt{\%rax} holding the value 0,
% (2) \texttt{sub \%rax, \%rax} is the second scenario, as it results in \texttt{\%rax} also holding the value 0, and
% (3) \texttt{sbb \%rax, \%rax} is also relevant, however, it will not result in a constant value and based on the current state might either result in \texttt{\%rax} containing 0 or 1.

\subsection{Callsite Analysis}
\label{section:callsiteanalysis}
Our callsite analysis classifies callsites according to the parameters they provide. Overestimations are allowed, however,
underestimations are not permitted. For this purpose we employ a customizable modified reaching definition algorithm, 
which we will show first. 
% Furthermore, we will highlight some corner cases we encountered.

% \subsubsection{Reaching Definitions}
% \label{subsection:reachindefinitionstheory}
% An assignment to a variable is a reaching definition after the execution of a set of instruction if that variable still exists in at least one possible execution path. 
% If applied to a callsite, this calculates the values that are provided by this callsite to the function it then invokes. 
% 
% \begin{algorithm}[!ht]
% %         \scriptsize
%         \footnotesize
% 	\SetAlgoLined
% 	\SetKwInOut{Input}{Input}
%         \SetKwInOut{Output}{Output}
%         \Input{The basic block to be analyzed - $block$ : $\texttt{INSTR}^*$}
%         \Output{The reaching definition state - $\mathcal{S}^\mathcal{R}$}
%         \BlankLine
% 	\SetKwProg{Fn}{Function}{ is}{end}
% 	\Fn{\texttt{analyze}($block$ : $\texttt{INSTR}^*$) : $\mathcal{S}^\mathcal{R}$}
% 	{
%  	$state$ = Bl                                   \Comment*[r]{Initialize the state with first block}
%  	
%  	\ForEach{inst $\in$ reversed(block)}{
%  	
%  		$state' = analyze\_instr(inst)$        \Comment*[r]{Calculate changes}
%  		
% 		$state = merge\_v(state, state')$      \Comment*[r]{Merge changes}
% 	}
% 
% 	$states$ = $\emptyset$                         \Comment*[r]{Set of predecessor states}
% 	
% 	$blocks$ = $pred(block)$                         \Comment*[r]{Get predecessors blocks}
% 	
% 	\ForEach{block' $\in$ blocks} {
% 	
%  		$state' = analyze(block')$             \Comment*[r]{Analyze pred. block}
%  		
% 		$states$ = $states$ $\cup$ \{$state'$\}  \Comment*[r]{Add pred. states}
% 	}
% 
% 	$state'$ = $merge\_h (states)$                   \Comment*[r]{Merge predecessors states}
% 
% 	\Return $merge\_v(state, state')$              \Comment*[r]{Merge to final state}
% 
% 	}
% \caption{Basic block reaching definition analysis.}
% \label{alg:reaching}
% \end{algorithm}
% % \vspace{-.5cm}
% Algorithm~\ref{alg:reaching} is based on the reaching definition analysis presented in~\cite{khedker2009data}, 
% which can be regarded as a reverse depth-first traversal of basic blocks of a program. For customization, we rely on 
% the implementation of several functions. $\mathcal{S}^\mathcal{R}$ is the set of possible register states 
% which depends on the specific reaching definition implementation of the following operations.
% 
% $\bullet$ $\texttt{merge\_v} : \mathcal{S}^\mathcal{R} \times \mathcal{S}^\mathcal{R} \mapsto \mathcal{S}^\mathcal{R}$, (merge vertically block states) describes how to merge the current state with the following state change.
% 
% $\bullet$ $\texttt{merge\_h} : \mathcal{P}(\mathcal{S}^\mathcal{R}) \mapsto \mathcal{S}^\mathcal{R}$, (merge horizontally block states) describes how to merge a set of states resulting from several paths.
% 
% $\bullet$ $\texttt{analyze\_instr} : \texttt{INSTR} \mapsto \mathcal{S}^\mathcal{R}$, (analyze instruction) calculates the state change that occurs due to the given instruction.
% 
% $\bullet$ $\texttt{pred} : \texttt{INSTR}^* \mapsto \mathcal{P}(\texttt{INSTR}^*)$, (predecessor of a basic block) calculates the predecessors of the given block.
% 
% In our specific case, the function \texttt{analyze\_instr} does not need to handle normal predecessors, as DynInst will resolve those for us. 
% However, there are several instructions that have to be handled as depicted in the following situations. 
% (1) If the current instruction is an indirect call or a direct call and the analysis algorithm is set not to follow calls, then return a state where all registers are considered trashed. 
% (2) If the instruction is a direct call and the analysis algorithm is set to follow calls, then we start an analysis of the target function. 
% (3) In all other cases we simply return the decoded state. This leaves us with the two merge functions and the undefined reaching definitions state $\mathcal{S}^\mathcal{R}$. 
% 
% %The book~\cite{khedker2009data} defines reaching definition analysis on blocks in the following manner:
% %\begin{subequations}
% %\label{eq:reachingbasedef}
% %\begin{align}
% %In_n &:= \left\{
% %  \begin{array}{lr}
% %    Bl & \text{n is start block}\\
% %    \underset{p \in pred(n)}{\bigcup} Out_p & \text{otherwise}
% %  \end{array}
% %\right. \label{eq:reachingbasedefInt}\\
% %Out_n &:= (In_n - Kill_n) \cup Gen_n \label{eq:reachingbasedefOut}
% %\end{align}
% %\end{subequations}
% %$Bl$ is the default state at the start of a path of execution and in our case reaching that state would mean that we do not 
% %know whether a value has been provided for the variable and therefore we assume that one has been provided, reaching an 
% %overestimation. The set $Kill_n$ describes all definitions that are removed within this block, meaning that the value of 
% %a variable has been overwritten. The set $Gen_n$ describes the new definitions that have been provided by the block $n$, 
% %meaning that the value of a variable has been assigned. Considering this, we can assume that $Gen_n \subseteq Kill_n$, 
% %as we can always create new definitions, but not simply remove definitions without assigning a new value to the variable.
% 
% %
% %
% %However, we cannot use reaching definition analysis as is, because the analysis is again based on potentially unbound 
% %variable sets, while we are restricted to a finite number of registers and states. This time however, the analysis provides us with an overestimation, 
% %we however, want to get a result as close as possible so we again want to customize merge functions. Furthermore, we have to define how 
% %to interpret the changes occuring withing one block based on the the change caused by its instructions. Considering this, w, we 
% %arrive at algorithm \ref{alg:reaching} to compute the liveness state at the start of a basic block.
% 
% Previous work~\cite{khedker2009data} provides a reaching definition analysis on blocks, which we use to arrive at the algorithm depicted in 
% Algorithm~\ref{alg:reaching} to compute the liveness state at the start of a basic block. We apply the reaching analysis at each indirect 
% callsite directly before each call instruction.
% 
% This algorithm relies on various functions that can be used to configure its behavior. We define the 
% function $merge\_v$, which describes how to compound the state change of the current instruction and the current state, 
% the function $merge\_h$, which describes how to merge the states of several paths, the instruction analysis function
% $analyze\_instr$. Note, that the function $pred$, which retrieves all possible predecessors of a block 
% is provided by the DynInst instrumentation framework.
% 
% % \begin{subequations}
% % \label{eq:livenesscustom}
% % \begin{align}
% % merge\_v &: \mathcal{S}^\mathcal{R} \times \mathcal{S}^\mathcal{R} \mapsto \mathcal{S}^\mathcal{L}\\
% % merge\_h &: \mathcal{P}(\mathcal{S}^\mathcal{R}) \mapsto \mathcal{S}^\mathcal{R}\\
% % analyze\_instr &: \mathcal {I} \mapsto \mathcal{S}^\mathcal{R} \\
% % pred &: \mathcal{I} \mapsto \mathcal{P}(\mathcal{I})
% % \end{align}
% % \end{subequations}
% 
% The $analyze\_instr$ function calculates the effect of an instruction and is the core of the analyze function (see Algorithm~\ref{alg:reaching}). It will also 
% handle non-jump and non-fall-through successors, as these are not handled by DynInst in our case. We essentially have three cases that we handle:
% (1) If the instruction is an indirect call or a direct call but we chose not to follow calls, then return a state where all trashed are considered written,
% (2) If the instruction is a direct call and we chose to follow calls, then we spawn a new analysis and return its result, and
% %\item if the instruction is a constant write (\textit{e.g.,} xor of two registers) then we remove the read portion before we return the decoded state
% (3) In all other cases, we simply return the decoded state.
% 
% This leaves us with the two merge functions remaining undefined and we will leave the implementation of these and the interpretation of the 
% liveness state $\mathcal{S}^\mathcal{L}$ into parameters up to the following subsections.
% 
% %The book~\cite{khedker2009data} defines reaching definition analysis on blocks, which we use to arrive at algorithm depicted in Algorithm~\ref{alg:reaching} to compute 
% %the liveness state at the start of a basic block. We apply the reaching analysis at each indirect callsite directly before each call instruction.
% 
% %This algorithm relies on various functions that can be used to configure its behavior. We need to define the 
% %function $merge\_v$, which describes how to compound the state change of the current instruction and the current state, 
% %the function $merge\_h$, which describes how to merge the states of several paths, the instruction analysis function
% %$analyze\_instr$. The function $pred$, which retrieves all possible predecessors of a block won't be implemented by us, 
% %because we rely on the DynInst instrumentation framework to achieve the following.
% %\vspace{-.3cm}
% %\begin{subequations}
% %\label{eq:livenesscustom}
% %\begin{align}
% %merge\_v &: \mathcal{S}^\mathcal{R} \times \mathcal{S}^\mathcal{R} \mapsto \mathcal{S}^\mathcal{L}\\
% %merge\_h &: \mathcal{P}(\mathcal{S}^\mathcal{R}) \mapsto \mathcal{S}^\mathcal{R}\\
% %analyze\_instr &: \mathcal {I} \mapsto \mathcal{S}^\mathcal{R} \\
% %pred &: \mathcal{I} \mapsto \mathcal{P}(\mathcal{I})
% %\end{align}
% %\end{subequations}
% %\vspace{-.8cm}
% 
% %As the $analyze\_instr$ function calculates the effect of an instruction and is the core of the analyze function. It will also 
% %handle non jump and non fall-through successors, as these are not handled by DynInst in our case. We essentially have three cases that we handle:
% %\textit{1)} if the instruction is an indirect call or a direct call but we chose not to follow calls, then return a state where all trashed 
% %are considered written,
% %\textit{2)}  if the instruction is a direct call and we chose to follow calls, then we spawn a new analysis and return its result, and
% %\item if the instruction is a constant write (\textit{e.g.,} xor of two registers) then we remove the read portion before we return the decoded state
% %\textit{3)} in all other cases we simply return the decoded state.
% 
% %This leaves us with the two merge functions remaining undefined and we will leave the implementation of these and the interpretation of 
% %the liveness state $\mathcal{S}^\mathcal{L}$ into parameters up to the following subsections.
% 
% %We have yet to define the functions $merge\_v$, which describes how to compound a function and the outgoing state, the function $merge\_h$, which describes how to
% %merge the states of several paths and the function $pred$, which essentially gives us the predecessors of the current instruction. To prevent cycles we keep 
% %track of the instructions visited within the current path and omit any instruction on the current path from the result of $pred$. These functions, the 
% %reaching state $\mathcal{S}^\mathcal{R}$  and its interpretation into parameters will be defined in the following subsections.
% %
% %
% %\subsection{Backward Graph Traversal}
% %\label{subsection:backwardgraphtraversal}

\subsubsection{Provided Parameter Width}
\label{subsection:providedparamwideness}
In order to implement our {type} policy, we use a finer representation of the states of one register, thus we consider:
(1) $T$ represents a trashed register,
(2) $s8, s16, s32, s64 S$ represents a set register with  8-, 16-, 32-, 64-bit width, and
(3) $U$ represents an untouched register.
%\begin{itemize}
%\item Was the register value trashed ? \\ We represent this with the state T.
%\item Was the register written to and how much ? \\ We represent this with the states $\{ s64, s32, s16, s8 \}$ using S as a placeholder for arbitrary writes.
%\item Was the register neither trashed nor written to ? \\ We represent this with the state U.
%\end{itemize}
This gives us the following $S^\mathcal{L} = \{ T, s64, s32, s16, s8, U \}$ register state which translates to the register 
super state $\mathcal{S}^\mathcal{R} = (S^\mathcal{R})^{16}$.

However, we are only interested in the first occurrence of a state that is not $U$ in a path, as following reads or writes do not give us more information. Therefore, we can use 
the same vertical merge function as for the \emph{count} policy, which is essentially a pass-through until the first non $U$ state.

Our horizontal merge $merge\_h$ function is a simple pairwise combination of the given set of states, which are then combined with an union like operator with $T$ 
preceding $S$ and $S$ preceding $U$. Note, that when both states are set, we pick the higher one.

%
%Our horizontal merge function is again a simple pairwise combination of the given set of states:
%\begin{align}
%merge\_h(\{s\}) &= s\\
%merge\_h(\{s\} \cup s') &= s \circ merge\_h(s')
%\end{align}
%
%However, we have different possibilities regarding the merge operator. Experiments with our implementations for callsite 
%classification in the \emph{count} policy have given us the following results:
%\begin{itemize}
%\item The best candidate to minimize the problematic matches is the union operator without following direct calls.
%\item The best candidate to maximize precision is the intersection operator with following direct calls.
%\end{itemize}
%
%We therefore arrive at three viable possibilities for our combination operator $\circ$, depicted in table \ref{fig:TYPEreachingmapping}, 
%which all (except one) give priority to $T$:
%\begin{itemize}
%\item [$\bigcap^{\mathcal{R}}$] is what we call the intersection operator, as it returns U, when combining U and S, similar to an 
%intersection furthermore we also calculate the intersection of states when both states are set 
%(the lower of the two is returned).
%\item [$\bigsqcap^{\mathcal{R}}$] is what we call the half intersection operator, as it returns U, when combining U and S, 
%similar to an intersection but we calculate the union of states when both states are set (the higher of the two is returned).
%\item [$\bigcup^{\mathcal{R}}$] is what we call the union operator, as it returns S, when combining U and S similar to an union
%furthermore we calculate the union of states when both states are set (the higher of the two is returned).
%\end{itemize}
%
%\begin{table}
%
%% \centering
%\resizebox{\columnwidth}{!}{%
%\begin{tabular}{c?c|c|c}
%$\bigcap^{\mathcal{R}}$  & U & S & T\\
%\Xhline{1pt}
%U & U & U & T\\
%\hline
%S & U & $\text{S}^{\cap{}{}}$ & T\\
%\hline
%T & T & T & T
%\end{tabular}
%\begin{tabular}{c?c|c|c}
%$\bigsqcap^{\mathcal{R}}$  & U & S & T\\
%\Xhline{1pt}
%U & U & U & T\\
%\hline
%S & U & $\text{S}^{\cup{}{}}$ & T\\
%\hline
%T & T & T & T
%\end{tabular}
%\begin{tabular}{c?c|c|c}
%$\bigcup^{\mathcal{R}}$  & U & S & T\\
%\Xhline{1pt}
%U & U & S & T\\
%\hline
%S & S & $\text{S}^{\cup{}{}}$ & T\\
%\hline
%T & T & T & T
%\end{tabular}}
%
%\caption{Different mappings for combining two reaching state values in horizontal matching for the \emph{type} policy.}
%\label{fig:TYPEreachingmapping}
%\end{table}

\subsubsection{Provided Parameter Count}
\label{subsection:providedparamcount}
For implementing our {count} policy, we use a coarse representation of the state of one register, thus we use the following representation.
(1) $T$ represents a trashed register,
(2) $S$ represents a set register (written to), and
(3) $U$ represents an untouched register.
This gives us the following $S^\mathcal{L} = \{ T, S, U \}$  register state which translates to the register super state $\mathcal{S}^\mathcal{R} = (S^\mathcal{R})^{16}$.

We are only interested in the first occurrence of a $S$ or $T$ within one path, as following reads or writes do not give us more information. Therefore, our vertical 
merge function $merge\_v$ behaves as follows. In case the first given state is $U$, than the return value is the second state and in all other cases it will return the first state.
%
%We are only interested in the first occurrence of a S or T within one path, as following reads or writes do not give us more information.
%Therefore, we can define our vertical merge function in the following way:
%\begin{align}
%merge\_v^{r} (cur, delta) &= \left\{
%  \begin{array}{lr}
%     delta & cur = U \\
%     cur & otherwise
%  \end{array}
%\right. \\
%merge\_v (cur, delta) &= (s'_0, ... s'_15) \text { with } s'_j = merge\_v^{r}(cur_j, delta_j)
%\end{align}


Our horizontal merge $merge\_h$ function is a pairwise combination of the given set of states, which are then combined with an union like operator with $T$ preceding $S$ and $S$ preceding $U$.
%
%Our horizontal merge function is a simple pairwise combination of the given set of states:
%\begin{align}
%merge\_h(\{s\}) &= s\\
%merge\_h(\{s\} \cup s') &= s \circ merge\_h(s')
%\end{align}
%
%We have four viable possibilities for our combination operator $\circ$, depicted in table \ref{fig:COUNTreachingmapping}, which all (except one) give priority to $T$:
%\begin{itemize}
%\item [$\bigsqcap^{\mathcal{R}}$] is what we call the destructive combination operator, as it returns T on any mismatch.
%\item [$\bigcap^{\mathcal{R}}$] is what we call the intersection operator, as it returns U, when combining U and S, similar to an intersection.
%\item [$\bigcup^{\mathcal{R}}$] is what we call the union operator, as it returns S, when combining U and S similar to an union.
%\item [$\bigsqcup^{\mathcal{R}}$] is what we call the true union operator, as it gives S precedence over everything and returns T or 
%U only when both sides are T or U being more inclusive than an union.
%\end{itemize}

\newcolumntype{?}{!{\vrule width 1pt}}
%
%\begin{table}
%% \centering
%{
%\resizebox{\columnwidth}{!}{%
%\begin{tabular}{c?c|c|c}
%$\bigsqcap^{\mathcal{R}}$ & U & S & T\\
%\Xhline{1pt}
%U & U & T & T\\
%\hline
%S & T & S & T\\
%\hline
%T & T & T & T
%\end{tabular}
%\begin{tabular}{c?c|c|c}
%$\bigcap^{\mathcal{R}}$  & U & S & T\\
%\Xhline{1pt}
%U & U & U & T\\
%\hline
%S & U & S & T\\
%\hline
%T & T & T & T
%\end{tabular}
%\begin{tabular}{c?c|c|c}
%$\bigcup^{\mathcal{R}}$  & U & S & T\\
%\Xhline{1pt}
%U & U & S & T\\
%\hline
%S & S & S & T\\
%\hline
%T & T & T & T
%\end{tabular}
%\begin{tabular}{c?c|c|c}
%$\bigsqcup^{\mathcal{R}}$  & U & S & T\\
%\Xhline{1pt}
%U & U & S & T\\
%\hline
%S & S & S & S\\
%\hline
%T & T & S & T
%\end{tabular}}
%}
%
%\caption{Different mappings for combining two reaching state values in horizontal matching for the \emph{count} policy.}
%
%\label{fig:COUNTreachingmapping}
%\end{table}

The index of the highest parameter register based on the used call convention that has the state $S$ is considered to be the number of parameters a callsite prepares at most.

\subsubsection{Void/Non-Void Callsite}
In order to determine if a callsite is a void or non-void return function
\textsc{TypeShield} looks at the callsite if there is an read before write on the \texttt{RAX} register. 
In case there is a read before write operation on the \texttt{RAX} register then
\textsc{TypeShield} infers that the callsite is non-void and thus expects a pointer to be provided 
when the called function returns.

% \subsubsection{Encountered Analysis Issues}
% Our experiments with this implementation highlighted two issues.
% First, parameter lists with \textit{holes} and address width underestimation.
% Second, register extension instructions can be also a cause for analysis problems. 
% Finally, to reduce analysis runtime overhead, we also restricted the maximum path depth to 10 blocks.
% 
% \paragraph{Parameter Lists with \textit{Holes}.} This refers to parameter lists that show one or more \texttt{void} parameters between start to the last actual parameter. 
% These are not existent in actual code, but our analysis has the possibility of generating them through the merge operations. An example would be the following: 
% A parameter list of $(64, 0, 64, 0, 0, 0)$ is concluded, although the actual parameter list might be $(64, 32, 64, 0, 0, 0)$. While the trailing 0s are 
% what we expect, the 0 at the second parameter position will cause difficulties, because it is an underestimation at the single parameter level, which we need to avoid.
% Our solution relies on scanning our reaching analysis result for these holes and replace them with the wideness $64$, causing a possible overestimation.
% 
% \paragraph{Address Width Unterestimation.} This refers to the issue that while in the callsite a constant value of 32-bit is written to a register, the calltarget uses the 
% whole 64-bit register. This can occur when pointers are passed from the callsite to the calltarget. Specifically this happens 
% when pointers to memory inside the \texttt{.bss}, \texttt{.data} or \texttt{.rodata} section of the binary are passed.
% Our solution is to enhance our instruction analysis to watch out for constant writes. In case a 32-bit constant value write is detected, we check if the
% value is an address within the \texttt{.bss}, \texttt{.data} or \texttt{.rodata} section of the binary. If this is the case, we simply return a write access of 64-bit 
% instead of 32-bit. This is not problematic, because we are looking for an overestimation of parameter wideness.
% It should be noted that the same problem can arise when a constant write causes the value 0 to be written to a 32-bit register. We use the same solution
% and set the width to 64-bit instead of 32-bit.
% %
% %\subsection{Address Taken Analysis}
% %\label{section:addresstakenanalysis}
% %As of now, we use the maximum available set of calltargets---the set of all function entry basic blocks---as input for our algorithm. 
% %To restrict the number of calltargets per callsite even further, we explored the possibility of incorporating an address taken analysis
% %into our application. We base our theory on the paper by Zhang \textit{et al.}~\cite{mingwei:sekar}, which introduced various types of taken
% %addresses. An address is considered to be taken, when it is loaded into memory or a register.
% %
% %\textbf{Address Taken Targets.}
% %Based on the notions of \cite{mingwei:sekar}, which classified taken addresses into several types of indirect control flow targets,
% %we only chose { Code Pointer Constants (CK)} and discarded the others:
% %\begin{itemize}
% %
% %\item { Code Pointer Constants (CK)} are addresses that are calculated during the compilation of the binary and point within
% %the possible range of addresses in the current module or to instruction boundaries. We are however, only interested in addresses
% %that directly point to an entry basic block of a function, as these are the only valid targets for any callsite.
% %
% %\item { Computed code pointers (CC)} are the result of simple pointer arithmetic, however, these are only used for intra-procedural jumps. 
% %We rely on DynInst to resolve those and only focus on indirect callsites, therefore these are of no interest to us.
% %
% %\item{ Exception handling addresses (EH)} are used to handle exceptions within C++ functions and are modeled as jumps within the function. 
% %These are therefore within the normal control flow that we rely on DynInst to resolve for us.
% %
% %\item{ Exported function addresses (ES)} are essentially functions that point outside of our current module (usually to dynamically 
% %linked libraries) and are implemented as jumps, which are of no concern to us, because our analysis is only concerned about the current object.
% %
% %\item { Return addresses (RA)}, which are the addresses next to a call instruction, are also of no interest to us, because we only 
% %implement forward { control flow integrity}.
% %\end{itemize}
% %
% %\textbf{Binary Analysis.}
% %Our approach of identifying taken addresses consists of two steps: First, we iterate over the raw binary content of data sections. Second,
% %we iterate over all functions within the disassembled binary. We rely on DynInst to provide us with the boundaries of the sections inside
% %the binary and in case of shared libraries with the needed translation to current memory addresses:
% %
% %\begin{itemize}
% %\item We look at three different data sections of the binary, which could possibly contain taken addresses: the .data, .rodata and .dynsym
% %sections. As \cite{mingwei:sekar} proposed, we slide a four byte window over the data within those sections and look for addresses that
% %point to function entry blocks. However, we are looking at x64 binaries therefore we additionally use an eight byte window. In case of 
% %shared libraries, we need to let DynInst translate the raw address, we extracted, so we can perform the function check.
% %
% %\item We specifically look for instructions that load a constant value into a register or memory, and again check whether the address 
% %points to the entry block of a function.
% %\end{itemize}
% 
% %\section{Runtime Enforcement}
% %\label{section:runtimeenforcement}
% %
% %\subsection{Calltarget Annotation}
% %\label{subsection:patchingschema}
% %
% %\subsection{Callsite Instrumentation}
% %\label{subsection:patchingschema}

% \subsubsection{C++ exceptions}
% During our analysis we encountered no issues with respect to C++ exceptions.

\subsection{Backward-Edge Analysis}
\label{Backward Edge Analysis}
In order to protect the backward edges of our previously 
determined calltargets for each callsite we designed an
analysis which can determine possible legitimate return target addresses.

\begin{algorithm}[!ht]
%         \scriptsize
        \footnotesize
	\SetAlgoLined
	\SetKwInOut{Input}{Input}
        \SetKwInOut{Output}{Output}
        \Input{Forward edge callsite to calltargets map - $fMap$}
        \Output{Backward edge to return addresses map - $rMap$}
        \BlankLine
	\SetKwProg{Fn}{Function}{ is}{end}
	\Fn{\texttt{backwardAddressMapping}($fMap$) : $rMap$}
	{
	
 	\Comment*[l]{visit all detected callsites in the binary}
 	
 	\ForEach{callsite $\in$ $fMap$}{  
 	  
 	  \Comment*[l]{get calltargets for callsite address key}
 	  
 	  $calltargetSet$ = $getCalltargetSet(callsite, fMap)$\;
           
           \Comment*[l]{calltarget is the function start address}
           
           \Comment*[l]{visit all calltargets of a callsite}
           
          \ForEach{calltarget $\in$ $calltargetSet$}{
 	  
 	    \Comment*[l]{get the next address after the callsite}
 	    
 	    $rTarget$ = $getNextAddress(callsiteKey)$\;
 	    
 	    \Comment*[l]{find the address of function return}
 	    
 	    $rAddress$ = $getReturnOfCalltarget(calltarget)$\; 
 	    
 	    \Comment*[l]{rAddress is map key; rTarget is value}
 	    
 	    $rMap$ = $rMap$ $\cup$ $rMap \ add \ (rAddress, rTarget)$\;
 	   
 	  }
 	 }
        
        \Comment*[l]{return the backward-edgeaddresses mappings}
        
	\Return $rMap$\;                         

	}
\caption{Calltarget return set analysis.}
\label{alg:returns}
\end{algorithm}

Algorithm~\ref{alg:returns} depicts how the forward mapping between 
callsites and calltargets is used to determine the backward address set 
for each return address contained in each address taken function. The 
$fMap$ is obtained after running the callsite and 
calltarget analysis (see \cref{section:calltargetanalysis} and \cref{section:callsiteanalysis}). 
These mapping contains for each callsite the legal calltargets where the forward-edge
indirect control flow transfer is allowed to jump to. This mapping is reflected back by construction a second
mapping between the return address of each function for which we have the start 
address and a return target address set. 

The return target address set for a function return is determined by getting the next address after each callsite address 
which is allowed to make the forward-edge control flow transfer
(\textit{i.e.,} recall the caller callee calling convention).
The $rMap$ is obtained by visiting each function return address and assigning to it the address next to the callsite
which was used in order to transfer the control flow to the function in first place. 
At the end of the analysis all callsites and all function returns have been visited and a set for each function return address of backward-edgeaddresses will be obtained.
Note that the function boundary address (\textit{i.e.,} retn) was detected by a linear basic block search from the beginning of 
the function (calltarget) until the first return instruction was encountered. We are aware that other promising approaches for recuperating 
function boundaries (\textit{e.g.,}~\cite{function:boundary}) exist, and plan to experiment with them in future work.

\subsection{Binary Instrumentation}
\subsubsection{Forward-Edge Policy Enforcement}
\label{Enforcing The Forward Edge Policy}
The result of the forward callsite and calltarget analysis is a mapping between the allowed calltargets for each callsite.
In order to enforce this mapping during runtime each callsite and calltarget contained in the previous mapping are instrumented
inside the binary program with two labels and a callsite located CFI-based checking mechanism. At each callsite the number of 
provided parameters are encoded as a series of six bits. At the calltarget the label contains six bits denoting how 
many parameters the calltarget expects. Additionally, at the callsite six bits encode which register wideness types each of the provided parameters have 
while at the calltarget another six bits are used to encode the types of the parameters expected. Further, at the callsite another bit 
is used to define if the function is expecting a \texttt{void} return type or not. All this information are written in labels before
each callsite and calltarget. During runtime before each callsite these labels are compared by performing a xor operation between 
the bits contained in the previously mentioned labels. In case the xor operation returns false
than the transfer is allowed else the program execution is terminated.

\subsubsection{Backward-Edge Policy Enforcement}
The previously determined $rMap$ in Algorithm~\ref{alg:returns} will be used to insert a check before each 
function (calltarget) return present in the $rMap$. We propose a mode of operation based on a single
CFI check which can be inserted before each function return instruction. 

% \textbf{Super fast mode.} Based on the $rMap$, for each AT function return the minimum and the maximum address out of the return set for a particular 
% $rAdress$ return address will be determined. Next, these two values will be used to insert a range check having as left and right boundaries these
% two values. Before the return instruction of the function is executed the value of the function return is compared against these two values previously
% mentioned. In case the check fails than the program will be terminated else the indirect control flow transfer will be allowed.
% Note that this check has insignificant runtime overhead but on the other side it could contain not legitimate return addresses depending 
% on the entropy of the $rAddress$es. In short, this means that as far as the $min$ and $max$ addresses are from each other the more leeway the attacker will have. 

% \textbf{Fast mode.} 
Based on the $rMap$, before each AT function return a randomly generated label (\textit{i.e.,} the value 7232943 will be loaded trough one level of indirection) 
value will be inserted. The same label will be inserted before each legitimate (\textit{i.e.,} based on the forward-edge policy) target address (next address after a legitimate callsite) 
of the function return. 
In this way a function return will be allowed to jump to only the instruction which follows next to the 
address of the callsites which are allowed to call the calltarget which contains this particular function return. 
For callsites which are allowed to call the calltarget mentioned and another calltarget than in this cases \textsc{TypeShield} will perform a search in order to detect if the callsite
has already a label attached to the next address after the callsite. In this case the label will be reused. In this situation two callsites share their labels. The solution to this is to 
use single labels for each function return address. In this case multiple labels have to be stored for each address following a legitimate callsite.
Further, addresses located after a callsite that are not allowed to call a particular calltarget will get another randomly generated label. 
In this way calltarget return labels are grouped together based on the $rMap$. This mode of operation allows 
at least (additionally the callsites which are allowed to call more than one calltarget are added) the same number of function return sites as the 
forward-edge policy enforces for each callsite and it is runtime efficient since label checking is based on a single efficient 
compare check.

% \textbf{Slow mode.} Based on the $rMap$, before each AT function return a series of comparison checks are inserted in the binary. 
% Before the return instruction of the function a series of comparison checks between the appropriate 
% addresses stored in $rMap$ and the address where the function wants to return are performed. In case one of the check fails than the 
% program will be terminated. The total number of comparison checks added is equal to the size of return address set which contains $rTarget$ values. 
% Note that these types of checks are precise since only legitimate addresses are allowed but on the other side the runtime overhead is higher than in the 
% case of the fast path because the number of checks is in general higher.
 




\section{Implementation}
\label{chapter:Implementation}

We implemented \textit{TypeShield} as a module pass for the \textit{di-opt} environment based on PathArmor~\cite{veen:cfi}, 
which relies on the DynInst~\cite{bernat:dyninst} instrumentation framewewor (we are using version 9.2.0). 
However, converting to a standalone executable is also possible, as we do not rely on any patharmor feature
except for the pass abstraction.

Our module pass relies on DynInst, to resolve the structure of the binaries that we analyse. The core part 
of our pass is an instruction analyser, which relies on the DynamoRIO~\cite{dynamorio:drmemory} library 
(version 6.6.1) to decode single instructions and provide access to its information. This analyser is then
used to implement basic analyses, especially our version of the reaching and liveness analyses, which can
be customised with relative ease, as we allow for arbitrary path merging functions, however we provide 
the three basic versions (destructive, intersection and union).

Furthermore, we had to patch the DynInst library to allow for local annotation of calltargets with arbitrary
information, leveraging its relocation schema, which relies on the BasicBlock abstraction.

Additionally, to measure the quality and performance of our tool, we wrote a pass for the Clang/LLVM framework
version 4.0.0 (trunk 283889) in the x86 target code generation portion, to generate ground truth data. This
data is then used to verify the output of our tool for several testtargets, which is done in our python 
evaluation and test environment.

In total we implemented \textit{TypeShield} in 5123 lines of C++ code, our Clang/LLVM pass in 200 lines 
of C++ code and our test environment in 2674 lines of Python code (all lines of code data is given 
excluding empty lines and comments).

At this stage of development we are restricted to analysis and instrumentation of x86-64 bit elf 
binaries using the SystemV call convention, because the DynInst library does not yet support the
Windows platform. However, there is currently work being done to allow DynInst to also work with
Windows binaries. We restricted ourselves to the SystemV call convention as most C/C++ compilers
on linux implement this ABI, however we encapsulated most ABI dependent behaviour, so it should 
be possible to implement other ABIs with relative ease. Therefore, we deem it possible to implement
\textit{TypeShield} for the Windows platform in the near future, as we do not use any other 
platform-dependent API's. 


\chapter{Evaluation}
\label{chapter:Evaluation}
We evaluated our tool X with Y popular servers, by instrumenting them with our tool.
We performed runtime performance test with the following applications.

Our Evaluation aims to answer the following research questions:
\begin{itemize}
 \item \textbf{R1:} How efective is out tool in securing binary programs against the COOP attack?
 
 \item \textbf{R2:} How precise is our tool in detecting the types of the caller/caller pairs?
 
 \item \textbf{R3:} What is the performance overhead of our tool?
 
 \item \textbf{R4:} What are the instumentation overheads imposed by our tool 
 
 \item \textbf{R5:} How many caller/called pairs are secured by our tool and how many remain unsecured?
 
 \item \textbf{R6:} Against which kind of attacks can our tool secure programs?
 
 \item \textbf{R7:} What are the Limitations of our Tool?
 
 \item \textbf{R8:} List is not exauhustive. Give another relevant research question. if there is one.
 
\end{itemize}

\textbf{Comparison methods.} Example: We used UBSAN (compare with TypeArmor), the state-of-art
tool for detecting bad-casting bugs, as our comparison tar-
get of C A V ER . Also, We used C A V ER - NAIVE , which dis-
abled the two optimization techniques described in §4.4,
to show their effectiveness on runtime performance opti-
mization.

\textbf{Experimental setup.} Example: All experiments were run on
Ubuntu 13.10 (Linux Kernel 3.11) with a quad-core 3.40
GHz CPU (Intel Xeon E3-1245), 16 GB RAM, and 1 TB
SSD-based storage.


\section{R1: Effectiveness of our Tool}
\begin{table}[H]
\centering
\caption{Classification CS}
\label{Integer overflow bug detection in CWE-190}
\resizebox{.99\columnwidth}{!}{%
\begin{tabular}{|l|l|l|l|l|l|l|l|l|l|l|l|l|} \hline
\textbf{target}  &\textbf{opt} & \textbf{\#CS}     & \textbf{problems}    &\textbf{0} & \textbf{-1}  & \textbf{-2} &\textbf{-3} &\textbf{-4} &\textbf{-5} &\textbf{-6}  &\textbf{non-void-ok}  &\textbf{non-void-probl.}   \\ \hline 
x                &x            &x                  &x                     &x          &x             &x            &x           &x           &x           &x            &x                     &x                          \\ \hline

\end{tabular}}
%  \vspace{-2.5em}
\end{table}


\begin{table}[H]
\centering
\caption{Compound}
\label{Integer overflow bug detection in CWE-190}
\resizebox{.99\columnwidth}{!}{%
\begin{tabular}{|l|l|l|l|l|l|l|l|l|l|l|l|} \hline
\textbf{opt}  & \textbf{\#CS}     & \textbf{cs args (perfect \%)}    &\textbf{cs args (problem \%)} & \textbf{cs non-void (correct \%)}  & \textbf{cs non-void (probl. \%)} &\textbf{\#ct} &\textbf{ct args (perfect \%)} &\textbf{ct args (probl. \%)} &\textbf{ct void (correct \%)}  &\textbf{ct void (correct \%)}   \\ \hline 
x             &x                  &x                                 &x                             &x                                   &x                                 &x             &x                             &x                            &x                              &x                               \\ \hline

\end{tabular}}
%  \vspace{-2.5em}
\end{table}

\begin{table}[H]
\centering
\caption{Classification CT}
\label{Integer overflow bug detection in CWE-190}
% \resizebox{.99\columnwidth}{!}{%
\begin{tabular}{|l|l|l|l|l|l|l|l|l|l|l|l|l|} \hline
\textbf{target}  &\textbf{opt} & \textbf{\#CS}     & \textbf{problems}    &\textbf{0} & \textbf{-1}  & \textbf{-2} &\textbf{-3} &\textbf{-4} &\textbf{-5} &\textbf{-6}  &\textbf{non-void-ok}  &\textbf{non-void-probl.}   \\ \hline 
x                &x            &x                  &x                     &x          &x             &x            &x           &x           &x           &x            &x                     &x                          \\ \hline

\end{tabular}
% }
%  \vspace{-2.5em}
\end{table}


\begin{table}[H]
\centering
\caption{Callsite Classification for paramcount}
\label{Integer overflow bug detection in CWE-190}
% \resizebox{.99\columnwidth}{!}{%
\begin{tabular}{|l|l|l|l|l|l|l|l|l|l|} \hline
\textbf{target}  & \textbf{opt}     & \textbf{problematic}    &\textbf{+0} & \textbf{+1}  & \textbf{+2} &\textbf{+3} &\textbf{+4} &\textbf{+5} &\textbf{+6}  \\ \hline 
x                &x                 &x                        &x           &x             &x            &x           &x           &x           &x            \\ \hline

\end{tabular}
% }
%  \vspace{-2.5em}
\end{table}

\begin{table}[H]
\centering
\caption{Calltarget Classification}
\label{Integer overflow bug detection in CWE-190}
% \resizebox{.99\columnwidth}{!}{%
\begin{tabular}{|l|l|l|l|l|l|l|l|l|l|} \hline
\textbf{target}  & \textbf{opt}     & \textbf{problematic}    &\textbf{-0} & \textbf{-1}  & \textbf{-2} &\textbf{-3} &\textbf{-4} &\textbf{-5} &\textbf{-6}  \\ \hline 
x                &x                 &x                        &x           &x             &x            &x           &x           &x           &x            \\ \hline

\end{tabular}
% }
%  \vspace{-2.5em}
\end{table}

\begin{table}[H]
\centering
\caption{Coumpound table}
\label{Integer overflow bug detection in CWE-190}
% \resizebox{.99\columnwidth}{!}{%
\begin{tabular}{|l|l|l|l|l|l|} \hline
\textbf{target}  & \textbf{opt}     & \textbf{\#}    &\textbf{Callsites: param perf. \%, probl \%} & \textbf{\#}  & \textbf{Callsites: param perf. \%, probl \%}  \\ \hline 
x                &x                 &x               &x                                      &x             &x                                        \\ \hline

\end{tabular}
% }
%  \vspace{-2.5em}
\end{table}


\begin{table}[H]
\centering
\caption{MAtching table}
\label{Integer overflow bug detection in CWE-190}
% \resizebox{.99\columnwidth}{!}{%
\begin{tabular}{|l|l|l|l|l|l|l|l|l|} \hline
\textbf{target}  & \textbf{opt}     & \textbf{ct}    &\textbf{Ct probl.} & \textbf{at}  & \textbf{at prob.} &\textbf{cs} & \textbf{clang cs probl.}  & \textbf{padyn cs probl.}  \\ \hline 
x                &x                 &x               &x                  &x             &x                  &x           &x                          &x   \\ \hline

\end{tabular}
% }
%  \vspace{-2.5em}
\end{table}


\begin{table}[H]
\centering
\caption{policy evaluation}
\label{Integer overflow bug detection in CWE-190}
% \resizebox{.99\columnwidth}{!}{%
\begin{tabular}{|l|l|l|l|l|l|l|l|l|l|l|} \hline
\textbf{target}  & \textbf{opt}     & \textbf{policy}    &\textbf{0} & \textbf{1}  & \textbf{2} &\textbf{3} & \textbf{4}  & \textbf{5} & \textbf{6}  & \textbf{sumarry}  \\ \hline 
x                &x                 &x                   &x          &x            &x           &x          &x            &x           &x            &x \\ \hline

\end{tabular}
% }
%  \vspace{-2.5em}
\end{table}

\begin{table}[H]
\centering
\caption{param wideness}
\label{Integer overflow bug detection in CWE-190}
% \resizebox{.99\columnwidth}{!}{%
\begin{tabular}{|l|l|l|l|l|l|l|} \hline
\textbf{5}  & \textbf{4}     & \textbf{3}    &\textbf{2} & \textbf{1}  & \textbf{0} &\textbf{param/wideness}  \\ \hline 
x           &x               &x              &x          &x            &x           &0                         \\ \hline
x           &x               &x              &x          &x            &x           &8                         \\ \hline
x           &x               &x              &x          &x            &x           &16                         \\ \hline
x           &x               &x              &x          &x            &x           &32                         \\ \hline
x           &x               &x              &x          &x            &x           &64                         \\ \hline

\end{tabular}
% }
%  \vspace{-2.5em}
\end{table}

\begin{table}[H]
\centering
\caption{tabelle 7}
\label{Integer overflow bug detection in CWE-190}
% \resizebox{.99\columnwidth}{!}{%
\begin{tabular}{|l|l|l|l|l|l|l|} \hline
\textbf{5}  & \textbf{4}     & \textbf{3}    &\textbf{2} & \textbf{1}  & \textbf{0} &\textbf{param/wideness}  \\ \hline 
x           &x               &x              &x          &x            &x           &0                         \\ \hline
x           &x               &x              &x          &x            &x           &8                         \\ \hline
x           &x               &x              &x          &x            &x           &16                         \\ \hline
x           &x               &x              &x          &x            &x           &32                         \\ \hline
x           &x               &x              &x          &x            &x           &64                         \\ \hline

\end{tabular}
% }
%  \vspace{-2.5em}
\end{table}

\begin{table}[H]
\centering
\caption{tabelle 7}
\label{Integer overflow bug detection in CWE-190}
% \resizebox{.99\columnwidth}{!}{%
\begin{tabular}{|l|l|l|l|l|l|l|} \hline
\textbf{5}  & \textbf{4}     & \textbf{3}    &\textbf{2} & \textbf{1}  & \textbf{0} &\textbf{param/wideness}  \\ \hline 
x           &x               &x              &x          &x            &x           &0                         \\ \hline
x           &x               &x              &x          &x            &x           &8                         \\ \hline
x           &x               &x              &x          &x            &x           &16                         \\ \hline
x           &x               &x              &x          &x            &x           &32                         \\ \hline
x           &x               &x              &x          &x            &x           &64                         \\ \hline

\end{tabular}
% }
%  \vspace{-2.5em}
\end{table}

% http://www.texample.net/tikz/examples/
% 
% \begin{tikzpicture}[node distance=1cm, auto,]
%  %nodes
%  \node[punkt] (market) {Market (b)};
%  \node[punkt, inner sep=5pt,below=0.5cm of market]
%  (formidler) {Intermediaries (c)};
%  % We make a dummy figure to make everything look nice.
%  \node[above=of market] (dummy) {};
%  \node[right=of dummy] (t) {Ultimate borrower}
%    edge[pil,bend left=45] (market.east) % edges are used to connect two nodes
%    edge[pil, bend left=45] (formidler.east); % .east since we want
%                                              % consistent style
%  \node[left=of dummy] (g) {Ultimate lender}
%    edge[pil, bend right=45] (market.west)
%    edge[pil, bend right=45] (formidler.west)
%    edge[pil,<->, bend left=45] node[auto] {Direct (a)} (t);
% \end{tikzpicture}
% \vspace{1em}
% \emph{Impact of CFi and CFC example.}

alternative for abobe:
\begin{figure}[!ht]
  \caption{impact of CFI and CFC}
  \centering
    \includegraphics[width=0.5\textwidth]{figures/impact_of_cfi_and_cfc.pdf}
\end{figure}

\begin{figure}[!ht]
  \caption{liveness iteration, dummy}
  \centering
    \includegraphics[width=0.9\textwidth]{figures/liveness_iteration.pdf}
\end{figure}

\begin{figure}[!ht]
  \caption{reaching iteration, dummy}
  \centering
    \includegraphics[width=0.9\textwidth]{figures/reaching_iteration.pdf}
\end{figure}





\begin{table}[H]
\centering
\caption{matching}
\label{matching}
\resizebox{.99\columnwidth}{!}{%
\begin{tabular}{|l|l|l|l|l|l|l|l|l|} \hline
\textbf{target}  & \textbf{opt}     & \textbf{fn\_count}    &\textbf{fn\_problem} & \textbf{at\_count}  & \textbf{at\_problem} &\textbf{cs\_count} & \textbf{cs\_clang} &\textbf{cs\_padyn} \\ \hline 
x                &x                 &x                      &x                    &x                    &x                     &0                  &0                   &0                  \\ \hline


\end{tabular}
}
%  \vspace{-2.5em}
\end{table}

\begin{table}[H]
\centering
\caption{pairings compares}
\label{matching}
% \resizebox{.99\columnwidth}{!}{%
\begin{tabular}{|l|l|l|l|l|l|l|l|l|l|l|} \hline
\textbf{target}  & \textbf{opt}     & \textbf{policy} & \textbf{0}    &\textbf{1} & \textbf{2}  & \textbf{3} &\textbf{4} & \textbf{5} &\textbf{6}  &\textbf{summary} \\ \hline 
proftpd          &x                 &x                &x              &x          &x            &0           &0          &0           &0           &0     \\ \hline
proftpd          &x                 &x                &x              &x          &x            &0           &0          &0           &0           &0      \\ \hline
vsftpd           &x                 &x                &x              &x          &x            &0           &0          &0           &0           &0      \\ \hline
vsftpd           &x                 &x                &x              &x          &x            &0           &0          &0           &0           &0      \\ \hline

\end{tabular}
% }
%  \vspace{-2.5em}
\end{table}

\begin{table}[H]
\centering
\caption{policy baseline}
\label{matching}
% \resizebox{.99\columnwidth}{!}{%
\begin{tabular}{|l|l|l|l|l|l|l|l|l|l|l|} \hline
\textbf{target}  & \textbf{opt}     & \textbf{policy} & \textbf{0}    &\textbf{1} & \textbf{2}  & \textbf{3} &\textbf{4} & \textbf{5} &\textbf{6}  &\textbf{summary} \\ \hline 
proftpd          &x                 &x                &x              &x          &x            &0           &0          &0           &0           &0     \\ \hline
proftpd          &x                 &x                &x              &x          &x            &0           &0          &0           &0           &0      \\ \hline
vsftpd           &x                 &x                &x              &x          &x            &0           &0          &0           &0           &0      \\ \hline
vsftpd           &x                 &x                &x              &x          &x            &0           &0          &0           &0           &0      \\ \hline

\end{tabular}
% }
%  \vspace{-2.5em}
\end{table}


\section{R2: Precision of our Tool}

\section{R3: Performance overhead of our Tool}
\begin{table}[H]
\centering
\caption{Performance Tabble, more better is a bar chart here!}
\label{Integer overflow bug detection in CWE-190}
% \resizebox{.99\columnwidth}{!}{%
\begin{tabular}{|l|l|l|l|l|l|l|l|l|l|} \hline
\textbf{target}  & \textbf{opt}     & \textbf{problematic}    &\textbf{+0} & \textbf{+1}  & \textbf{+2} &\textbf{+3} &\textbf{+4} &\textbf{+5} &\textbf{+6}  \\ \hline 
x                &x                 &x                        &x           &x             &x            &x           &x           &x           &x            \\ \hline

\end{tabular}
% }
%  \vspace{-2.5em}
\end{table}
\section{R4: Instrumentation overhead of our Tool}

\section{R5: Security coverage of our tool}

\section{R6: Which kind of attacks can our tool defend off}

\section{R7: Whar are the limitations of our Tool}

\section{R8: To Do.}


it is easier for the reader if we can directly map those section from underneath on the section from above.

\section{Classification}
\subsection{Callsites}
overestimation param count. table.
number of parameters.

\subsection{Calltargets}
underestimation param table.

\section{Patching Policies}
Two types of diagrams. Table 5 from TypeArmor and a CDF to compare param count and param type. (baseline).
\subsection{AT}
\subsection{ParamCount}
table, cdf, baseline vs. server. approximations.

\subsection{ParamType}
table, cdf, baseline vs. server. approximations.

\section{Security Evaluation}

\section{Performance}
spec 2006.
% \chapter{Discussion}
\label{chapter:Discussion}

We have to define which points make sense and then talk about each other


punkt 1.

Punkt 2.


 
\chapter{Related Work}
\label{chapter:Related_Work}

\section{Binary-level CFI}
1/2 page

\section{Source-level CFI}
1/2 page

\section{TypeArmor Paper}
1/2 page
% \chapter{Future Work}
\label{chapter:Future_Work}

This chapter presents in section~\ref{Future Work} the future steps in order to improve the
precision and efficiency of \textit{TypeShild}.

\section{Future Work} 
\label{Future Work} 

In future we want to address the following points in order to increase the precision, efficiency
and coverage of \textit{TypeShild}.

\textbf{Type inference.} The type inference precision of function parameters in \textit{TypeShild} can be 
increased by blaaa....
\todo[inline]{to do}

\textbf{Function parameter counting.}
The counting of function parameters can be made more reliable by tackling the following points.
\todo[inline]{to do}
\section{Conclusion}
\label{chapter:Conclusion}
%version 1
% In this paper, we introduced \textsc{TypeShield} a tool for binary harding of forward indirect
% calls based on function parameter type and count.
% Advanced code reuse attacks such as COOP and its extensions or Control Jujutsu manifest due to a combination of facts and problems, 
% like memory corruption or predictable binary layout and the fact that the larger our binaries get, the 
% higher the chance they contain useful gadgets for an attacker. However, due to their nature, traditional 
% CFI cannot detect them, as they do not actually replace code to modify the control flow, but change pointers
% in memory, which redirects the targets of indirect callsites, which are uncertain at the time of compilation. 
% Two of the most common targets are the pointers to virtual function tables to implemented inheritance in C++ 
% and global function pointers. The control flow exhibited by the binary while functioning normal and while 
% under attack will seem the same. Address taken analysis alread helped cutting down the number of possible 
% calltargets one could inject by a considerable amount. And typearmor improved on that by implementing 
% invariants for both callsites and calltargets based on the number of parameters. We had no access to
% their sourcecode and therefore had to rely on their paper to implement an approximation for which 
% we generated comparable results regarding precision. We improved on that solution by implementing \textsc{TypeShield}, 
% which allows for a more fine-grained classification of calltargets and indirect callsites by implementing a rather 
% simplistic register wideness based type analysis. However, as simplistic as that analysis might be, we showed that 
% except for special cases (nginx), we were able to improve upon a parameter count based implementation by reducing 
% the average target count by about 20\%.



%version 2
% The family of forward indirect call based attacks which can manifest due to a series of factors such as
% memory corruptions, binary layout leackages and availability of useful
% gadgets in a sufficiently large executable poses a serious security threat.
In this paper, we presented \textsc{TypeShield}, a runtime fine-grained CFI-enforcing tool which 
operates on program binaries. Our tool precisely and efficiently filters legitimate from 
illegitimate forward indirect control flow transfers by using a novel runtime type-checking 
technique based on function parameter type-checking and parameter-counting. 
%Further, we maintain a comparable performance overhead to existing tools. (REDUNDANT WITH BELOW)

We have implemented~\textsc{TypeShield} and applied it to real software such as web servers, FTP servers and the SPEC CPU2006 benchmark. 
We demonstrated through extensive experiments that~\textsc{TypeShield} has 
higher precision than existing state-of-the-art tools, while maintaining a comparable runtime performance overhead. 
To date, we were able to improve on parameter count based techniques by reducing the possible calltargets per 
callsite ratio by 20\% with an overall reduction of about 9\% when comparing with similar state-of-the-art approaches. 
Next to a more precise analysis, the tangible outcome is a considerably reduced attack surface.
In the spirit of open research, we have made the source code of \textsc{TypeShield} and the evaluation results publicly available, 
thus we support reproducibility in this fast-moving research field by providing comprehensive descriptions of our experiments.




%todos list
% \listoftodos

%the bibliography

\bibliographystyle{IEEEtran} 
% \bibliographystyle{sigplanconf-eurosys} 
\bibliography{literatur}  



% \theendnotes

%appendix, taken out for now.
% \section*{Appendix}
\label{appendix}
todo.

\end{document}
