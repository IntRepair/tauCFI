\section{Conclusion}
\label{chapter:Conclusion}
%version 1
% In this paper, we introduced \textsc{TypeShield} a tool for binary harding of forward indirect
% calls based on function parameter type and count.
% Advanced code reuse attacks such as COOP and its extensions or Control Jujutsu manifest due to a combination of facts and problems, 
% like memory corruption or predictable binary layout and the fact that the larger our binaries get, the 
% higher the chance they contain useful gadgets for an attacker. However, due to their nature, traditional 
% CFI cannot detect them, as they do not actually replace code to modify the control flow, but change pointers
% in memory, which redirects the targets of indirect callsites, which are uncertain at the time of compilation. 
% Two of the most common targets are the pointers to virtual function tables to implemented inheritance in C++ 
% and global function pointers. The control flow exhibited by the binary while functioning normal and while 
% under attack will seem the same. Address taken analysis alread helped cutting down the number of possible 
% calltargets one could inject by a considerable amount. And typearmor improved on that by implementing 
% invariants for both callsites and calltargets based on the number of parameters. We had no access to
% their sourcecode and therefore had to rely on their paper to implement an approximation for which 
% we generated comparable results regarding precision. We improved on that solution by implementing \textsc{TypeShield}, 
% which allows for a more fine-grained classification of calltargets and indirect callsites by implementing a rather 
% simplistic register wideness based type analysis. However, as simplistic as that analysis might be, we showed that 
% except for special cases (nginx), we were able to improve upon a parameter count based implementation by reducing 
% the average target count by about 20\%.



%version 2
% The family of forward indirect call based attacks which can manifest due to a series of factors such as
% memory corruptions, binary layout leackages and availability of useful
% gadgets in a sufficiently large executable poses a serious security threat.
In this paper, we presented \textsc{TypeShield}, a runtime fine-grained CFI-policy enforcing tool which 
operates on program binaries. Our tool precisely and efficiently filters legitimate from 
illegitimate forward indirect control flow transfers by using a novel runtime type-checking 
technique based on function parameter type-checking and parameter-counting. 
%Further, we maintain a comparable performance overhead to existing tools. (REDUNDANT WITH BELOW)

We have implemented~\textsc{TypeShield} and applied it to real software such as web servers, FTP servers and the SPEC CPU2006 benchmark. 
We demonstrated through extensive experiments that~\textsc{TypeShield} has 
higher precision w.r.t. the calltarget set per callsite than existing state-of-the-art tools, while maintaining a comparable runtime performance overhead of 4\%. 
To date, we were able to improve on parameter count based techniques by reducing the possible calltargets per 
callsite ratio by 35\% with an overall reduction of more than 13\% when comparing with similar state-of-the-art tools. 
Next to a more precise analysis, the tangible outcome is a considerably reduced attack surface which can be further improved by 
tweaking our analysis. Finally, in the spirit of open research, we have made the source code of \textsc{TypeShield} and the evaluation results publicly available, 
thus we support reproducibility in this fast-moving research field by providing comprehensive descriptions of our experiments.


