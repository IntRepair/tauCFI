\section{Design}
\label{chapter:Design}

In this section, we present 
in~\cref{Parameter Count Based Policy} we present our function parameter count based policy,
in \cref{section:typepolicy}, we present the details of our type policy, and 
in~\cref{section:instructionanalysis} we introduce the definitions for our instructions analysis based on register states,  
in~\cref{section:calltargetanalysis} we present the design of our calltarget analysis, while
in~\cref{section:callsiteanalysis} we depict the design of our callsite analysis\footnote{Callsites detection in the 
binary is based on the capabilities of DynInst.}.
Finally,  
in~\cref{Enforcing The Forward Edge Policy} we present our forward-edge policy instrumentation strategy, and 
in~\cref{Backward Edge Analysis} we highlight our function backward-edgeanalysis and policy instrumentation strategy.

\subsection{Parameter Register Wideness Based Policy}
\label{section:typepolicy}

% \begin{figure}[h!]
% \center
% \resizebox{.65\columnwidth}{!}{
% \begin{tikzpicture}
% 
% \fill[black!10!white] (0,7.5) rectangle (12,6);
% \fill[black!20!white] (0,6) rectangle (12,4.5);
% \fill[black!30!white] (0,4.5) rectangle (12,3);
% \fill[black!40!white] (0,3) rectangle (12,1.5);
% \fill[black!50!white] (0,1.5) rectangle (12,0);
% 
% \draw[-triangle 45, thick] (-0.5,0) -- node[sloped, anchor=center, above] {\LARGE{growing required bits/parameter}} (-0.5,7.5);
% \draw[-triangle 45, thick] (12.5,7.5) -- node[sloped, anchor=center, above] {\LARGE{growing provided bits/parameter}} (12.5,0);
% 
% \draw (0,7.5)  --node[anchor=south] {\rot{\LARGE{p. 6}}} (2,7.5);
% \draw (2,7.5)  --node[anchor=south] {\rot{\LARGE{p. 5}}} (4,7.5);
% \draw (4,7.5)  --node[anchor=south] {\rot{\LARGE{p. 4}}} (6,7.5);
% \draw (6,7.5)  --node[anchor=south] {\rot{\LARGE{p. 3}}} (8,7.5);
% \draw (8,7.5)  --node[anchor=south] {\rot{\LARGE{p. 2}}} (10,7.5);
% \draw (10,7.5) --node[anchor=south] {\rot{\LARGE{p. 1}}} (12,7.5);
% 
% \draw (0,7.5)  rectangle node[anchor=center] {\LARGE{0-bits}} (2,6);
% \draw (2,7.5)  rectangle node[anchor=center] {\LARGE{0-bits}} (4,6);
% \draw (4,7.5)  rectangle node[anchor=center] {\LARGE{0-bits}} (6,6);
% \draw (6,7.5)  rectangle node[anchor=center] {\LARGE{0-bits}} (8,6);
% \draw (8,7.5)  rectangle node[anchor=center] {\LARGE{0-bits}} (10,6);
% \draw (10,7.5) rectangle node[anchor=center] {\LARGE{0-bits}} (12,6);
% 
% \draw (0,6)  rectangle node[anchor=center] {\LARGE{8-bits}} (2,4.5);
% \draw (2,6)  rectangle node[anchor=center] {\LARGE{8-bits}} (4,4.5);
% \draw (4,6)  rectangle node[anchor=center] {\LARGE{8-bits}} (6,4.5);
% \draw (6,6)  rectangle node[anchor=center] {\LARGE{8-bits}} (8,4.5);
% \draw (8,6)  rectangle node[anchor=center] {\LARGE{8-bits}} (10,4.5);
% \draw (10,6) rectangle node[anchor=center] {\LARGE{8-bits}} (12,4.5);
% 
% \draw (0,4.5)  rectangle node[anchor=center] {\LARGE{16-bits}} (2,3);
% \draw (2,4.5)  rectangle node[anchor=center] {\LARGE{16-bits}} (4,3);
% \draw (4,4.5)  rectangle node[anchor=center] {\LARGE{16-bits}} (6,3);
% \draw (6,4.5)  rectangle node[anchor=center] {\LARGE{16-bits}} (8,3);
% \draw (8,4.5)  rectangle node[anchor=center] {\LARGE{16-bits}} (10,3);
% \draw (10,4.5) rectangle node[anchor=center] {\LARGE{16-bits}} (12,3);
% 
% \draw (0,3)  rectangle node[anchor=center] {\LARGE{32-bits}} (2,1.5);
% \draw (2,3)  rectangle node[anchor=center] {\LARGE{32-bits}} (4,1.5);
% \draw (4,3)  rectangle node[anchor=center] {\LARGE{32-bits}} (6,1.5);
% \draw (6,3)  rectangle node[anchor=center] {\LARGE{32-bits}} (8,1.5);
% \draw (8,3)  rectangle node[anchor=center] {\LARGE{32-bits}} (10,1.5);
% \draw (10,3) rectangle node[anchor=center] {\LARGE{32-bits}} (12,1.5);
% 
% \draw (0,1.5)  rectangle node[anchor=center] {\LARGE{64-bits}} (2,0);
% \draw (2,1.5)  rectangle node[anchor=center] {\LARGE{64-bits}} (4,0);
% \draw (4,1.5)  rectangle node[anchor=center] {\LARGE{64-bits}} (6,0);
% \draw (6,1.5)  rectangle node[anchor=center] {\LARGE{64-bits}} (8,0);
% \draw (8,1.5)  rectangle node[anchor=center] {\LARGE{64-bits}} (10,0);
% \draw (10,1.5) rectangle node[anchor=center] {\LARGE{64-bits}} (12,0);
% \end{tikzpicture}
% }
% \caption{Type policy schema for callsites and calltargets.}
% \label{fig:TYPEschema}
% \vspace{-.5cm}
% \end{figure}
% 
% Figure~\ref{fig:TYPEschema} depicts the basic principle of our function parameter type policy.
% Note that p. means parameter. As it is depicted in this example, when requiring parameter width, one starts at the 
% bottom of the above matrix and grows to the top, as it is always possible to accept more parameters than required.
% Also, the reverse is true for providing parameters, as it is possible to accept less parameters than provided.
% Note that accepting more parameters than provided is not allowed.

We use the register width of the function parameter in order to infer the type information. As previously mentioned, there are 4 types 
of reading and writing accesses. Therefore, our set of possible types for parameters is $\texttt{TYPE} = \{64, 32, 16, 8, 0\}$; where 0 models the absence of a 
parameter. Since Itanium C++ ABI specifies 6 registers (\textit{i.e.,} \texttt{rdi}, \texttt{rsi}, \texttt{rdx}, \texttt{rcx}, \texttt{r8}, and \texttt{r9}) as 
parameter passing registers during function calls, we classify our callsites and calltargets into $\texttt{TYPE}^6$. Similar to
our count policy, we allow overestimations of callsites and underestimations of calltargets, on the parameter types as well. Therefore, for a 
callsite $cs$ it is possible to call a calltarget $ct$, only if for each parameter of $ct$ the corresponding parameter of $cs$ is not smaller w.r.t. the register width.
This results in a finer-grained policy which is further restricting the possible set of calltargets for each callsite.

%What we call the \emph{type} policy is the idea of not only relying on the parameter count but also on the parameter type. However, due to complexity reasons,
%we are restricting ourselves to the general purpose registers, which the Intanium C++ ABI designates as parameter registers. Furthermore, we are not inferring 
%the actual type of the data but the wideness of the data stored in the register. The schema again is that we have calltargets requiring wideness and the
%callsite providing it as depicted in Figure \ref{fig:TYPEschema}.

%e are currently interested in x86-64 binaries, the registers we are looking at are 64-bit registers that can be accessed in four different ways:
%\textit{1)} the whole 64-bit of the register, meaning a wideness of 64,
%\textit{2)} the lower 32-bit of the register, meaning a wideness of 32,
%\textit{3)} the lower 16-bit of the register, meaning a wideness of 16, and
%\textit{4)} the lower 8-bit of the register, meaning a wideness of 8.

%Four of those registers can also directly access the higher 8-bit of the lower 16-bit of the register. For our purpose we register this access as a 16-bit access. 

%Based on this information,we can assign a register one of 5 possible types $\mathcal{T} = \{64, 32, 16, 8, 0\}$. We also included the type 0 to model the absence of data within a register. 
%Similar to the \emph{count} policy, we allow overestimation of types in callsites and underestimation of types in calltargets. However, the matching idea is different, 
%because as can we depict in Figure \ref{fig:TYPEschema}, the type of a calltarget and a callsite no longer depends solely on its parameter count, 
%each callsite and calltarget has its type from the set of $\mathcal{T}^6$, with the following comparison operator:
%$
%	u \leq_{type} v :\Longleftrightarrow  
%	\forall_{i = 0}^{5} {u_i \leq v_i} , \text {with } u, v \in \mathcal{T}^6
%$.

%Again we allow any callsite $cs$ call any calltarget $ct$, when it fulfills the requirement $ct \leq cs$. 
%The way we represent this is by letting the type for a calltarget parameter progress from 64-bit to 0-bit---if a calltarget requires a 32-bit value in its 1st 
%parameter, it also should accept a 64-bit value from its callsite---and similarly we let the type for a callsite progress from 0-bit to 64-bit - If a 
%callsite provides a 32-bit value in its 1st parameter it also provides a 16-bit, 8-bit and 0-bit to a calltarget. Now the advantage of the \emph{type} policy
%in comparison to the \emph{count} policy is that while our type comparison implies the count comparison, the other direction does not hold.
%Meaning, just having an equal or lesser number of parameters than a callsite, does no longer allow a calltarget being called there, thus restricting the number of calltargets per 
%callsite even further. A function that requires 64-bit in its first parameter, and 0-bit in all other parameters, would have been callable by a callsite providing 8-bit 
%in its first and second parameter when using the \emph{count} policy, however in the \emph{type} policy this is no longer possible. Thus, it should decrease the number of targets per bucket.

% \subsection{Parameter Count Based Policy}
% \label{Parameter Count Based Policy}
% \textbf{\emph{Count} Policy.}
% \label{section:countpolicy}
% \vspace{-1.1em}
% \begin{figure}[!h]
% \centering
% \resizebox{0.2\textwidth}{!}{
% \begin{tikzpicture}
% 
% %\fill[black!40!white] (0,0) rectangle (9,9);
% %\fill[black!30!white] (0,0) rectangle (7,7);
% %\fill[black!20!white] (0,0) rectangle (4,4);
% %\fill[black!10!white] (0,0) rectangle (2,2);
% %
% %\draw (0,0) --node[anchor=south] {0 params}  (2,0)  -- (2,2) -- (0,2) -- (0,0) ;
% %\draw (0,0) -- (2,0) --node[anchor=south] {1 param} (4,0) -- (4,4) -- (0,4) -- (0,0);
% %\draw (0,0) --(4,0) --node[anchor=south] {2 ... 5 params} (7,0) -- (7,7) -- (0,7) -- (0,0);
% %\draw (0,0) --(7,0) --node[anchor=south] {6 params} (9,0) -- (9,9) -- (0,9) -- (0,0);
% %\draw[dashed] (4,4) -- (7,7);
% 
% 
% \fill[black!00!white] (0,7) rectangle (6,6);
% \fill[black!10!white] (0,6) rectangle (6,5);
% \fill[black!20!white] (0,5) rectangle (6,4);
% \fill[black!30!white] (0,4) rectangle (6,3);
% \fill[black!40!white] (0,3) rectangle (6,2);
% \fill[black!50!white] (0,2) rectangle (6,1);
% \fill[black!60!white] (0,1) rectangle (6,0);
% 
% 
% \draw[-triangle 45, thick] (-0.5,0) -- node[sloped, anchor=center, above] {\Large{growing required \# of bytes per param.}} (-0.5,7);
% \draw[-triangle 45, thick] (6.5,7) -- node[sloped, anchor=center, above]  {\Large{growing provided \# of bytes per param.}} (6.5,0);
% 
% \draw (0,7) rectangle node[anchor=center] {\LARGE{0 parameters}} (6,6);
% 
% \draw (0,6) rectangle node[anchor=center] {\LARGE{1 parameter}}  (6,5);
% 
% \draw (0,5) rectangle node[anchor=center] {\LARGE{2 parameters}} (6,4);
% 
% \draw (0,4) rectangle node[anchor=center] {\LARGE{3 parameters}} (6,3);
% 
% \draw (0,3) rectangle node[anchor=center] {\LARGE{4 parameters}} (6,2);
% 
% \draw (0,2) rectangle node[anchor=center] {\LARGE{5 parameters}} (6,1);
% 
% \draw (0,1) rectangle node[anchor=center] {\LARGE{6 parameters}} (6,0);
% 
% 
% \end{tikzpicture}
% }
% \caption{Callsite \& calltargets {count} policy classification.}
% \label{fig:COUNTschema}
% \vspace{-.5cm}
% \end{figure}
% 
% Figure~\ref{fig:COUNTschema} depicts the used matching schema which shows that calltargets require
% parameters whereas callsites provide these parameters. 

Further, we built a function parameter count-based policy similar to~\cite{veen:typearmor}. Calltargets are classified based on the number of parameters 
that these provide and callsites are classified by the number of parameters that these require. 
Further, we consider the generation of high precision measurements for such classification with binaries as the only source of information rather difficult. 
Therefore, over-estimations of parameter count for callsites and underestimations of the parameter count for calltargets is deemed acceptable. 
This classification is based on the general purpose registers that the call convention of the current ABI---in this case the 
Itanium C++ ABI~\cite{itanium:abi}---designates as parameter registers. Furthermore, we do not consider floating point registers or multi-integer registers for simplicity
reasons.
The \emph{count} policy is based on allowing any callsite $cs$, which provides $c_{cs}$ parameters, to call any calltarget $ct$, 
which requires $c_{ct}$ parameters, iff $c_{ct} \leq c_{cs}$ holds. However, the main problem is that while there is a significant 
restriction of calltargets for the lower callsites, the restriction capability drops rather rapidly when reaching higher parameter 
counts, with callsites that use 6 or more parameters being able to call all possible calltargets.
This is more precisely expressed as 
$\forall$ $cs_1$, $cs_2$; $c_{cs_1}$ $\leq$ $c_{cs_2}$ $\rightarrow$  $\|$ $\{ct \in \mathcal{F}$ $|$ $c_{ct}$ $\leq$ $c_{cs_1}$ $\} \| \leq$ $\|$ $\{ct \in \mathcal{F} | c_{ct} \leq c_{cs_2}  \} \|$.

One possible remedy would be the ability to introduce an upper bound for the classification deviation of parameter counts, 
however, as of now, this does not seem feasible with current technology. Another possibility would be the overall reduction
of callsites, which can access the same set of calltargets, a route which we will explore within this work.

\subsection{Analysis of Register States}
\label{section:instructionanalysis}

Our register state analysis is register state based, another alternative would be to do symbol-based data-flow analysis which we will leave as future work.
In order for the reader to understand our analysis we will first give some definitions.
The set $\texttt{INSTR}$ describes all possible instructions that can occur within the executable section of a program binary. In our case,
this is based on the x86-64 instruction set. An instruction $i \in \texttt{INSTR}$ can non-exclusively perform two kinds of operations on any number of existing 
registers. Note that there are registers that can directly access the higher 
8-bit of the lower 16-bit. For our purpose, we register this access as a 16-bit access.
(1) Read $n$-bit from the register with $n \in \{ 64, 32, 16, 8 \}$, and 
(2) Write $n$-bit to the register with $n \in \{ 64, 32, 16, 8 \}$.

Next, we describe the possible change within one register as $\delta \in \Delta$ with $\Delta = \{ w64, w32, w16, w8, 0 \} \times \{r64, r32, r16, r8, 0 \}$. 
Note that 0 represents 
the absence of either a write or read access and $(0, 0)$ represents the absence of both. Furthermore, $wn$ or $rn$ with $n \in \{64,32,16,8\}$ implies all $wm$ or $rm$ with $m \in 
\{64,32,16,8\}$ and $m < n$ (\textit{e.g.,} $r64$ implies $r32$). Note that we exclude 0, as it means the absence of any access.
Intanium C++ ABI specifies 16 general purpose integer registers. Therefore, we represent the change occurring at the processor level as $\delta_p \in \Delta^{16}$. 
In our analysis, we calculate this change for each instruction $i \in \texttt{INSTR}$ via the function $decode : \texttt{INSTR} \mapsto \Delta^{16}$.

% Finally, in \cref{section:addresstakenanalysis} we introduce a version of 
% address taken analysis based on \cite{mingwei:sekar} to restrict the number of available calltargets even more. 
%At last we introduce a patching schema for callsites and calltargets to enforce the invariants we inferred.

%Usually data-flow analysis algorithms are based on set of variable or sets of definitions, which both are basically unbounded. However, we are analyzing the state of registers, 
%which are baked into hardware and therefore their number is given, thus requiring us to adapt the data-flow theory to work on tuples.
%
%The set $\mathcal{I}$ describes all possible instructions that can occur within the executable section of a binary. In our case this is based on the instruction set for x86-64 processors.
%
%An instruction $i \in \mathcal{I}$ can non-exclusively perform two kinds of operations on any number of existing registers:
%\textit{1)} Read $n$-bit from the register with $n \in \{ 64, 32, 16, 8 \}$, and
%\textit{2)} Write $n$-bit to the register with $n \in \{ 64, 32, 16, 8 \}$.
%
%Thus, we describe the possible change that occurs in one register with the set $S = \{ w64, w32, w16, w8, 0 \} \times \{r64, r32, r16, r8, 0 \}$. Note that 0 signals the absence
%of either a write or read access and $(0, 0)$ signals the absence of both. Furthermore, $wn$ or $rn$ with $n \in \{64,32,16,8\}$ implies all $wm$ or $rm$ with $m \in \{64,32,16,8\}$ 
%and $m < n$ (\textit{e.g.,} $r64$ implies $r32$). Note that we exclude 0, as it means the absence of any access.

%Intanium C++ ABI specifies 16 general purpose integer registers, thus for our purpose we represent the change occurring at the processor level as $\mathcal{S} = S^{16}$.

%At last we declare a function, which calculates the change occurring in the processor state, when executing an instruction from $\mathcal{I}$:
%$decode : \mathcal{I} \mapsto \mathcal{S}$.

%However, we do not go into detail how this function actually calculates this sate, because we rely on external libraries to perform this task. Implementing this function our self 
%is out of scope due to the lengthy work required, as the x86-64 instruction set is quite large.

\subsection{Calltarget Analysis}
\label{section:calltargetanalysis}
Our calltarget analysis classifies calltargets according to the parameters they expect. Underestimations are allowed, however, overestimations 
are not permitted. For this purpose, we employ a customizable modified liveness analysis 
algorithm, which iterates over address-taken\footnote{A program function is defined to have its address taken if there is at least one binary instruction
which loads the function entry point into memory. Note that by definition, indirect calls can only target AT functions.} functions 
with the goal of analyzing register state information in order to determine if these registers are used for arguments passing.
% Furthermore, we will present certain corner cases we encountered, at the end of this section.

\subsubsection{Liveness Analysis}
A variable is alive before the execution of an instruction, if at least one of the originating paths performs a read access before any write access on that variable. 
If applied to a function, this calculates the variables that need to be alive at the beginning, as these are its parameters. 

\begin{algorithm}[h!]
%         \scriptsize
        \footnotesize
 	\SetAlgoLined
	\SetKwInOut{Input}{Input}
        \SetKwInOut{Output}{Output}
        \Input{The basic block to be analyzed - block : $\texttt{INSTR}^*$}
        \Output{The liveness state - $\mathcal{S}^\mathcal{L}$}
        \BlankLine
	\SetKwProg{Fn}{Function }{ is}{end}
	\Fn{\texttt{analyze} (block : $\texttt{INSTR}^*$) : $\mathcal{S}^\mathcal{L}$}
	{
 	$state$ = Bl \Comment*[r]{Initialize the state with first block}
 	
 	\ForEach{inst $\in$ block}{
 	
 		$state' = \texttt{analyze\_instr}(inst)$                     \Comment*[l]{Calc. changes}

		$state = \texttt{merge\_h}(state, state')$                     \Comment*[r]{Merge changes}
	}

	$states$ = $\emptyset$                                                 \Comment*[r]{Set of succ. states}
	
	$blocks$ = $\texttt{successor(block)}$                                   \Comment*[r]{Get succ. blocks}
	
	\ForEach{block' $\in$ blocks} {
	
 		$state'$ = $\texttt{ analyze}(block')$   \Comment*[r]{Analyze succ. block}
 		
		$states$ = $states$ $\cup$ \{$state'$\}  \Comment*[r]{Add succ. states}
	}

	$state'$ = $\texttt{merge\_h }(states)$   \Comment*[r]{Merge succ. states}

	\Return $merge\_v(state, state')$  \Comment*[r]{Merge to final state}

	}
\caption{Basic block liveness analysis.}
\label{alg:liveness}
\end{algorithm}
% \vspace{-.5cm}
Algorithm~\ref{alg:liveness} is based on the liveness analysis algorithm presented in~\cite{khedker2009data}, which basically is a depth-first traversal of basic blocks. 
For customization, we rely on the implementation of several functions which we will present next. $\mathcal{S}^\mathcal{L}$ is the set of possible register states which depends on the specific 
implementations of the following operations.

$\bullet$ $\texttt{merge\_v} : \mathcal{S}^\mathcal{L} \times \mathcal{S}^\mathcal{L} \mapsto \mathcal{S}^\mathcal{L}$, (merge vertically block states) describes how to merge the current state with the following state change.

$\bullet$ $\texttt{merge\_h} : \mathcal{P}(\mathcal{S}^\mathcal{L}) \mapsto \mathcal{S}^\mathcal{L}$, (merge horizontally block states) describes how to merge a set of states resulting from several paths.

$\bullet$ $\texttt{analyze\_instr} : \texttt{INSTR} \mapsto \mathcal{S}^\mathcal{L}$, (analyze instruction) calculates the state change that occurs due to the given instruction.

$\bullet$ $\texttt{succ} : \texttt{INSTR}^* \mapsto \mathcal{P}(\texttt{INSTR}^*)$, (successor of a basic block) calculates the successors of the given block.

In our specific case, the function \texttt{analyze\_instr} needs to also to handle non-jump and non-fall-through successors, as these are not handled by DynInst. 
Essentially, there are three relevant cases.
First, if the current instruction is an indirect call or a direct call and the analysis algorithm is set not to follow calls, 
then our analysis will return a state where all registers are considered to be written before read. Second, if the current instruction is
a direct call and the analysis algorithm is set not to follow calls, 
then we start an analysis of the target function an return its result.
If the instruction is a constant write (\textit{e.g.,} xor of two registers), 
then we remove the read portion before we return the decoded state.
Finally, in any other case, we simply return the decoded state.
This leaves us with the two undefined merge functions and the undefined liveness state $\mathcal{S}^\mathcal{L}$. 

\subsubsection{Required Parameter Wideness}
\label{subsection:requiredparamwideness}
For our type policy, we need a finer representation of the state of one register as follows.
(1) $W$ represents write before read access,
(2) $r8, r16, r32, r64$ represents read before write access with 8-, 16-, 32-, 64-bit width, and
(3) $C$ represents the absence of access.
%\begin{enumerate}
%\item Was the register written to before its value could be read ? \\ We represent this with the state $W$.
%\item How much was read from the register before its value was overwritten? \\ We represent this with the states $\{ r8, r16, r32, r64 \}$ 
%using $R$ as a placeholder for arbitrary reads.
%\item Did neither read nor write access occur for the register ? \\ We represent this with the state $C$.
%\end{enumerate}
This gives us the following $S^\mathcal{L} = \{ C, r8, r16, r32, r64, W \}$ register state which translates to the register super state 
$\mathcal{S}^\mathcal{L} = (S^\mathcal{L})^{16}$.
%Now, we assume that unless the instructions we are looking at does discard the value it is reading (\texttt{xor rax rax} would be such 
%an instruction that we call const\_write) that reading does precede the writing withing one instruction.
As there could be more than one read of a register before it is written, we might be interested in more than just the first occurrence of a write or read on a path. 
To permit this, we allow our merge operations to also return the value $RW$, which represents the existence of both read and write access and then can use $W$ with the functionality of an end marker.
% We arrive therefore at three possible vertical merge functions:
%\begin{itemize}
%	\item The same vertical merge operator as used in the \emph{count} policy, which only gives us the first non $C$ state ($merge\_v^{r}$).
%	\item A vertical merge operator that conceptually intersects all read accesses along a path until the first write occurs ($merge\_v^{i}$).
%	\item A vertical merge operator that conceptually calculates the union of all read accesses along a path until the first write occurs ($merge\_v^{u}$).
%\end{itemize}
Therefore, our vertical merge operator conceptually intersects all read accesses along a path until the first write 
occurs $merge\_v^{i}$. In any other case, it behaves like the previously mentioned vertical merge function.
%Our horizontal merge function is a simple pairwise combination of the given set of states:
%\begin{align}
%merge\_h(\{s\}) &= s\\
%merge\_h(\{s\} \cup s') &= s \circ merge\_h(s')
%\end{align}
Our horizontal merge $merge\_h$ function is a pairwise combination of the given set of states, which are then combined with an union-like operator 
with $W$ preceding $WR$ and $WR$ preceding $R$ and $R$ preceding $C$. Unless one side is $W$, read accesses are combined in such a way that always the higher one is selected.

%The results of our experiments with the implementation of calltarget classification gave presented us with essentially one possible candidate
%that we can base our horizontal merge function on, namely the union operator with an analysis function that follows into direct calls. The 
%basic schema of the merging is depicted in \ref{tbl:TYPECTunion} and it essentially behaves as if it was the union operator (when both states
%are set, the higher one is chosen). However, we have to account for W being used as an end marker, which is why we added mapping for RW, 
%which is essentially that. 
%
%\begin{table}[h]
%\centering
%% \resizebox{\columnwidth}{!}{%
%\begin{tabular}{c?c|c|c|c}
%$\bigcup^{\mathcal{L}}$  & C & R & W & RW\\
%\Xhline{1pt}
%C & C & R & W & RW\\
%\hline
%R & R & $\text{R}^{\cup}$ & W & $\text{R}^{\cup}$W\\
%\hline
%W & W & W & W & W\\Liveness analysis of a 
%\hline
%RW & RW & $\text{R}^{\cup}$W & W & RW\\
%\end{tabular}
%% }
%\caption{The union mapping operator for liveness in the \emph{type} policy.}
%\label{tbl:TYPECTunion}
%\end{table}

\subsubsection{Required Parameter Count} 
For our {count} policy, we need a coarse representation of the state of one register, thus we use the following representation.
(1) $W$ represents write before read access, 
(2) $R$ represents read before write access, and 
(3) $C$ represents the absence of access. 
Further, this gives us the $S^\mathcal{L} = \{ C, R, W \}$ as register state, which translates to the register super state $\mathcal{S}^\mathcal{L} = (S^\mathcal{L})^{16}$. 
We implement \texttt{merge\_v} in such a way that a state within a superstate is only updated if the corresponding register was not accessed, as represented by $C$. 
Our reasoning is that the first access is the relevant one in order to determine read before write.
Our horizontal \textit{merge($merge\_h$)} function is a simple pairwise combination of the given set of states, which are then combined with an union like operator with $W$ preceding $R$ and $R$ preceding $C$.
The index of highest parameter register based on the used call convention that has the state R considered to be the number of parameters a function at least requires to be prepared by a callsite.

%
%This leaves us with the two merge functions remaining undefined and we will leave the implementation of these and the interpretation of the  liveness state $\mathcal{S}^\mathcal{L}$ into parameters up to the following subsections.
%
%\textbf{Required Parameter Count.}
%\label{subsection:requiredparamcount}
%To implement the \emph{count} policy, we only need a coarse representation of the state of one register, thus we use the same representation as TypeArmor:
%\textit{1)} $W$ represents write before read access,
%\textit{2)} $R$ represents read before write access, and
%\textit{3)} $C$ represents the absence of access.
%
%This gives us the following register state $S^\mathcal{L} = \{ C, R, W \}$ which translates to the register super state $\mathcal{S}^\mathcal{L} = (S^\mathcal{L})^{16}$.
%We are only interested in the first occurrence of a $R$ or $W$ within one path, as following reads or writes do not give us more information. 
%Therefore, our vertical merge function ($merge\_v$) behaves in the following way that only when the first given state is $C$, 
%then the return value is represented by the second state. In all other cases it will return the first state.
%
%\begin{align}
%merge\_v^{r} (cur, delta) &= \left\{
%  \begin{array}{lr}
%     delta & cur = C \\
%     cur & otherwise
%  \end{array}
%\right. \\Liveness analysis of a 
%merge\_v (cur, delta) &= (s'_0, ... s'_15) \text { with } s'_j = merge\_v^{r}(cur_j, delta_j)
%\end{align}

%Our horizontal merge($merge\_h$) function is a simple pairwise combination of the given set of states, 
%which are then combined with an union like operator with $W$ preceding $R$ preceding $C$.
%\begin{align}
%merge\_h(\{s\}) &= s\\
%merge\_h(\{s\} \cup s') &= s \circ merge\_h(s')
%\end{align}

%
%We have three viable possibilities for our combination operator $\circ$, depicted in Table \ref{fig:COUNTlivenessmapping}, which all give priority to $W$:
%\begin{itemize}
%\item [$\bigsqcap^{\mathcal{L}}$] is what we call the destructive combination operator, as it returns W on any mismatch.
%\item [$\bigcap^{\mathcal{L}}$] is what we call the intersection operator, as it returns C, when combining C and R, similar to an intersection.
%\item [$\bigcup^{\mathcal{L}}$] is what we call the union operator, as it returns R, when combining C and R similar to an union.
%\end{itemize}


\newcolumntype{?}{!{\vrule width 1pt}}
%
%\begin{table}
%
%% \centering
%\resizebox{\columnwidth}{!}{%Liveness analysis of a 
%\begin{tabular}{c?c|c|c}
%$\bigsqcap^{\mathcal{L}}$ & C & R & W\\
%\Xhline{1pt}
%C & C & W & W\\
%\hline
%R & W & R & W\\
%\hline
%W & W & W & W
%\end{tabular}
%\begin{tabular}{c?c|c|c}
%$\bigcap^{\mathcal{L}}$  & C & R & W\\(merge vertically block states)
%\Xhline{1pt}
%C & C & C & W\\
%\hline
%R & C & R{todo} & W\\
%\hline
%W & W & W & W
%\end{tabular}
%\begin{tabular}{c?c|c|c}
%$\bigcup^{\mathcal{L}}$  & C & R & W\\
%\Xhline{1pt}
%C & C & R & W\\
%\hline
%R & R & R & W\\
%\hline
%W & W & W & W
%\end{tabular}}
%\caption{Different mappings for combining two liveness state values in horizontal matching for the \emph{count} policy.}
%
%\label{fig:COUNTlivenessmapping}
%\end{table}
%The index of highest parameter register based on the used call convention that has the state R is considered to be the number of parameters a function at least requires to be prepared by a callsite.

\subsubsection{Void/Non-Void Calltarget}
In order to determine if a calltarget is a void or non-void return function
\textsc{TypeShield} traverses backwards the basic blocks from the return instruction of the function an looks for the \texttt{RAX} register.
In case  there is a write operation on the \texttt{RAX} register then \textsc{TypeShield}
infers that the function return is non-void and thus provides a pointer value back.

% \subsubsection{Encountered Analysis Issues}
% \paragraph{Variadic Functions}
% \label{subsection:variadicfunctions}
% Variadic functions are a special type of C/C++ functions that have a basic set of parameters, 
% which they always require and a variadic set of parameters, which 
% may vary. A prominent example of this would be the $printf$ function, which is used 
% to output text to \texttt{stdout}.
% 
% % \begin{figure}[thp] % the figure provides the caption
% % \centering          % which should be centered
% % \begin{tabular}{c}  % the tabular makes the listing as small as possible and centers it
% % \footnotesize
% % \begin{lstlisting}
% % 00000000004222f0 <make_cmd>:
% %  4222f0:push   %r15
% %  4222f2:push   %r14
% %  4222f4:push   %rbx
% %  4222f5:sub    $0xd0,%rsp
% %  4222fc:mov    %esi,%r15d
% %  4222ff:mov    %rdi,%\begin{figure}[!h]
% %  422302:test   %al,%al
% %  422304:je     42233d <make_cmd+0x4d>
% %  422306:movaps %xmm0,0x50(%rsp)
% %  42230b:movaps %xmm1,0x60(%rsp)
% %  422310:movaps %xmm2,0x70(%rsp)
% %  422315:movaps %xmm3,0x80(%rsp)
% %  42231d:movaps %xmm4,0x90(%rsp)
% %  422325:movaps %xmm5,0xa0(%rsp)
% %  42232d:movaps %xmm6,0xb0(%rsp)
% %  422335:movaps %xmm7,0xc0(%rsp)
% %  42233d:mov    %r9,0x48(%rsp)
% %  422342:mov    %r8,0x40(%rsp)
% %  422347:mov    %rcx,0x38(%rsp)
% %  42234c:mov    %rdx,0x30(%rsp)
% %  422351:mov    $0x50,%esi
% %  422356:mov    %r14,%rdi
% %  422359:callq  409430 <pcalloc>
% % \end{lstlisting}
% % \end{tabular}
% % \caption{ASM code of the \texttt{make\_cmd} function with optimize level O2, which has a variadic parameter list.}
% % \label{fig:asmvariadic}
% % \end{figure}
% 
% \newsavebox{\firstlistingA}
% \begin{lrbox}{\firstlistingA}
% \begin{minipage}[c]{0.52\linewidth}
% \begin{minted}[
% % frame=lines,
% framesep=2mm,
% linenos,
% frame=none,
% firstnumber=1,
% linenos,
% highlightlines={10-17},
% highlightcolor={lightgray},
% numbersep=2pt,
% %gobble=2,
% %frame=lines,
% framesep=2mm,
% % fontsize=\tiny       
% fontsize=\scriptsize      
% % baselinestretch=1.2,
% % bgcolor=LightGray,
% % fontsize=\footnotesize,
% %use in terminal: pygmentize -L lexers to see all code highlighting options
% ]{asm}
% 00000000004222f0 <make_cmd>:
%  4222f0:push   %r15
%  4222f2:push   %r14
%  4222f4:push   %rbx
%  4222f5:sub    $0xd0,%rsp
%  4222fc:mov    %esi,%r15d
%  4222ff:mov    %rdi,%\begin{figure}[!h]
%  422302:test   %al,%al
%  422304:je     42233d <make_cmd+0x4d>
%  422306:movaps %xmm0,0x50(%rsp)
%  42230b:movaps %xmm1,0x60(%rsp)
%  422310:movaps %xmm2,0x70(%rsp)
%  422315:movaps %xmm3,0x80(%rsp)
%  42231d:movaps %xmm4,0x90(%rsp)
%  422325:movaps %xmm5,0xa0(%rsp)
%  42232d:movaps %xmm6,0xb0(%rsp)
%  422335:movaps %xmm7,0xc0(%rsp)
%  42233d:mov    %r9,0x48(%rsp)
%  422342:mov    %r8,0x40(%rsp)
%  422347:mov    %rcx,0x38(%rsp)
%  42234c:mov    %rdx,0x30(%rsp)
%  422351:mov    $0x50,%esi
%  422356:mov    %r14,%rdi
%  422359:callq  409430 <pcalloc>
% \end{minted}
% \end{minipage}
% \end{lrbox}
% \begin{figure}[H]
% \centering
% % \subfloat[]
% {\usebox{\firstlistingA}} 
% \caption{Assembly code of the \texttt{make\_cmd} function which was compiled with Clang -O2 flag, and has a variadic parameter list which is shaded gray above.}
% \label{fig:asmvariadic}
% \end{figure}
% 
% Figure~\ref{fig:asmvariadic} depicts the binary code of a variadic function which allows an easier processing of parameters
% due to the fact that all potential variadic parameters are moved into a contiguous block of memory.
% Our analysis interprets this functions as a read access on all parameters and thus, we arrive at a potentially problematic overestimation. 
% In our solution we opted to find these spurious reads and ignore them for now. A compiler will implement this type of operation very 
% similar for all cases, thus we can achieve our desired outcome using the following steps:
% (1) we search for (what we call) the xmm-passthrough block, which entirely consists of moving values of registers \texttt{xmm0} to \texttt{xmm7} into
% contiguous memory, (in our case basic block [\texttt{0x422306}, \texttt{0x42233d} [ );
% (2) we look at the predecessor of the xmm-passthrough block, which we call the entry block, next we %[\texttt{0x4222f0},\texttt{0x4222f2} [ )
% check if the successors of the entry block consist of the xmm-passthrough block and the successor of the xmm-passthrough block (we call the param-passthrough block), and
% (in our case basic block [\texttt{0x42233d} Liveness analysis of a, \texttt{0x42235e} [ );
% (3) we look at the param-passthrough block and set all instructions that move the value of a parameter register into memory to be ignored. 
% (in our case the instructions \texttt{0x42233d}, \texttt{0x422342}, \texttt{0x422347} and \texttt{0x42234c})
% 
% \paragraph{Ignoring Reads} When one instruction writes and reads a register at the same time, we give the read access precedence, however, there 
% are exceptions (also mentioned in TypeArmor). However, we expand slightly on that as follows:
% (1) \texttt{xor \%rax, \%rax} is the first scenario, as it will always result in \texttt{\%rax} holding the value 0,
% (2) \texttt{sub \%rax, \%rax} is the second scenario, as it results in \texttt{\%rax} also holding the value 0, and
% (3) \texttt{sbb \%rax, \%rax} is also relevant, however, it will not result in a constant value and based on the current state might either result in \texttt{\%rax} containing 0 or 1.

\subsection{Callsite Analysis}
\label{section:callsiteanalysis}
Our callsite analysis classifies callsites according to the parameters they provide. Overestimations are allowed, however,
underestimations are not permitted. For this purpose we employ a customizable modified reaching definition algorithm, 
which we will show first. 
% Furthermore, we will highlight some corner cases we encountered.

\subsubsection{Reaching Definitions}
\label{subsection:reachindefinitionstheory}
An assignment to a variable is a reaching definition after the execution of a set of instruction if that variable still exists in at least one possible execution path. 
If applied to a callsite, this calculates the values that are provided by this callsite to the function it then invokes. 

\begin{algorithm}[!ht]
%         \scriptsize
        \footnotesize
	\SetAlgoLined
	\SetKwInOut{Input}{Input}
        \SetKwInOut{Output}{Output}
        \Input{The basic block to be analyzed - $block$ : $\texttt{INSTR}^*$}
        \Output{The reaching definition state - $\mathcal{S}^\mathcal{R}$}
        \BlankLine
	\SetKwProg{Fn}{Function}{ is}{end}
	\Fn{\texttt{analyze}($block$ : $\texttt{INSTR}^*$) : $\mathcal{S}^\mathcal{R}$}
	{
 	$state$ = Bl                                   \Comment*[r]{Initialize the state with first block}
 	
 	\ForEach{inst $\in$ reversed(block)}{
 	
 		$state' = analyze\_instr(inst)$        \Comment*[r]{Calculate changes}
 		
		$state = merge\_v(state, state')$      \Comment*[r]{Merge changes}
	}

	$states$ = $\emptyset$                         \Comment*[r]{Set of predecessor states}
	
	$blocks$ = $pred(block)$                         \Comment*[r]{Get predecessors blocks}
	
	\ForEach{block' $\in$ blocks} {
	
 		$state' = analyze(block')$             \Comment*[r]{Analyze pred. block}
 		
		$states$ = $states$ $\cup$ \{$state'$\}  \Comment*[r]{Add pred. states}
	}

	$state'$ = $merge\_h (states)$                   \Comment*[r]{Merge predecessors states}

	\Return $merge\_v(state, state')$              \Comment*[r]{Merge to final state}

	}
\caption{Basic block reaching definition analysis.}
\label{alg:reaching}
\end{algorithm}
\vspace{-.5cm}
Algorithm~\ref{alg:reaching} is based on the reaching definition analysis presented in~\cite{khedker2009data}, 
which basically is a reverse depth-first traversal of basic blocks of a program. For customization, we rely on 
the implementation of several functions. $\mathcal{S}^\mathcal{R}$ is the set of possible register states 
which depends on the specific reaching definition implementation of the following operations.

$\bullet$ $\texttt{merge\_v} : \mathcal{S}^\mathcal{R} \times \mathcal{S}^\mathcal{R} \mapsto \mathcal{S}^\mathcal{R}$, (merge vertically block states) describes how to merge the current state with the following state change.

$\bullet$ $\texttt{merge\_h} : \mathcal{P}(\mathcal{S}^\mathcal{R}) \mapsto \mathcal{S}^\mathcal{R}$, (merge horizontally block states) describes how to merge a set of states resulting from several paths.

$\bullet$ $\texttt{analyze\_instr} : \texttt{INSTR} \mapsto \mathcal{S}^\mathcal{R}$, (analyze instruction) calculates the state change that occurs due to the given instruction.

$\bullet$ $\texttt{pred} : \texttt{INSTR}^* \mapsto \mathcal{P}(\texttt{INSTR}^*)$, (predecessor of a basic block) calculates the predecessors of the given block.

In our specific case, the function \texttt{analyze\_instr} does not need to handle normal predecessors, as DynInst will resolve those for us. 
However there are several instructions that have to be handled as depicted in the following situations. 
(1) If the current instruction is an indirect call or a direct call and the analysis algorithm is set not to follow calls, then return a state where all registers are considered trashed. 
(2) If the instruction is a direct call and the analysis algorithm is set to follow calls, then we start an analysis of the target function. 
(3) In all other cases we simply return the decoded state. This leaves us with the two merge functions and the undefined reaching definitions state $\mathcal{S}^\mathcal{R}$. 

%The book~\cite{khedker2009data} defines reaching definition analysis on blocks in the following manner:
%\begin{subequations}
%\label{eq:reachingbasedef}
%\begin{align}
%In_n &:= \left\{
%  \begin{array}{lr}
%    Bl & \text{n is start block}\\
%    \underset{p \in pred(n)}{\bigcup} Out_p & \text{otherwise}
%  \end{array}
%\right. \label{eq:reachingbasedefInt}\\
%Out_n &:= (In_n - Kill_n) \cup Gen_n \label{eq:reachingbasedefOut}
%\end{align}
%\end{subequations}
%$Bl$ is the default state at the start of a path of execution and in our case reaching that state would mean that we do not 
%know whether a value has been provided for the variable and therefore we assume that one has been provided, reaching an 
%overestimation. The set $Kill_n$ describes all definitions that are removed within this block, meaning that the value of 
%a variable has been overwritten. The set $Gen_n$ describes the new definitions that have been provided by the block $n$, 
%meaning that the value of a variable has been assigned. Considering this, we can assume that $Gen_n \subseteq Kill_n$, 
%as we can always create new definitions, but not simply remove definitions without assigning a new value to the variable.

%
%
%However, we cannot use reaching definition analysis as is, because the analysis is again based on potentially unbound 
%variable sets, while we are restricted to a finite number of registers and states. This time however, the analysis provides us with an overestimation, 
%we however, want to get a result as close as possible so we again want to customize merge functions. Furthermore, we have to define how 
%to interpret the changes occuring withing one block based on the the change caused by its instructions. Considering this, w, we 
%arrive at algorithm \ref{alg:reaching} to compute the liveness state at the start of a basic block.

Previous work~\cite{khedker2009data} provides a reaching definition analysis on blocks, which we use to arrive at the algorithm depicted in 
Algorithm~\ref{alg:reaching} to compute the liveness state at the start of a basic block. We apply the reaching analysis at each indirect 
callsite directly before each call instruction.

This algorithm relies on various functions that can be used to configure its behavior. We define the 
function $merge\_v$, which describes how to compound the state change of the current instruction and the current state, 
the function $merge\_h$, which describes how to merge the states of several paths, the instruction analysis function
$analyze\_instr$. Note, that the function $pred$, which retrieves all possible predecessors of a block 
is provided by the DynInst instrumentation framework.

% \begin{subequations}
% \label{eq:livenesscustom}
% \begin{align}
% merge\_v &: \mathcal{S}^\mathcal{R} \times \mathcal{S}^\mathcal{R} \mapsto \mathcal{S}^\mathcal{L}\\
% merge\_h &: \mathcal{P}(\mathcal{S}^\mathcal{R}) \mapsto \mathcal{S}^\mathcal{R}\\
% analyze\_instr &: \mathcal {I} \mapsto \mathcal{S}^\mathcal{R} \\
% pred &: \mathcal{I} \mapsto \mathcal{P}(\mathcal{I})
% \end{align}
% \end{subequations}

The $analyze\_instr$ function calculates the effect of an instruction and is the core of the analyze function (see Algorithm~\ref{alg:reaching}). It will also 
handle non-jump and non-fall-through successors, as these are not handled by DynInst in our case. We essentially have three cases that we handle:
(1) If the instruction is an indirect call or a direct call but we chose not to follow calls, then return a state where all trashed are considered written,
(2) If the instruction is a direct call and we chose to follow calls, then we spawn a new analysis and return its result, and
%\item if the instruction is a constant write (\textit{e.g.,} xor of two registers) then we remove the read portion before we return the decoded state
(3) In all other cases, we simply return the decoded state.

This leaves us with the two merge functions remaining undefined and we will leave the implementation of these and the interpretation of the 
liveness state $\mathcal{S}^\mathcal{L}$ into parameters up to the following subsections.

%The book~\cite{khedker2009data} defines reaching definition analysis on blocks, which we use to arrive at algorithm depicted in Algorithm~\ref{alg:reaching} to compute 
%the liveness state at the start of a basic block. We apply the reaching analysis at each indirect callsite directly before each call instruction.

%This algorithm relies on various functions that can be used to configure its behavior. We need to define the 
%function $merge\_v$, which describes how to compound the state change of the current instruction and the current state, 
%the function $merge\_h$, which describes how to merge the states of several paths, the instruction analysis function
%$analyze\_instr$. The function $pred$, which retrieves all possible predecessors of a block won't be implemented by us, 
%because we rely on the DynInst instrumentation framework to achieve the following.
%\vspace{-.3cm}
%\begin{subequations}
%\label{eq:livenesscustom}
%\begin{align}
%merge\_v &: \mathcal{S}^\mathcal{R} \times \mathcal{S}^\mathcal{R} \mapsto \mathcal{S}^\mathcal{L}\\
%merge\_h &: \mathcal{P}(\mathcal{S}^\mathcal{R}) \mapsto \mathcal{S}^\mathcal{R}\\
%analyze\_instr &: \mathcal {I} \mapsto \mathcal{S}^\mathcal{R} \\
%pred &: \mathcal{I} \mapsto \mathcal{P}(\mathcal{I})
%\end{align}
%\end{subequations}
%\vspace{-.8cm}

%As the $analyze\_instr$ function calculates the effect of an instruction and is the core of the analyze function. It will also 
%handle non jump and non fall-through successors, as these are not handled by DynInst in our case. We essentially have three cases that we handle:
%\textit{1)} if the instruction is an indirect call or a direct call but we chose not to follow calls, then return a state where all trashed 
%are considered written,
%\textit{2)}  if the instruction is a direct call and we chose to follow calls, then we spawn a new analysis and return its result, and
%\item if the instruction is a constant write (\textit{e.g.,} xor of two registers) then we remove the read portion before we return the decoded state
%\textit{3)} in all other cases we simply return the decoded state.

%This leaves us with the two merge functions remaining undefined and we will leave the implementation of these and the interpretation of 
%the liveness state $\mathcal{S}^\mathcal{L}$ into parameters up to the following subsections.

%We have yet to define the functions $merge\_v$, which describes how to compound a function and the outgoing state, the function $merge\_h$, which describes how to
%merge the states of several paths and the function $pred$, which essentially gives us the predecessors of the current instruction. To prevent cycles we keep 
%track of the instructions visited within the current path and omit any instruction on the current path from the result of $pred$. These functions, the 
%reaching state $\mathcal{S}^\mathcal{R}$  and its interpretation into parameters will be defined in the following subsections.
%
%
%\subsection{Backward Graph Traversal}
%\label{subsection:backwardgraphtraversal}

\subsubsection{Provided Parameter Width}
\label{subsection:providedparamwideness}
In order to implement our {type} policy, we use a finer representation of the states of one register, thus we consider:
(1) $T$ represents a trashed register,
(2) $s8, s16, s32, s64 S$ represents a set register with  8-, 16-, 32-, 64-bit width, and
(3) $U$ represents an untouched register.
%\begin{itemize}
%\item Was the register value trashed ? \\ We represent this with the state T.
%\item Was the register written to and how much ? \\ We represent this with the states $\{ s64, s32, s16, s8 \}$ using S as a placeholder for arbitrary writes.
%\item Was the register neither trashed nor written to ? \\ We represent this with the state U.
%\end{itemize}
This gives us the following $S^\mathcal{L} = \{ T, s64, s32, s16, s8, U \}$ register state which translates to the register 
super state $\mathcal{S}^\mathcal{R} = (S^\mathcal{R})^{16}$.

However, we are only interested in the first occurrence of a state that is not $U$ in a path, as following reads or writes do not give us more information. Therefore, we can use 
the same vertical merge function as for the \emph{count} policy, which is essentially a pass-through until the first non $U$ state.

Our horizontal merge $merge\_h$ function is a simple pairwise combination of the given set of states, which are then combined with an union like operator with $T$ 
preceding $S$ and $S$ preceding $U$. Note, that when both states are set, we pick the higher one.

%
%Our horizontal merge function is again a simple pairwise combination of the given set of states:
%\begin{align}
%merge\_h(\{s\}) &= s\\
%merge\_h(\{s\} \cup s') &= s \circ merge\_h(s')
%\end{align}
%
%However, we have different possibilities regarding the merge operator. Experiments with our implementations for callsite 
%classification in the \emph{count} policy have given us the following results:
%\begin{itemize}
%\item The best candidate to minimize the problematic matches is the union operator without following direct calls.
%\item The best candidate to maximize precision is the intersection operator with following direct calls.
%\end{itemize}
%
%We therefore arrive at three viable possibilities for our combination operator $\circ$, depicted in table \ref{fig:TYPEreachingmapping}, 
%which all (except one) give priority to $T$:
%\begin{itemize}
%\item [$\bigcap^{\mathcal{R}}$] is what we call the intersection operator, as it returns U, when combining U and S, similar to an 
%intersection furthermore we also calculate the intersection of states when both states are set 
%(the lower of the two is returned).
%\item [$\bigsqcap^{\mathcal{R}}$] is what we call the half intersection operator, as it returns U, when combining U and S, 
%similar to an intersection but we calculate the union of states when both states are set (the higher of the two is returned).
%\item [$\bigcup^{\mathcal{R}}$] is what we call the union operator, as it returns S, when combining U and S similar to an union
%furthermore we calculate the union of states when both states are set (the higher of the two is returned).
%\end{itemize}
%
%\begin{table}
%
%% \centering
%\resizebox{\columnwidth}{!}{%
%\begin{tabular}{c?c|c|c}
%$\bigcap^{\mathcal{R}}$  & U & S & T\\
%\Xhline{1pt}
%U & U & U & T\\
%\hline
%S & U & $\text{S}^{\cap{}{}}$ & T\\
%\hline
%T & T & T & T
%\end{tabular}
%\begin{tabular}{c?c|c|c}
%$\bigsqcap^{\mathcal{R}}$  & U & S & T\\
%\Xhline{1pt}
%U & U & U & T\\
%\hline
%S & U & $\text{S}^{\cup{}{}}$ & T\\
%\hline
%T & T & T & T
%\end{tabular}
%\begin{tabular}{c?c|c|c}
%$\bigcup^{\mathcal{R}}$  & U & S & T\\
%\Xhline{1pt}
%U & U & S & T\\
%\hline
%S & S & $\text{S}^{\cup{}{}}$ & T\\
%\hline
%T & T & T & T
%\end{tabular}}
%
%\caption{Different mappings for combining two reaching state values in horizontal matching for the \emph{type} policy.}
%\label{fig:TYPEreachingmapping}
%\end{table}

\subsubsection{Provided Parameter Count}
\label{subsection:providedparamcount}
For implementing our {count} policy, we use a coarse representation of the state of one register, thus we use the following representation.
(1) $T$ represents a trashed register,
(2) $S$ represents a set register (written to), and
(3) $U$ represents an untouched register.
This gives us the following $S^\mathcal{L} = \{ T, S, U \}$  register state which translates to the register super state $\mathcal{S}^\mathcal{R} = (S^\mathcal{R})^{16}$.

We are only interested in the first occurrence of a $S$ or $T$ within one path, as following reads or writes do not give us more information. Therefore, our vertical 
merge function $merge\_v$ behaves as follows. In case the first given state is $U$, than the return value is the second state and in all other cases it will return the first state.
%
%We are only interested in the first occurrence of a S or T within one path, as following reads or writes do not give us more information.
%Therefore, we can define our vertical merge function in the following way:
%\begin{align}
%merge\_v^{r} (cur, delta) &= \left\{
%  \begin{array}{lr}
%     delta & cur = U \\
%     cur & otherwise
%  \end{array}
%\right. \\
%merge\_v (cur, delta) &= (s'_0, ... s'_15) \text { with } s'_j = merge\_v^{r}(cur_j, delta_j)
%\end{align}


Our horizontal merge $merge\_h$ function is a pairwise combination of the given set of states, which are then combined with an union like operator with $T$ preceding $S$ and $S$ preceding $U$.
%
%Our horizontal merge function is a simple pairwise combination of the given set of states:
%\begin{align}
%merge\_h(\{s\}) &= s\\
%merge\_h(\{s\} \cup s') &= s \circ merge\_h(s')
%\end{align}
%
%We have four viable possibilities for our combination operator $\circ$, depicted in table \ref{fig:COUNTreachingmapping}, which all (except one) give priority to $T$:
%\begin{itemize}
%\item [$\bigsqcap^{\mathcal{R}}$] is what we call the destructive combination operator, as it returns T on any mismatch.
%\item [$\bigcap^{\mathcal{R}}$] is what we call the intersection operator, as it returns U, when combining U and S, similar to an intersection.
%\item [$\bigcup^{\mathcal{R}}$] is what we call the union operator, as it returns S, when combining U and S similar to an union.
%\item [$\bigsqcup^{\mathcal{R}}$] is what we call the true union operator, as it gives S precedence over everything and returns T or 
%U only when both sides are T or U being more inclusive than an union.
%\end{itemize}

\newcolumntype{?}{!{\vrule width 1pt}}
%
%\begin{table}
%% \centering
%{
%\resizebox{\columnwidth}{!}{%
%\begin{tabular}{c?c|c|c}
%$\bigsqcap^{\mathcal{R}}$ & U & S & T\\
%\Xhline{1pt}
%U & U & T & T\\
%\hline
%S & T & S & T\\
%\hline
%T & T & T & T
%\end{tabular}
%\begin{tabular}{c?c|c|c}
%$\bigcap^{\mathcal{R}}$  & U & S & T\\
%\Xhline{1pt}
%U & U & U & T\\
%\hline
%S & U & S & T\\
%\hline
%T & T & T & T
%\end{tabular}
%\begin{tabular}{c?c|c|c}
%$\bigcup^{\mathcal{R}}$  & U & S & T\\
%\Xhline{1pt}
%U & U & S & T\\
%\hline
%S & S & S & T\\
%\hline
%T & T & T & T
%\end{tabular}
%\begin{tabular}{c?c|c|c}
%$\bigsqcup^{\mathcal{R}}$  & U & S & T\\
%\Xhline{1pt}
%U & U & S & T\\
%\hline
%S & S & S & S\\
%\hline
%T & T & S & T
%\end{tabular}}
%}
%
%\caption{Different mappings for combining two reaching state values in horizontal matching for the \emph{count} policy.}
%
%\label{fig:COUNTreachingmapping}
%\end{table}

The index of the highest parameter register based on the used call convention that has the state $S$ is considered to be the number of parameters a callsite prepares at most.

\subsubsection{Void/Non-Void Callsite}
In order to determine if a callsite is a void or non-void return function
\textsc{TypeShield} looks at the callsite if there is an read before write on the \texttt{RAX} register. 
In case there is a read before write operation on the \texttt{RAX} register then
\textsc{TypeShield} infers that the callsite is non-void and thus expects a pointer to be provided 
when the called function returns.

% \subsubsection{Encountered Analysis Issues}
% Our experiments with this implementation highlighted two issues.
% First, parameter lists with \textit{holes} and address width underestimation.
% Second, register extension instructions can be also a cause for analysis problems. 
% Finally, to reduce analysis runtime overhead, we also restricted the maximum path depth to 10 blocks.
% 
% \paragraph{Parameter Lists with \textit{Holes}.} This refers to parameter lists that show one or more \texttt{void} parameters between start to the last actual parameter. 
% These are not existent in actual code, but our analysis has the possibility of generating them through the merge operations. An example would be the following: 
% A parameter list of $(64, 0, 64, 0, 0, 0)$ is concluded, although the actual parameter list might be $(64, 32, 64, 0, 0, 0)$. While the trailing 0s are 
% what we expect, the 0 at the second parameter position will cause difficulties, because it is an underestimation at the single parameter level, which we need to avoid.
% Our solution relies on scanning our reaching analysis result for these holes and replace them with the wideness $64$, causing a possible overestimation.
% 
% \paragraph{Address Width Unterestimation.} This refers to the issue that while in the callsite a constant value of 32-bit is written to a register, the calltarget uses the 
% whole 64-bit register. This can occur when pointers are passed from the callsite to the calltarget. Specifically this happens 
% when pointers to memory inside the \texttt{.bss}, \texttt{.data} or \texttt{.rodata} section of the binary are passed.
% Our solution is to enhance our instruction analysis to watch out for constant writes. In case a 32-bit constant value write is detected, we check if the
% value is an address within the \texttt{.bss}, \texttt{.data} or \texttt{.rodata} section of the binary. If this is the case, we simply return a write access of 64-bit 
% instead of 32-bit. This is not problematic, because we are looking for an overestimation of parameter wideness.
% It should be noted that the same problem can arise when a constant write causes the value 0 to be written to a 32-bit register. We use the same solution
% and set the width to 64-bit instead of 32-bit.
% %
% %\subsection{Address Taken Analysis}
% %\label{section:addresstakenanalysis}
% %As of now, we use the maximum available set of calltargets---the set of all function entry basic blocks---as input for our algorithm. 
% %To restrict the number of calltargets per callsite even further, we explored the possibility of incorporating an address taken analysis
% %into our application. We base our theory on the paper by Zhang \textit{et al.}~\cite{mingwei:sekar}, which introduced various types of taken
% %addresses. An address is considered to be taken, when it is loaded into memory or a register.
% %
% %\textbf{Address Taken Targets.}
% %Based on the notions of \cite{mingwei:sekar}, which classified taken addresses into several types of indirect control flow targets,
% %we only chose { Code Pointer Constants (CK)} and discarded the others:
% %\begin{itemize}
% %
% %\item { Code Pointer Constants (CK)} are addresses that are calculated during the compilation of the binary and point within
% %the possible range of addresses in the current module or to instruction boundaries. We are however, only interested in addresses
% %that directly point to an entry basic block of a function, as these are the only valid targets for any callsite.
% %
% %\item { Computed code pointers (CC)} are the result of simple pointer arithmetic, however, these are only used for intra-procedural jumps. 
% %We rely on DynInst to resolve those and only focus on indirect callsites, therefore these are of no interest to us.
% %
% %\item{ Exception handling addresses (EH)} are used to handle exceptions within C++ functions and are modeled as jumps within the function. 
% %These are therefore within the normal control flow that we rely on DynInst to resolve for us.
% %
% %\item{ Exported function addresses (ES)} are essentially functions that point outside of our current module (usually to dynamically 
% %linked libraries) and are implemented as jumps, which are of no concern to us, because our analysis is only concerned about the current object.
% %
% %\item { Return addresses (RA)}, which are the addresses next to a call instruction, are also of no interest to us, because we only 
% %implement forward { control flow integrity}.
% %\end{itemize}
% %
% %\textbf{Binary Analysis.}
% %Our approach of identifying taken addresses consists of two steps: First, we iterate over the raw binary content of data sections. Second,
% %we iterate over all functions within the disassembled binary. We rely on DynInst to provide us with the boundaries of the sections inside
% %the binary and in case of shared libraries with the needed translation to current memory addresses:
% %
% %\begin{itemize}
% %\item We look at three different data sections of the binary, which could possibly contain taken addresses: the .data, .rodata and .dynsym
% %sections. As \cite{mingwei:sekar} proposed, we slide a four byte window over the data within those sections and look for addresses that
% %point to function entry blocks. However, we are looking at x64 binaries therefore we additionally use an eight byte window. In case of 
% %shared libraries, we need to let DynInst translate the raw address, we extracted, so we can perform the function check.
% %
% %\item We specifically look for instructions that load a constant value into a register or memory, and again check whether the address 
% %points to the entry block of a function.
% %\end{itemize}
% 
% %\section{Runtime Enforcement}
% %\label{section:runtimeenforcement}
% %
% %\subsection{Calltarget Annotation}
% %\label{subsection:patchingschema}
% %
% %\subsection{Callsite Instrumentation}
% %\label{subsection:patchingschema}

% \subsubsection{C++ exceptions}
% During our analysis we encountered no issues with respect to C++ exceptions.

\subsection{Backward-Edge Analysis}
\label{Backward Edge Analysis}
In order to protect the backward edges of our previously 
determined calltargets for each callsite we designed an
analysis which can determine possible legitimate return target addresses.

\begin{algorithm}[!ht]
%         \scriptsize
        \footnotesize
	\SetAlgoLined
	\SetKwInOut{Input}{Input}
        \SetKwInOut{Output}{Output}
        \Input{Forward edge callsite to calltargets map - $fMap$}
        \Output{Backward edge to return addresses map - $rMap$}
        \BlankLine
	\SetKwProg{Fn}{Function}{ is}{end}
	\Fn{\texttt{backwardAddressMapping}($fMap$) : $rMap$}
	{
	
 	\Comment*[l]{visit all detected callsites in the binary}
 	
 	\ForEach{callsite $\in$ $fMap$}{  
 	  
 	  \Comment*[l]{get calltargets for callsite address key}
 	  
 	  $calltargetSet$ = $getCalltargetSet(callsite, fMap)$\;
           
           \Comment*[l]{calltarget is the function start address}
           
           \Comment*[l]{visit all calltargets of a callsite}
           
          \ForEach{calltarget $\in$ $calltargetSet$}{
 	  
 	    \Comment*[l]{get the next address after the callsite}
 	    
 	    $rTarget$ = $getNextAddress(callsiteKey)$\;
 	    
 	    \Comment*[l]{find the address of function return}
 	    
 	    $rAddress$ = $getReturnOfCalltarget(calltarget)$\; 
 	    
 	    \Comment*[l]{rAddress is map key; rTarget is value}
 	    
 	    $rMap$ = $rMap$ $\cup$ $rMap \ add \ (rAddress, rTarget)$\;
 	   
 	  }
 	 }
        
        \Comment*[l]{return the backward-edgeaddresses mappings}
        
	\Return $rMap$\;                         

	}
\caption{Calltarget return set analysis.}
\label{alg:returns}
\end{algorithm}

Algorithm~\ref{alg:returns} depicts how the forward mapping between 
callsites and calltargets is used to determine the backward address set 
for each return address contained in each address taken function. The 
$fMap$ is obtained after running the callsite and 
calltarget analysis (see \cref{section:calltargetanalysis} and \cref{section:callsiteanalysis}). 
These mapping contains for each callsite the legal calltargets where the forward-edge
indirect control flow transfer is allowed to jump to. This mapping is reflected back by construction a second
mapping between the return address of each function for which we have the start 
address and a return target address set. 

The return target address set for a function return is determined by getting the next address after each callsite address 
which is allowed to make the forward-edge control flow transfer
(\textit{i.e.,} recall the caller callee calling convention).
The $rMap$ is obtained by visiting each function return address and assigning to it the address next to the callsite
which was used in order to transfer the control flow to the function in first place. 
At the end of the analysis all callsites and all function returns have been visited and a set for each function return address of backward-edgeaddresses will be obtained.
Note that the function boundary address (\textit{i.e.,} retn) was detected by a linear basic block search from the beginning of 
the function (calltarget) until the first return instruction was encountered. We are aware that other promising approaches for recuperating 
function boundaries (\textit{e.g.,}~\cite{function:boundary}) exist, and plan to experiment with them in future work.

\subsection{Binary Instrumentation}
\subsubsection{Forward-Edge Policy Enforcement}
\label{Enforcing The Forward Edge Policy}
The result of the forward callsite and calltarget analysis is a mapping between the allowed calltargets for each callsite.
In order to enforce this mapping during runtime each callsite and calltarget contained in the previous mapping are instrumented
inside the binary program with two labels and a callsite located CFI-based checking mechanism. At each callsite the number of 
provided parameters are encoded as a series of six bits. At the calltarget the label contains six bits denoting how 
many parameters the calltarget expects. Additionally, at the callsite six bits encode which register wideness types each of the provided parameters have 
while at the calltarget another six bits are used to encode the types of the parameters expected. Further, at the callsite another bit 
is used to define if the function is expecting a \texttt{void} return type or not. All this information are written in labels before
each callsite and calltarget. During runtime before each callsite these labels are compared by performing a xor operation between 
the bits contained in the previously mentioned labels. In case the xor operation returns false
than the transfer is allowed else the program execution is terminated.

\subsubsection{Backward-Edge Policy Enforcement}
The previously determined $rMap$ in Algorithm~\ref{alg:returns} will be used to insert a check before each 
function (calltarget) return present in the $rMap$. We propose three modes of operation based on three types
of checks which can be inserted before each function return instruction. Depending on the specific needs
one of the following modes of operation can be selected. Depending on the runtime time needed to perform each backward-edge
indirect transfer we define the following three modes of operation.

\textbf{Super fast mode.} Based on the $rMap$, for each AT function return the minimum and the maximum address out of the return set for a particular 
$rAdress$ return address will be determined. Next, these two values will be used to insert a range check having as left and right boundaries these
two values. Before the return instruction of the function is executed the value of the function return is compared against these two values previously
mentioned. In case the check fails than the program will be terminated else the indirect control flow transfer will be allowed.
Note that this check has insignificant runtime overhead but on the other side it could contain not legitimate return addresses depending 
on the entropy of the $rAddress$es. In short, this means that as far as the $min$ and $max$ addresses are from each other the more leeway the attacker will have. 

\textbf{Fast mode.} Based on the $rMap$, before each AT function return a randomly generated label (\textit{i.e.,} the value 7232943 loaded trough one level of indirection) 
value will be inserted. The same label will be inserted before each legitimate (\textit{i.e.,} based on the forward-edge policy) target address (next address after a legitimate callsite) of the function return. 
In this way a function return will be allowed to jump to only the instruction which follows next to the 
address of the callsites which are allowed to call the calltarget which contains this particular function return. 
For callsites which are allowed to call the the calltarget mentioned and another calltarget than in this cases \textsc{TypeShield} will perform a search in order to detect if the callsite
has already a label attached to the next address after the callsite. In this case the label will be reused. In this situation two callsites share their labels. The solution to this is to 
use single labels for each function return address. In this case multiple labels have to be stored for each address following a legitimate callsite.
Further, addresses located after a callsite that are not allowed to call a particular calltarget will get another randomly generated label. 
In this way calltarget return labels are grouped together based on the $rMap$. This mode allows 
at least (additionally the callsites which are allowed to call more than one calltarget are added) the same number of function return sites as the 
forward-edge policy enforces for each callsite and it is runtime efficient since label checking is based on a single compare check.

\textbf{Slow mode.} Based on the $rMap$, before each AT function return a series of comparison checks are inserted in the binary. 
Before the return instruction of the function a series of comparison checks between the appropriate 
addresses stored in $rMap$ and the address where the function wants to return are performed. In case one of the check fails than the 
program will be terminated. The total number of comparison checks added is equal to the size of return address set which contains $rTarget$ values. 
Note that these types of checks are precise since only legitimate addresses are allowed but on the other side the runtime overhead is higher than in the 
case of the fast path because the number of checks is in general higher.
 



