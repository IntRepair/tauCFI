
\section{Limitations}
\label{chapter:Limitations}

% \subsection{Comparison with TypeArmor}
% \label{section:comptype}
% In order to situate our discussion of results, we have to state two limitations. First of all, we aim to compare the overall precision of our 
% implementation of the COUNT policy with the results from TypeArmor to set the groundwork for the evaluation of precision of our TYPE policy. 
% Unfortunately, we cannot compare data regarding overestimations of calltargets or underestimations of callsites, as the TypeArmor paper does 
% not provide sufficient data. Secondly, we also aim for a comparison of the reduction of calltargets per callsite, however, this comparison 
% remains somewhat crude, as we most surely do not have the same measuring environment and not have sufficient data to infer the facets of 
% the measurement environment of the TypeArmor research project. However, despite these backwards-looking limitations, we also want to note 
% that we aim to provide a sufficiently thorough description of our measurement environment to allow for detailed comparisons by researchers
% conducting follow-up research.
% 
% \subsubsection{Precision of Classification}
% TypeArmor reports a geometric mean of 83.26\% for the perfect classification of calltargets regarding parameter count in optimization level O2, 
% which compares rather well to our result of 82.24\%. Regarding the perfect classification of callsites, we report a geometric mean of 81.6\% 
% perfect classification regarding parameter count, while TypeArmor reports a geometric mean of 79.19\%. However, we also have a geometric mean 
% of about 7\% regarding underestimations in the callsite classification with an upper
% bound of 16\%, while TypeArmor reports that it does not incur underestimations in their callsites. For our type based classification, we 
% incur the cost for two error sources. First, the error from the parameter count classification, which we base our type analysis on and, second, 
% for the type analysis itself. For the perfect classification of calltargets regarding parameter types, we report a 72.25\% geometric mean of 
% perfect classification, which is 87.85\% of our precision regarding parameter
% counts. However, we report a geometric mean of 57.36\% for perfect classification of callsites, which although seemingly low, is still 69.74\% 
% of our precision regarding parameter counts.
% 
% \subsubsection{Reduction of Available Calltargets}
% The fact that our count-based precision-focused implementation achieves a reduction in the same ballpark as TypeArmor regarding our test targets, 
% lets us believe that our implementation of their classification schema is a sufficient approximation to compare against. However, we cannot safely
% compare those numbers. Not only is the information regarding their test environment rather sparse, but the only data available is the median, 
% which in our opinion discards valuable information from the actual result set. As we needed more metrics to compare \textsc{TypeShield} and
% TypeArmor regarding calltargets, we therefore implemented an approximation. Using average and sigma, we can report that our precision-focused
% type-based classification can reduce the number of calltargets, by up to 35\% more than parameter number based classification with an overall
% reduction of about 13\%.

% \textbf{TypeArmor Discrepancies.}
% \label{section:discrep}
% As we have no access to source code of TypeArmor, we implemented an approximation
% of TypeArmor. Using this approximation we found some discrepancies between the data that we collected
% and data that was presented in the TypeArmor paper.
% A minor discrepancy between our results and the results of TypeArmor is that, while they basically implemented
% what we call a destructive merge operator for the liveness analysis. However, our data suggests that this
% operator is marginally inferior to the union path merge operator, when we compared them in our implementation.
% A major concern is the classification of calltargets, while we were able to reduce the number of overestimations
% of calltargets regarding parameter counts to essentially 0, the number of underestimations of calltarget did
% stay at a geometric mean of 7\%. This error rate is rather large when compared to the reported 0\% underestimation
% of TypeArmor, however we are not entirely sure what has caused this discrepancy. A possibility is the differing
% test environments, or a bug within our implementation that we are not aware of, or simply reaching definitions
% analysis alone is not the best possible algorithm for this particular problem.

% \textbf{Improving \textsc{TypeShield}.}
% \label{section:venuesimp}
% To improve our type analysis, we see at least two possibilities. Incorporating refined data flow analysis and 
% expanding the scope to also include memory. The main point of improvement is not the precision but for now 
% more importantly the reduction of underestimations in the callsite analysis.
% 
% To refine the data flow analysis, we propose the actual tracking of data values and simple operations, as these
% can be used to better differentiate the actual wideness stored within the current register. The highest gain, 
% we see here would be the establishment of upper and lower bounds regarding values within the register, which 
% would allow for more sophisticated callsite and calltarget invariants. Essentially we would have to resort 
% to symbolic execution or some other sort of precise abstract interpretation.
% 
% Expanding the scope to also include memory, is another possible way of improving the type analysis, as it 
% would allow us to distinguish normal 32-bit or 64-bit values and pointer addresses. Although we already have a 
% limited approach of that in our reaching implementation, we still see room for improvement, as we only check
% whether a value is within one of three binary sections or 0.

% \subsection{Limitations}
First, \textsc{TypeShield} is limited by the capabilities of the DynInst instrumentation environment, where non-returning functions like exit are 
not detected reliably in some cases. As a result, we cannot test the Pure-FTP server, as it heavily relies on these functions. 
The problem is that those non-returning functions usually appear as a second branch within a function that occurs after the normal 
control flow, causing basic blocks from the following function to be attributed to the current function. This results in a malformed 
control flow graph and erroneous attribution of callsites and problematic misclassifications for both calltargets and callsites.

Second, \textsc{TypeShield} draws on variety within the binary. In particular, we rely on the fact that functions use more than only 
64-bit values or pointers within their parameter list, otherwise, \textsc{TypeShield} is equivalent to a parameter count-based implementation. 
Occurrences of such situations are quite rare, as we learned with our experiments. With a study based on source level information, we could not 
detect a difference between our \textit{type} policy and a \textit{count} policy. However, when using our tool, we were able to detect differences,
which reinforce the fact, that we do not rely on declaration of parameters, but usage of those.

Third, \textsc{TypeShield} can protect forward and backward indirect edges 
in a binary program and can complement a shadow stack~\cite{dang:asiaccs} protection technique. For this reason, we assume that \textsc{TypeShield} can run side by side with an 
ideal backward-edge protection mechanism such as a shadow stack~\cite{conti:ccs}. However, the main goal of \textsc{TypeShield} is to complement 
shadow stack based defenses which fail to account for attacks not violating the backward-edge calling conventions such as the COOP attack.

Fourth, \textsc{TypeShield} is not intended to be more precise than source code based tools such as IFCC/VTV~\cite{vtv:tice}. However, 
\textsc{TypeShield} is highly useful in situations where the source code is typically not available (\textit{e.g.,} off-the-shelf programs), 
where programs rely on many libraries, and where the recompilation of all the shared libraries is not possible. 
Further, binary based tools such as \textsc{TypeShield} can offer precise protection when source code is not available or 
recompilation is not feasible or desirable.

Finally, while a major step forward, \textsc{TypeShield} cannot thwart all possible attacks, as even solutions with access to source 
code are unable to protect against all possible attacks~\cite{carlini:bending}. In contrast, \textsc{TypeShield}, our binary-based tool, 
can stop all currently COOP attacks published to date and significantly raises the bar for an adversary when compared to
TypeArmor and similar tools. Moreover, \textsc{TypeShield} provides a strong mitigation for other types of code-reuse attacks as well.
