
%Paper abstract
% High security, high performance and high availability 
% applications are usually implemented in C/C++ for modularity, 
% performance and compatibility to name just a few reasons.
% Virtual functions, which facilitate late binding,
% are a key ingredient in facilitating runtime polymorphism
% in C++ because these allow an object to use general (its own) 
% or specific fields and functions (inherited) contained in the class hierarchy.
% Despite the alarmingly high number of virtual pointer (\textit{vptr}) corruption
% vulnerabilities, the \textit{vptr} corruption problem has not
% been thoroughly addressed by researchers, thus the high number of vulnerabilities
% based on it.
% 
% In this paper, we present \textsc{TypeShield}, a binary runtime \textit{vptr} protection tool which is
% based on program executables instrumentation at load time.
% \textsc{TypeShield} uses a novel runtime type and function parameter counter technique
% in order to overcome the limitations of available approaches and efficiently verify dynamic 
% dispatches during runtime. In particular, \textsc{TypeShield} can be automatically and easily used
% in conjunction with legacy applications or where source code is missing to harden such binaries.
% \textsc{TypeShield} achieves higher caller/caller matching (\textit{i.e.,} precision) with reasonable runtime overhead than 
% the state-of-the-art tool. We have applied \textsc{TypeShield} to
% web servers, FTP servers and SPEC CPU2006 benchmark and were able to efficiently
% and with low performance overhead protect these applications from forward indirect edge
% \textit{vptr} based corruptions.
% Our evaluation proves that our target reduction technique achieves an additional
% reduction of the possible calltargets per callsite of up to 20\% with an overall 
% reduction of about 9\% when comparing with other state-of-the-art parameter only
% count based solutions.


%jens version
Applications aiming for high performance and availability draw on several features in the C/C++ programming language. 
A key building block are virtual functions, which facilitate late binding, and thereby support runtime polymorphism. 
However, practice-driven and academic research have identified an alarmingly high number of virtual pointer corruption 
vulnerabilities which undercut security in significant ways and are still in need of a thorough solution approach.

We contribute to this research area by proposing \textsc{TypeShield}, a binary runtime virtual pointer protection tool 
which is based on instrumentation of program executables at load time. \textsc{TypeShield} applies a novel runtime 
type and function parameter counter control-flow integrity (CFI) policy in order to overcome the limitations of available approaches and to 
efficiently verify dynamic dispatches during runtime. To enhance practical applicability, \textsc{TypeShield} can 
be automatically and easily used in conjunction with legacy applications or where source code is missing to harden 
binaries.
We have applied \textsc{TypeShield} to web servers, FTP servers and the SPEC CPU2006 benchmark and were able to 
efficiently and with low performance overhead protect these applications from forward indirect edge corruptions 
based on virtual pointers. Further, in a direct comparison with the state-of-the-art tool, \textsc{TypeShield} 
achieves higher caller/callee matching (\textit{i.e.,} precision), while maintaining a more favorable 
runtime performance overhead ($\approx$ 4\%) than other state-of-the-art tools.
Focusing the evaluation on target reduction techniques, we can demonstrate that our approach achieves a notable 
additional reduction of the possible calltargets per callsite of up to 35\% associated with an overall reduction
of about 13\% and a comparable runtime performance overhead as other state-of-the-art parameter-only count-based tools.
Finally, we want to particularly emphasize that in this paper we provide for each experiment a precise description
w.r.t. setup, results, tool misses, mean, median and geomean values which clearly increases the reproducibility.