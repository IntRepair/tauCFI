% German abstract for the CAMP report document
% Included by MAIN.TEX


\clearemptydoublepage






\vspace*{2cm}
\begin{center}
{\Large \bf Zusammenfassung}
\end{center}
\vspace{1cm}

Ultraschall ist ein weit verbreitetes bildgebendes Verfahren f�r die Diagnose und Verlaufskontrolle in vielen medizinischen Disziplinen. Die neueste Generation von Ultraschallsonden ist heutzutage in der Lage direkt 3D Bilder aufzunehmen, die meisten Ultraschallsysteme nehmen jedoch immer noch nur 2D Bilder auf. F�r viele medizinische Anwendungen ist der klinische Nutzen von 2D Bilder nicht ausreichend. Es ist deshalb w�nschenswert in der Lage zu sein 3D Bilder aufzunehmen. Eine bevorzugte M�glichkeit 3D Bilder, ohne eine 3D Ultraschallsonde, aufzunehmen ist eine Sequenz von 2D Ultraschallbildern aufzunehmen, w�hrend die Ultraschallsonde bewegt wird und dabei die Position und Orientierung der Ultraschallsonde aufzunehmen. Die aufgenommen Bilder und Positionsdaten k�nnen dann genutzt werden um ein 3D Bild zu erzeugen. Daf�r muss aber die relative Lage des Ultraschallbildkoordinatensystems in Bezug zum Tracking Sensor Koordinatensystem bekannt sein, deshalb muss man eine Kalibrierung durchf�hren.

In der Literatur sind zahlreiche Methode beschrieben wie die Kalibrierung berechnet werden kann. Jedoch wurden die meisten dieser Methoden f�r die Forschung und prototypische Arbeit entwickelt und nicht f�r den Einsatz in einem industriellen Herstellungsprozess. Innerhalb dieser Arbeit haben wir eine Methode f�r die vollautomatische Kalibrierung f�r getrackte Ultraschallsonden entwickelt. Unsere Methode wurde mit dem Ziel die Kalibrierung von intravaskul�ren Ultraschallkathetern, mit integriertem magnetischen Trackingsensor, als Teil des Herstellungsprozess durchzuf�hren entwickelt. 


