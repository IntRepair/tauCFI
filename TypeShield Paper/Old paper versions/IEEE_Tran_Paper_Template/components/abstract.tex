High security, high performance and high availability 
applications are usually implemented in C/C++ for modularity, 
performance and compatibility to name just a few reasons.
Virtual functions, which facilitate late binding,
are a key ingredient in facilitating runtime polymorphism
in C++ because it allows and object to use general (its own) 
or specific functions (inherited) contained in the class hierarchy.
Despite the alarmingly high number of \textit{vptr} corruption
vulnerabilities, the \textit{vptr} corruption problem has not
been sufficiently addressed by researchers.

% However, because of the specific implementation of late binding,
% which performs no verification in order to check where an indirect call site 
% (virtual object dispatch through virtual pointers (\textit{vptrs})) is allowed to
% call inside the class hierarchy, this opens a large attack surface which
% was successfully exploited by the COOP attack.
% Since manipulation (changing or inserting new \textit{vptrs}) violates the 
% programmer initial pointer semantics and allows an attacker to
% redirect the control flow of the program as he desires, \textit{vptrs} corruption
% has serious security consequences similar to those of other 
% data-only corruption vulnerabilities.
% Despite the alarmingly high number of \textit{vptr} corruption
% vulnerabilities, the \textit{vptr} corruption problem has not
% been sufficiently addressed by the researchers.

In this paper, we present \textit{TypeShield}, a runtime \textit{vptr} corruption
detection tool. It is based on instrumentation of executables at load time
and uses a novel runtime type and function parameter counter technique
in order to overcome the limitations of current approaches and efficiently
verify dynamic dispatching during runtime.
In particular, \textit{TypeShield} can be automatically and easily used
in conjunction with legacy applications or where source code is missing.
It achieves higher caller/caller matching (i.e., precision) with reasonable runtime overhead.
We have applied \textit{TypeShield} to
web servers, FTP servers and SPEC CPU2006 benchmark and were able to efficiently
and with low performance overhead protect these applications from forward indirect edge
\textit{vptr} based corruptions.
Our evaluation shows that our target reduction schema achieves an additional
reduction of the possible calltargets per callsite of up to 
20\% with an overall reduction of about 9\% when comparing with other state-of-the-art
parameter count based approaches.
