\section{Conclusion}
\label{chapter:Conclusion}
%version 1
% In this paper, we introduced \textsc{TypeShield} a tool for binary harding of forward indirect
% calls based on function parameter type and count.
% Advanced code reuse attacks such as COOP and its extensions or Control Jujutsu manifest due to a combination of facts and problems, 
% like memory corruption or predictable binary layout and the fact that the larger our binaries get, the 
% higher the chance they contain useful gadgets for an attacker. However, due to their nature, traditional 
% CFI cannot detect them, as they do not actually replace code to modify the control flow, but change pointers
% in memory, which redirects the targets of indirect callsites, which are uncertain at the time of compilation. 
% Two of the most common targets are the pointers to virtual function tables to implemented inheritance in C++ 
% and global function pointers. The control flow exhibited by the binary while functioning normal and while 
% under attack will seem the same. Address taken analysis alread helped cutting down the number of possible 
% calltargets one could inject by a considerable amount. And typearmor improved on that by implementing 
% invariants for both callsites and calltargets based on the number of parameters. We had no access to
% their sourcecode and therefore had to rely on their paper to implement an approximation for which 
% we generated comparable results regarding precision. We improved on that solution by implementing \textsc{TypeShield}, 
% which allows for a more fine-grained classification of calltargets and indirect callsites by implementing a rather 
% simplistic register wideness based type analysis. However, as simplistic as that analysis might be, we showed that 
% except for special cases (nginx), we were able to improve upon a parameter count based implementation by reducing 
% the average target count by about 20\%.



%version 2
% The family of forward indirect call based attacks which can manifest due to a series of factors such as
% memory corruptions, binary layout leackages and availability of useful
% gadgets in a sufficiently large executable poses a serious security threat.
In this paper, we presented \textsc{TypeShield}, a program binary based runtime fine-grained CFI enforcing
tool which can mitigate forward indirect call based attacks by precisely filtering legitimate from illegitimate 
forward indirect control flow transfers in program binaries.
\textsc{TypeShield} uses a novel runtime type checking technique based on function parameter
type checking and parameter counting in order to efficiently filter-out legitimate
and illegitimate forward indirect transfers.
It provides a more precise analysis then existing approaches with a
comparable performance overhead.
We have implemented it and applied it to real software such as web servers and FTP servers.
We demonstrated through extensive experiments and comparisons with related tools
that \textsc{TypeShield} has higher precision and comparable performance overhead than 
existing state-of-the-art tools. To date, we were able to provide a more precise
technique than parameter count based techniques by reducing the possible calltargets 
per callsite ratio by 20\% with an overall reduction of about 
9\% when comparing with similar state-of-the-art approaches.
The outcome is a more precise analysis and a considerably reduced attack surface.
In the spirit of open research,
we have made the source code of \textsc{TypeShield}
publicly available at \url{https://github.com/stub/typeshield}.


