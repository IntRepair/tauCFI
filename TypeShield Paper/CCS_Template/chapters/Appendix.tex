\section*{Appendix}
\label{appendix}

\section{Extended Background}
\subsection{Polymorphism in C++ Programs}
% \textbf{Polymorphism in C++.}
\label{Polymorphism in C++}
Polymorphism, along inheritance and encapsulation, are the most used modern object-oriented concepts in C++. In C++, polymorphism allows accessing different types of objects through a common base class. A pointer of the type of the base object can be used to point to object(s) which are derived from the base class. In C++, there are several types of polymorphism:
\textit{a)} compile-time (or static, usually is implemented with templates), 
\textit{b)} runtime (dynamic, is implemented with inheritance and virtual functions), 
\textit{c)} ad-hoc (\textit{e.g.,} if the range of actual types that can be used is finite and the combinations must be individually specified prior to use), and
\textit{d)} parametric (\textit{e.g.,} if code is written without mention of any specific type and thus can be used transparently with any number of new types). 
The first two are implemented through early and late binding, respectively. In C++, overloading concepts fall under the category of \textit{c)} and virtual functions, templates or parametric classes fall under the category of pure polymorphism. However, C++ provides polymorphism through: 
\textit{i)} virtual functions,
\textit{ii)} function name overloading, and 
\textit{iii)} operator overloading. 
In this paper, we are concerned with dynamic polymorphism, based on virtual functions (10.3 and 11.5 in ISO/IEC N3690~\cite{iso:iecN3690}), because it can be exploited to call: 
\textit{x)} illegitimate virtual table entries (not) contained in the class hierarchy by varying or not the number of parameters and types,
\textit{y)} legitimate virtual table entries (not) contained in the class hierarchy by varying or not the number of parameters and types, and 
\textit{z)} fake virtual tables entries not contained in the class hierarchy by varying or not the number of parameters and types.
By legitimate and illegitimate virtual table entries we mean those virtual table entries which for a single indirect callsite lie in the virtual table hierarchy. More precisely, a virtual table entry is legitimate for a callsite if from the callsite to the virtual table containing the entry there is an inheritance path (see~\cite{haller:shrinkwrap}). Virtual functions have several uses and issues associated, but for the scope of this paper we will look at the indirect callsites which are exploited by calling illegitimate virtual table entries (\textit{i.e.,} functions) with varying number and type of parameters, \textit{x)}. More precisely, 
\textit{1) load-time enforcement:} as calling each indirect callsite (\textit{i.e.,} callee) requires a fixed number of parameters which are passed each time the caller is calling, we enforce a fine-grained CFI policy by statically determining the number and types of all function parameter that belong to an indirect callsite, and
\textit{2) runtime verification:} as differentiating during runtime legitimate from illegitimate indirect caller/callee pairs requires parameter type (along parameter number), we check during run-time before each indirect callsite if the caller matches with the callee based on the previously added checks.

\section{Security Analysis}

\subsection{Checking Indirect Calls in Practice}
% \textbf{Checking Indirect Forward-Edge Calls in Practice.}
\label{C++ Indirect Calls in Practice}
To the best of our knowledge, only the IFCC/VTV~\cite{vtv:tice} tools (up to 8.7\% performance overhead) are deployed in practice
which can be used to check legitimate from illegitimate indirect forward-edge calls during runtime. vPointers are checked based on the class hierarchy. Furthermore, ShrinkWrap~\cite{haller:shrinkwrap} (to the best of our knowledge not deployed in practice) is a tool which further reduces the legitimate virtual table ranges for a given indirect callsite through precise analysis of the program class hierarchy and virtual table hierarchy. Evaluation results show similar performance overhead but more precision with respect to legitimate virtual table entries per callsite. We noticed by analyzing the previous research results that the overhead incurred by these security checks can be very high due to the fact that for each callsite many range checks have to be performed during runtime. Therefore, in our opinion, despite its security benefit these types of checks cannot be applied to high performance applications.

A number of other highly promising tools (albeit also not deployed in practice) can overcome some of the drawbacks of the previously described tools. Bounov \textit{et al.}~\cite{bounov:interleaving} presented a tool ($\approx$ 1\% runtime overhead)
for indirect forward-edge callsite checking based on virtual table layout interleaving. The tool has better performance than VTV and better precision with respect to allowed virtual tables per indirect callsite. Its precision (selecting legitimate virtual tables for each callsite) compared to ShrinkWrap is lower since it does not consider virtual table inheritance paths. vTrust~\cite{zhang:vtrust} (average runtime overhead 2.2\%) enforces two layers of defense (virtual function type enforcement and virtual table pointer sanitization) against virtual table corruption, injection and reuse. TypeArmor~\cite{veen:typearmor} ($\le$ than 3 \% runtime overhead) enforces a CFI-policy based on runtime checking of caller/callee pairs and function parameter count matching. It is important to note 
that there are no C++ language semantics which can be used to enforce type and parameter count matching for indirect caller/callee pairs, this could be addressed with specifically intended language constructs in the future.

\subsection{Security Implications of Indirect Calls}
% \textbf{Security Implications of Forbidden Indirect Calls.}
\label{Security Implications of Forbidden Forward Indirect Calls}
The C++ language standard (N3690~\cite{iso:iecN3690}) does not specify what happens when calling different virtual table entries from an indirect callsite. The standard says that we have a virtual function-related undefined behavior when: \textit{a virtual function call uses an explicit class member access and the object expression refers to the complete object of x or one of that object's base class subobjects but not x or one of its base class subobjects}. As undefined behavior is not a clearly defined concept, we argue that in order to be able to deal with undefined behavior or unspecified behavior related to virtual function calls one needs to know how these language-dependent concepts are implemented inside the used compilers.

Forbidden forward-edge indirect calls are the result of a vPointer corruption. A vPointer corruption is not a vulnerability, but rather a capability which can be the result of a spatial or temporal memory corruption triggered by: 
(1) bad-casting~\cite{byoungyoung:typecasting} of C++ objects, 
(2) buffer overflow in a buffer adjacent to a C++ object or a use-after-free condition~\cite{schuster:coop}.
A vPointer corruption can be exploited in several ways. A manipulated vPointer can be exploited by pointing it in any existing or added program virtual table entry or into a fake virtual table which was added by an attacker. For example in case a vPointer
was corrupted than the attacker could highjack the control flow of the program and start a COOP attack~\cite{schuster:coop}.

vPointer corruptions are a real security threat which can be exploited if there is a memory corruption (\textit{e.g.,} buffer overflow) which is adjacent to the C++ object or a use-after-free condition. As a consequence, each corruption which can reach an object (\textit{e.g.,} bad object casts) is a potential exploit vector for a vPointer corruption. Interestingly to notice in this context is that through:
(1) memory layout analysis (through highly configurable compiler tool chains) of source code based locations which are highly prone to memory corruptions such as declarations and uses of buffers, integers or pointer deallocations one can obtain the internal machine code layout representation.
(2) analysis of a code corruption which is adjacent (based on (1)) to a C++ object based on application class hierarchy, the virtual table hierarchy and each location in source code where an object is declared and used (\textit{e.g.,} modern compiler tool chains can spill out this information for free), one can derive an analysis which can determine---up to a certain extent---if a memory corruption can influence (\textit{e.g.,} is adjacent) to a C++ object.

Finally, tools based on these two concepts (\textit{i.e.,} (1) and (2)) can be used by attackers, \textit{e.g.,} to find new vulnerabilities, and by defenders to harden the source code only at the places which are most exposed to such vulnerabilities (\textit{i.e.,} targeted security hardening).

\subsection{Imprecise Parameter-Count Policies}
% \textbf{Permissive Parameter-Count-Based Policies.}
\label{Too Permissive Parameter-Based Policies}
TypeArmor~\cite{veen:typearmor} is a tool that can enforce a CFI runtime policy for dispatching of callsites based only on
parameter count. The authors argue that their policy reports only an \textit{overestimation} for the parameters prepared by a callsite and \textit{underestimation} for the number of parameters consumed by the matching calltargets. The authors suggest that their technique is effective against COOP attacks. 

We do not fully agree with this claim and, furthermore, we believe that their callsite vs. calltarget set enforcing policy is too permissive and thus many potential indirect forward edge based control flow transfers are possible. Consider the following example. In the best case for each callsite preparing, say, $p=4 \in [1, 6]$ parameters their policy could theoretically allow only the calltargets which consume the same number as parameters as prepared, $c=4 \in [1, 6]$. Note that this does not hold due to 
the aforementioned callsite overestimation and calltarget underestimation, thus all possible numerical mismatches are allowed by their policy as long as $p$ is greater or equal to $c$.

\begin{itemize}[leftmargin=.12in]
\item TypeArmor \textbf{\textit{ideally}} would allow for a single callsite a set of calltargets containing a maximum of $117649$ possibilities if we consider the maximum value of provided parameters to be $p=6$ (due to $p \in [1, 6]$ possible provided parameters). Now, consider 7 C++ integer parameter types $t$: $int$, $char$, $unsigned char$, $bool$, $long$, $unsigned long$, and $short$. Thus, we obtain $t^{p}=7^{6}=117649$ allowed calltargets per callsite if TypeArmor is used. Note that for simplicity reasons we considered $t=7$ but in practice $t$ is often even larger since there are many types of parameters in C++. The complete list of fundamental C++ types contains 20 types; not including data structures or object types. Thus, all these data types would be ignored by TypeArmor. Also, note that all other callsites having more than 6 parameters would be not checked by TypeArmor as well.

\item TypeArmor \textbf{\textit{actually}} allows more than $t^{p}$ calltargets per callsite. If we have $t=7$ integer types due to TypeArmors overestimation and underestimation we get for each callsite an additional number of calltargets. Let $p=6$, then we get $c = 6x + 5y+ 4z + 3t + 2p + 1v$ where:
$x$ is the sum of all calltargets consuming 6 parameters, 
$y$ is the sum of all calltargets consuming 5 parameters 
and so on down to 0 parameters. Note that this holds since TypeArmor allows more parameters to be provided than consumed by the calltarget.
Then, $c = 2100 = 600 + 500 + 400 + 300 + 200 + 100 \ iff \ x=y=z=t=p=v=100$. 
Note that $x=100$ is feasible under realistic conditions in large applications (\textit{i.e.,} Google Chrome, Firefox). 
Next $2100$ is added to $7^{6}$. Thus, for a single callsite providing $p=6$ parameters TypeArmor allows theoretically in 
total $7^{6} + 2100 = 1197496$ calltargets for each callsite.
Similar reasoning applies to $p=5$ where we get $7^{5} + (1500 = 500 + 400 + 300 + 200 + 100) = \ 18307 \ iff \ x=y=z=t=p=v=100$ 
allowed calltarget per callsite, or $p \in [1, 4]$, too.
\end{itemize}

Finally, as TypeArmor is too permissive we present \textsc{TypeShield} which deals with the variable type state explosion due to different parameter types by considering an approximation (note that alias analysis and thus type analysis in binaries is undecidable~\cite{alias:undecidable}) of parameter types based on register width. Consequently, the allowed calltarget set for each callsite is drastically reduced.

\subsection{Real COOP Attack Example}
\label{Real COOP Attack Example}
% \textbf{Real COOP Attack Example.}
\label{Running Example: CVE X}
%%second pic
\begin{figure}[h!]
    \centering
    \includegraphics[width=0.47\textwidth]{figures/class_hierarchy.pdf}
\caption{Class inheritance hierarchy of the classes involved in the COOP attack against the Firefox browser. Red letters 
indicate forbidden virtual table entries and green letters indicate allowed virtual table entries for the given indirect callsite
contained in the main loop gadget.}
\label{Class exploit}
\vspace{-.29cm}
\end{figure}
Figure~\ref{Class exploit} depicts the COOP attack example used as proof of concept exploit presented in~\cite{schuster:coop} and to perform a COOP attack on the Firefox browser. A buffer overflow bug was used in order to call into existing virtual table entries by using a main loop gadget. The attack concludes with the opening of a Unix shell. 

A real-world bug, CVE-2014-3176, was exploited by Crane \textit{et al.}~\cite{crane:readactor++} in order to perform another COOP attack, on the Chromium browser. The details of the second attack are highly complex (\textit{i.e.,} involving not properly handled interaction of extensions, IPC, the sync API, and Google V8) and for this reason we only briefly present the first documented COOP exploit on a Linux machine.

The C++ class \texttt{nsMultiplexInputStream} contains a main loop gadget inside the \texttt{nsMultiplexInputStream::Close(void)} function which is performing indirect calls by dispatching indirect calls on the objects contained in the array. The objects contained in the array during normal execution are of type \texttt{nsInputStream} and each of the objects will call the \texttt{Close(void)} function in order to close each of the previously opened streams. For performing the COOP attack, the attacker crafts a C++ program containing an array buffer holding six fake objects. These fake objects can call inside (and outside) the initial class and virtual table hierarchies with no constraints. During the attack a buffer is created in order to hold the fake objects.
The crafted buffer will be used instead of the real code in order to call different functions available in the program code. For example, the attacker calls a function contained in the class \texttt{xpcAccessibleGeneric} which is not in the class hierarchy or virtual table hierarchy of the initially intended type of objects used inside the array. Moreover, the header file of this class (\texttt{xpcAccessibleGeneric}) is not included in the class \texttt{nsMultiplex-InputStream}. In total six fake objects are used to call into functions residing in unrelated class hierarchies with varying number of parameters and return types. The final goal of this attack is to prepare the program memory such that a Unix shell can be opened at the end of this attack.

This example illustrates why detecting vPointer corruptions is not trivial for real-world applications. As depicted in Figure~\ref{Class exploit}, the class \texttt{nsInputStream} has 11 classes which inherit directly or indirectly from this class. The classes \texttt{nsSeekableStream}, \texttt{nsIPCSeria- lizableInputStream} and \texttt{nsCloneableInputStream} provide additional inherited virtual tables which represent illegitimate calltargets for the initial \texttt{nsInputStream} objects and legitimate calltargets for the six fake objects which were added during the attack. Furthermore, declaration and usage of the objects can be widely spread out in the source code. This makes detection of the object types (\textit{i.e.,} base class), range of virtual tables (\textit{i.e.,} longest virtual table inheritance path for a particular callsite) and parameter types of the virtual table entries (\textit{i.e.,} functions) in which it is allowed to call a trivial task for source code applications, but a hard task when one wants to apply similar security policies 
(\textit{e.g.,} which rely on parameter types of virtual table entries) to binary executables.

\section{Variadic Function Example}
\begin{figure}[thp] % the figure provides the caption
\centering          % which should be centered
\begin{tabular}{c}  % the tabular makes the listing as small as possible and centers it
\footnotesize
\begin{lstlisting}
00000000004222f0 <make_cmd>:
 4222f0:push   %r15
 4222f2:push   %r14
 4222f4:push   %rbx
 4222f5:sub    $0xd0,%rsp
 4222fc:mov    %esi,%r15d
 4222ff:mov    %rdi,%\begin{figure}[!h]
 422302:test   %al,%al
 422304:je     42233d <make_cmd+0x4d>
 422306:movaps %xmm0,0x50(%rsp)
 42230b:movaps %xmm1,0x60(%rsp)
 422310:movaps %xmm2,0x70(%rsp)
 422315:movaps %xmm3,0x80(%rsp)
 42231d:movaps %xmm4,0x90(%rsp)
 422325:movaps %xmm5,0xa0(%rsp)
 42232d:movaps %xmm6,0xb0(%rsp)
 422335:movaps %xmm7,0xc0(%rsp)
 42233d:mov    %r9,0x48(%rsp)
 422342:mov    %r8,0x40(%rsp)
 422347:mov    %rcx,0x38(%rsp)
 42234c:mov    %rdx,0x30(%rsp)
 422351:mov    $0x50,%esi
 422356:mov    %r14,%rdi
 422359:callq  409430 <pcalloc>
\end{lstlisting}
\end{tabular}
\caption{ASM code of the \texttt{make\_cmd} function with optimize level O2, which has a variadic parameter list.}
\label{fig:asmvariadic}
\end{figure}


