\section{Discussion and Limitations}
\label{chapter:Discussion}

\subsection{Limitations of \textsc{TypeShield}}
% \label{section:limit}
First, \textsc{TypeShield} is limited by the capabilities of the DynInst instrumentation environment, where non-returning functions like exit are 
not detected reliably in some cases. As a result, we cannot test the Pure-FTP server, as it heavily relies on these functions. 
The problem is that those non-returning functions usually appear as a second branch within a function that occurs after the normal 
control flow, causing basic blocks from the following function to be attributed to the current function. This results in a malformed 
control flow graph and erroneous attribution of callsites and problematic mis-classifications for both calltargets and callsites.

Second, \textsc{TypeShield} draws on variety within the binary. In particular, we rely on the fact that functions use more than only 
64-bit values or pointers within their parameter list, otherwise, \textsc{TypeShield} is equivalent to a parameter count-based implementation. 
Occurrences of such situations are quite rare, as we learned with our experiments. With a study based on source level information, we could not 
detect a difference between our \textit{type} policy and a \textit{count} policy. However, when using our tool, we were able to detect differences,
which reinforce the fact, that we do not rely on declaration of parameters, but usage of those.

% Third, \textsc{TypeShield} can protect forward indirect edges 
% in a binary program and can complement a shadow stack~\cite{dang:asiaccs} protection technique. For this reason, we assume that \textsc{TypeShield} can run orthogonal with an 
% ideal backward-edge protection mechanism such as a shadow stack~\cite{conti:ccs}. However, the main goal of \textsc{TypeShield} is to complement 
% shadow stack based defenses which fail to account for attacks not violating the backward-edge calling conventions such as the COOP attack.

Fourth, function parameter passing trough floating point registers is currently not supported by \textsc{TypeShield}. We plan 
to address this limitation in future research. Tail calls which make that there is not 1 to 1 matching between callers and callees 
are as for now not supported by \textsc{TypeShield} bu we plan to address this limitation in future work as well.

% Finally, \textsc{TypeShield} is not intended to be more precise than source code based tools such as IFCC/VTV~\cite{vtv:tice}. However, 
% \textsc{TypeShield} is highly useful in situations where the source code is typically not available (\textit{e.g.,} off-the-shelf programs), 
% where programs rely on third party libraries, and where the recompilation of all the shared libraries is not possible. 
% Further, binary based tools such as \textsc{TypeShield} can offer precise protection when source code is not available or 
% recompilation is not feasible or desirable.

% Finally, while a major step forward, \textsc{TypeShield} cannot thwart all possible attacks, as even solutions with access to source 
% code are unable to protect against all possible attacks~\cite{carlini:bending}. In contrast, \textsc{TypeShield}, our binary-based tool, 
% can stop all currently COOP attacks published to date and significantly raises the bar for an adversary when compared to
% TypeArmor and similar tools. Moreover, \textsc{TypeShield} provides a strong mitigation for other types of code-reuse attacks as well
% which violate the caller callee calling convention.

% \subsection{Comparison with TypeArmor}
% \label{section:comptype}
% We are looking at two sets of results. First of all, we compare the overall precision of our implementation
% of the COUNT policy with the results from TypeArmor to set the perspective for the precision of our TYPE 
% policy. We cannot compare data regarding overestimations of calltargets or underestimations of callsites, 
% as TypeArmor did not provide sufficient data. The second point of comparison is the reduction of calltargets
% per callsite, however, this comparison is rather crude, as we most surely do not have the same measuring
% environment and not sufficient data to infer its quality.

% \subsubsection{Precision of Classification}
% TypeArmor reports a geometric mean of 83.26\% for the perfect classification of calltargets regarding 
% parameter count in optimization level O2, which compares rather well to our result of 86.\%. Regarding
% the perfect classification of callsites we report a geometric mean of 71\% perfect classification 
% regarding parameter count, while TypeArmor reports a geometric mean of 79.19\%. However, we also have
% a geometric mean of about 7\% regarding underestimations in the callsite classification with an upper
% bound of 16\%, while TypeArmor reports that it does not incur underestimations in their callsites.
% Now, for our type based classification we incur the cost for two error sources. First, the error from
% the parameter count classification, which we base our type analysis on and second for the type analysis
% itself. The numbers for the perfect classification of calltargets regarding parameter types we report a
% 72.25\% geometric mean of perfect classification, which is 87.85\% of our precision regarding parameter
% counts. However, we report a geometric mean of 57.36\%
% for perfect classification of callsites, which although seemingly low, is still 69.74\% of our precision
% regarding parameter counts.

\subsection{Policy Discovery Trade-offs}
\paul{address the point w.r.t. how hard/easy is for an attacker to discover the used policy and state the trade-offs}
\paul{clarify the point w.r.t. false negatives}

% \subsubsection{Reduction of Available Calltargets}
% While our count based precision focused implementation achieves a reduction in the same ballpark as
% TypeArmor regarding our test targets, lets us believe that our implementation of their classification
% schema is a sufficient approximation to compare against. However, we cannot safely compare those numbers,
% as the information regarding their test environment are rather sparse and the only data available is the
% median, which in our opinion does discard valuable information from the actual result set. This is the
% main reason we implemented an approximation, because we needed more metrics to compare \textsc{TypeShield}
% and TypeArmor regarding calltargets. Using average and sigma, we can report that our precision focused
% type based classification can reduce the number of calltargets, by up to 35\% more than parameter number
% based classification with an overall reduction of about 13\%.

% \subsection{TypeArmor Discrepancies}
% \label{section:discrep}
% Next we will report some discrepancies which we think are worthwhile to be mentioned herein.
% A minor discrepancy between our results and the results of TypeArmor is that, while TypeArmor basically implements
% what we call a destructive merge operator for the liveness analysis. However, our data suggests that this
% operator is marginally inferior to the union path merge operator, when we compared them in our implementation.
% A major concern is the classification of calltargets, while we were able to reduce the number of overestimations
% of calltargets regarding parameter counts to essentially 0, the number of underestimations of calltarget did
% stay around 5\%. This error rate is rather large when compared to the reported 0\% underestimation
% of TypeArmor. An explanation could be: 
% (1) the the different test environments, 
% (2) a bug within our implementation that we are not aware of, 
% (3) or the reaching definitions analysis alone is not the best possible algorithm for this particular scenario.

% \subsection{Improving \textsc{TypeShield}}
% \label{section:venuesimp}
% % In order to improve our type analysis, we briefly highlight two possibilities. 
% 
% First, incorporating a refined data flow analysis and 
% expanding the scope of the analysis include memory analysis as well. The main point of improvement is not the precision but for now 
% more importantly the reduction of underestimations in the callsite analysis.
% In order to refine the data flow analysis, we propose the actual tracking of data values and simple operations, as these
% can be used to better differentiate the actual wideness stored within the current register. The highest gain, 
% we see here would be the establishment of upper and lower bounds regarding values within the register, which 
% would allow for more sophisticated callsite and calltarget invariants. Essentially we would have to resort 
% to symbolic execution or some other sort of precise abstract interpretation.
% 
% Second, expanding the scope to include memory analysis, is another possible way of improving the type analysis, as it 
% would allow to distinguish normal 32-bit or 64-bit values and pointer addresses. Although we already have a 
% limited approach of that in our reaching implementation. Further, we see room for improvement, as we only check
% whether a value is within one of three binary sections or 0.


