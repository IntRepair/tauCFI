\chapter{Discussion}
\label{chapter:Discussion}

We are going to discuss several aspects of interest that came up when comparing against TypeArmor, which are our first two topics of discussion, namely the comparison of results between TypeArmor and TypeShield and whether we found any discrepancies between the data we got from their paper and the data we collected. The second part of our discussion is concerned about the limitations of TypeShield and possible venues of improvement.

\section{Comparison with TypeArmor}
We are looking at two sets of results, first of all we compare the overall precision of our implementation of the COUNT policy with the results from TypeArmor to set the perspective for the precision of our TYPE policy. we cannot compare regarding under estimation of callsites or overestimation of calltargets, as TypeArmor did not provide sufficient data. The second point of comparison is the reduction of calltargets per callsite, however, this comparison is rather crude, as we most surely do not have the same measuring environment and not sufficient data to infer its quality.

\subsection{Precision of Classification}
TypeArmor has a reports a gemoetric mean of 83.26\% for the perfect classification of calltargets regarding parameter count in optimization level O2, which compares rather well to our result of 82.24\%. Regarding the perfect classification of callsites we report a geometric mean of 81.6\% perfect classification regarding parameter count, while TypeArmor reprots a geometric mean of 79.19\%. Howevver we also have a geometric mean of about 7\% regarding underestimations in the callsite classification with an upper bound of 16\%, while TypeArmor reports that it does not incur underestimations in their callsites.
Now, for our type based classification we incur the cost for two error sources, first the error from the parameter count classification, which we base our type analysis on and second for the type analysis itselfs. The numbers for the perfect classification of calltargets regarding parameter types we report a 72.25\% geometric mean of perfect classification, which is 87.85\% of our precision regarding parameter counts. However we report a geometric mean of 57.36\%
for perfect classification of callsites, which altough seemingly low, is still 69.74\% of our precsion regarding parameter counts.

\subsection{Reduction of Available Calltargets}
While our count based precision focused implementation achieves a reduction in the same ballpark as TypeArmour regarding our test targets, lets us believe that our implementation of their classification schema is a sufficient approximation to compare against. However, we cannot safely compare those numbers, as the information regarding their test environment are rather sparse and the only data available is the median, which in our opinion does discard valuable information from the actual result set. This is the main reason we implemented an approximation, because we needed more metrics to compare TypeShield and TypeArmor regarding calltargets. Using average and sigma, we can report that our precision focused type based classificationb can reduce the number of calltargets, by up to 20\% more than parameter number based classification.


\section{Discrepancies or Problems}
Our main problem is that we had no access to source code, which is why we implemented an approximation of TypeArmor, however there are some discrepancies between the data that we collected and data that was presented.
A minor discrepancy between our results and the results of TypeArmor, are that while they basically implemented what we call a destructive pathmerge operator as the best possible. Our data however suggested that this operator is marginally inferior to the union pathmerge operator, when we compared them in our implementation.
A major concern is the classification of calltargets, while we were able to reduce the number of overerstimations of calltargets regarding parameter counts to essentially 0, the number of underestimations of calltarget did stay at a geometric mean of 7\%. This error rate is rather large when compared to the reported 0\% underestimation of TypeArmor, however we are not entirely sure what has caused this discrepancy. A possibility is the differing test environments, or a bug within our implementation that we are not aware of, or simply reaching defintions analysis alone is not the best possible algorithm for this particular problem.

\section{Limitations of TypeShield}
First of all, we are limited by the capabilities of the DynInst Instrumentation Environment, the main problem, we are facing here is that non returning functions like exit are not detected reliably in some cases, which is why we were not able to test the Pure-FTP server, as it heavily relies on these functions. The problem is that those non returning functions usually appear as a second branch within a function that occurs after the normal control flow, causing basic blocks from the following function to be attributed to the current function. This results in a malformed control flow graph and erroneous attribution of callsites and problematic misclassifications for both calltargets and callsites.

Another limitation of TypeShield is that it relies on variety within the binary, in particular we rely on the fact that functions use more than only 64bit values or pointers within their parameter list. Should this scenario occur, our analysis has nothing to work with and essentially degrades into a parameter count based implementation. Thankfully this occurence is quite rare, as we experienced within our experiments. When working based on source level information, we could not detect a difference between our TYPE and a COUNT policies. However when leveraging our tool, we were able to detect differences, which reinforces the fact, that we do not rely on declaration of parameters but usage of those.


\section{Possible Venues of Improvement}
To improve our typeanalysis we see atleast two possibilities, incorporating refined dataflow analysis and expanding the scope to also include memory. The main point of improvement is however not the precision but for now more importantly the reduction of underestimations in the callsite analysis.

To refine the dataflow analysis, we propse the actual tracking of data values and simple operations, as these can be used to better differentiate the actual wideness stored within the current register. The highest gain we see here would be the establishment of upper and lower bounds regarding values within the register, which would allow for more sophisticated callsite and calltarget invariants. Essentially we would have to resort to symbolic execution or some other sort of precise abstract interpretation.

Expanding the scope to also include memory is another possible way of improving the type analysis, as it would allow us to distinguish normal 32 or 64 bit values and pointer addresses. Although we already have a limited approach of that in our reaching implementation, we still see room for improvement, as we only check whether a value is within one of three binary sections or 0.
