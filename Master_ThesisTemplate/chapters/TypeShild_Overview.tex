\chapter{TypeShild Overview}
\label{chapter:TypeShild Overview}

after the Design and Implementation section is done
we pick the most important points of TypeShild design and Implementation and describe them here.
The goal of this section is to be an appetizer for the whole design and Implementation section.
Which are usually dry (trocken).m

\section{1. Select Important Point from Design Chapter}
\section{2. Select Important Point from Design Chapter}
\section{3. Select Important Point from Design Chapter}

\section{Adversary Model}
\label{Adversary Model}
this section is just an example from the typearmor paper, of course we can 
replace it with our one but we need to address roughly the same points, namely
how TypeShild defends against COOP.

example from coop paper:

In general, code reuse attacks against C++ applications
oftentimes start by hijacking a C++ object and its vptr.
Attackers achieve this by exploiting a spatial or temporal
memory corruption vulnerability such as an overflow in a
buffer adjacent to a C++ object or a use-after-free condition.
When the application subsequently invokes a virtual function
on the hijacked object, the attacker-controlled vptr is deref-
erenced and a vfptr is loaded from a memory location of the
attacker’s choice. At this point, the attacker effectively controls
the program counter (rip in x64) of the corresponding thread
in the target application. Generally for code reuse attacks,
controlling the program counter is one of the two basic
requirements. The other one is gaining (partial) knowledge on
the layout of the target application’s address space. Depending
on the context, there may exist different techniques to achieve
this [8], [28], [44], [48].
For COOP, we assume that the attacker controls a C++
object with a vptr and that she can infer the base address of
this object or another auxiliary buffer of sufficient size under
her control. Further, she needs to be able to infer the base
addresses of a set of C++ modules whose binary layouts are
(partly) known to her. For instance, in practice, knowledge on
the base address of a single publicly available C++ library in
the target address space can be sufficient.
These assumptions conform to the attacker settings of most
defenses against code reuse attacks. In fact, many of these
defenses assume far more powerful adversaries that are, e. g.,
able to read and write large (or all) parts of an a

\section{TypeShild: Invariants for Targets and Callsites}
\label{TypeShild: Invariants for Targets and Callsites}
this section is just an example from the typearmor paper, of course we can 
replace it with our one but we need to address roughly the same points, namely
how TypeShild defends against COOP.

\section{TypeShild Impact on COOP}
\label{TypeShild Impact on COOP}
this section is just an example from the typearmor paper, of course we can 
replace it with our one but we need to address roughly the same points, namely
how TypeShild defends against COOP.

