\chapter{Implementation}
\label{chapter:Implementation}


In the Master thesis this section should be no longer than 1 DIN A page:

example of implementation text from USENIX caver Paper.

Please write in the same style.


We implemented C A V ER based on the LLVM Compiler
project [43] (revision 212782, version 3.5.0). The static in-
strumentation module is implemented in Clang’s CodeGen
module and LLVM’s Instrumentation module. The
runtime library is implemented using the compiler-rt
module based on LLVM’s Sanitizer code base. In to-
tal, C A V ER is implemented in 3,540 lines of C++ code
(excluding empty lines and comments).
C A V ER is currently implemented for the Linux x86
platform, and there are a few platform-dependent mech-
anisms. For example, the type and tracing functions for
global objects are placed in the .ctors section of ELF . As
these platform-dependent features can also be found in
other platforms, we believe C A V ER can be ported to other
platforms as well. C A V ER interposes threading functions
to maintain thread contexts and hold a per-thread red-
black tree for stack objects. C A V ER also maintains the
top and bottom addresses of stack segments to efficiently
check pointer membership on the stack. We also modified
the front-end drivers of Clang so that users of C A V ER can
easily build and secure their target applications with one
extra compilation flag and linker flag, respectively.