\chapter{Implementation}
\label{chapter:Related_Work}
We implemented {!todo! name} based on the di-opt environment from patharmor\cite{patharmor}, which relies on the Dyninst library \cite{dyninst} (updated to version 9.2.0). We provided our own pass to analyze and patch given binaries, which consists of {!todo! measure locs} lines of C++ code. The core part of our pass is an instruction analyzer, which relies on the DynamoRIO library \cite{dynamorio} (version 6.6.1) to decode instructions an provide information. This analyser is then used to implement an address taken analysis and configurable versions of the backward reaching definitions algorithm and the forward liveness analysis algorithm, which are both based on the algorithms presented in \cite{typearmor}. Callsites and calltargets are then analyzed using those two algorithms, which are configured based on the origin and the chosen policy, which is decided during compilation time.

Additionally to measure the quality of our tool, we wrote a \texttt{MachineFunction} pass for llvm/clang version 4.0.0 (trunk 283889) in the x86 target code generation portion, to have the lowest possible difference between the binary and the IR of llvm. This pass provides us with ground truth data regarding callsites and calltargets, which we then use in our python evaluation and test environment. We are able to measure various metrics using this environment, which we present in this work.

Our current implementation allows for analysis and patching of any native elf binary on linux x86-64. 



In the Master thesis this section should be no longer than 1 DIN A page:

example of implementation text from USENIX caver Paper.

Please write in the same style.


We implemented C A V ER based on the LLVM Compiler
project [43] (revision 212782, version 3.5.0). The static in-
strumentation module is implemented in Clang’s CodeGen
module and LLVM’s Instrumentation module. The
runtime library is implemented using the compiler-rt
module based on LLVM’s Sanitizer code base. In to-
tal, C A V ER is implemented in 3,540 lines of C++ code
(excluding empty lines and comments).
C A V ER is currently implemented for the Linux x86
platform, and there are a few platform-dependent mech-
anisms. For example, the type and tracing functions for
global objects are placed in the .ctors section of ELF . As
these platform-dependent features can also be found in
other platforms, we believe C A V ER can be ported to other
platforms as well. C A V ER interposes threading functions
to maintain thread contexts and hold a per-thread red-
black tree for stack objects. C A V ER also maintains the
top and bottom addresses of stack segments to efficiently
check pointer membership on the stack. We also modified
the front-end drivers of Clang so that users of C A V ER can
easily build and secure their target applications with one
extra compilation flag and linker flag, respectively.