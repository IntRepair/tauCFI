\chapter{Conclusion}
\label{chapter:Conclusion}
Advanced code reuse attacks like COOP and its extensions or Control Jujutsu manifest due to a combination of facts and problems, 
like memory corruption or predictable binary layout and the fact that the larger our binaries get, the 
higher the chance they contain useful gadgets for an attacker. However, due to their nature, traditional 
CFI cannot detect them, as they do not actually replace code to modify the control flow, but change pointers
in memory, which redirects the targets of indirect callsites, which are uncertain at the time of compilation. 
Two of the most common targets are the pointers to virtual function tables to implemented inheritance in C++ 
and global function pointers. The control flow exhibited by the binary while functioning normal and while 
under attack will seem the same. Address taken analysis alread helped cutting down the number of possible 
calltargets one could inject by a considerable amount. And typearmor improved on that by implementing 
invariants for both callsites and calltargets based on the number of parameters. We had no access to
their sourcecode and therefore had to rely on their paper to implement an approximation for which 
we generated comparable results regarding precision. We improved on that solution by implementing \textit{TypeShield}, 
which allows for a more fine-grained classification of calltargets and indirect callsites by implementing a rather 
simplistic register wideness based type analysis. However, as simplistic as that analysis might be, we showed that 
except for special cases (nginx), we were able to improve upon a parameter count based implementation by reducing 
the average target count of up to 20\%.


%This chapter contains in section~\ref{Conclusion} the conclusion and respectively,
%in section~\ref{Future Work} the future work.
%
%\section{Conclusion} 
%\label{Conclusion} 
%
%%this is the short conclusion version. This can be extended if needed
%The COOP attack which can manifest due to a series of factors such as
%a memory corruption, layout leackage of the binary and presence of useful
%gadgets in sufficient large execuatbles is a serious security threat.
%We have developed \textit{TypeShild}, a runtime fine grained CFI enforcing
%tool which can precissely filter legitimate from illegitimate indirect forward
%gadgets in executables.
%It uses a novel run-time type checking technique based on function parameter
%type checking and parameter counting in order to efficiently filter-out legitimate
%and illegitimate forward edges.
%\textit{TypeShild} provides more precise coverage than existing approaches with
%smaller performance overhead.
%We have implemented \textit{TypeShild} and applied it to real software such as:
%web servers, JavaScript engines, FTP servers and 
%large-scale software including Chrome and Firefox browsers.
%We demonstrated through extensive experiments and comparisons with related software
%that \textit{TypeShild} has higher precision and lower performance overhead than 
%the existing state-of-the-art tools.
%
%\todo[inline]{you can extend this up to a page. If you do than please keep (comment it out) also the old version of the 
%conclusion comment it out}


