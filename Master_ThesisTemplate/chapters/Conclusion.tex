\chapter{Conclusion}
\label{chapter:Conclusion}
This chapter contains in section~\ref{Conclusion} the conclusion and respectively,
in section~\ref{Future Work} the future work.

\section{Conclusion} 
\label{Conclusion} 

%this is the short conclusion version. This can be extended if needed
The COOP attack which can manifest due to a series of factors such as
a memory corruption, layout leackage of the binary and presence of useful
gadgets in sufficient large execuatbles is a serious security threat.
We have developed \textit{TypeShild}, a runtime fine grained CFI enforcing
tool which can precissely filter legitimate from illegitimate indirect forward
gadgets in executables.
It uses a novel run-time type checking technique based on function parameter
type checking and parameter counting in order to efficiently filter-out legitimate
and illegitimate forward edges.
\textit{TypeShild} provides more precise coverage than existing approaches with
smaller performance overhead.
We have implemented \textit{TypeShild} and applied it to real software such as:
web servers, JavaScript engines, FTP servers and 
large-scale software including Chrome and Firefox browsers.
We demonstrated through extensive experiments and comparisons with related software
that \textit{TypeShild} has higher precision and lower performance overhead than 
the existing state-of-the-art tools.

\todo[inline]{you can extend this up to a page. If you do than please keep (comment it out) also the old version of the 
conclusion comment it out}


