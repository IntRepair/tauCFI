\chapter{Introduction}
\label{chapter:Introduction}
Control-Flow Integrity (CFI)~\cite{abadi:cfi, abadi:cfi2} is one of the most used techniques
to secure program execution flows against advanced Code-Reuse Attacks (CRAs).

Advanced CRAs such as COOP~\cite{schuster:coop}

Present the virtual function concept in C++, What does it is good for and what security implications does it have?

Talk about the security implications of vptr corruptions and give some CVEs numbers here.

Briefly talk about source code based solutions which protect against vptr corruptions and n the 
end against COOP. Talk about SafeDispatch, ShrinkWrap, Bounov et al, IFCC/VTV etc.


Give a presentation of TypeArmor

TypeArmor~\cite{veen:typearmor} is a run-time based
tool which enforces a fine-grained forward edge policy
in executables based on caller/callee matching based on 
parameter count. 

The introduction should answer this questions:

1.What is the problem? 
There are no mechanisms in C++ implemented which check during late binding
(realized through virtal call sited, indirect call sites in binaries) that
the target of an indirect call site is legitimate or illegitimate.

Specify The Problem statement. 
In this thesis we want to develop a tool which can mitigate one 
of the most dangerous attack which exploits the missing security checks from above.

2.What are the current solutions?
There are three lines of defence against COOP attacks, source code based,
binary based and runtime based (thre is no real application which can really defend against).

TypeArmor~\cite{veen:typearmor} is the most similar tool to \textit{TypeShild} and it used
function parameter coundting by enforcing a fine-grained policy on valid indirect caller/callee pairs
inside a binary. The policy is checked during run-time by cheching that the number of parameters which 
an indirect call site provides matches with the number of parameter the calle expects. This invariant
helps to defend against COOP and Control Flow Jujutsu.

\todo[inline]{@Matthias: add some limitations of type Armor.}

3.Where the solutions lack?
TypeArmor lacks in precision w.r.t. caller/calle matching since it relies only on parameter counting.

4.What is your idea?
Our insight is to enforce a fine-grained CFI policy by combining function parameter count, type matching.
This offers higher precision than TypeArmot and probably higher performance than TypeArmor has since we 
add less checks in the binary. Thus checking fewer checks in the binary results in a better performance 
overhead than comparable solutions.

5.Contributions.
In summary, we make the following contributions:

\begin{enumerate}
 \item We did this 
 
 \item We did this 
 
 \item We did this.
 
 The rest of the MA is organized as follows.
 
\end{enumerate}

\section{Motivation}
here coms the Motivation.
\\
2-3 sentences.

\section{Research Goals}
here coms the Motivation.
\\
2-3 sentences.

\section{Outline}
The remainder of this thesis is organized as follows.
We start by giving an overview of how \textit{TypeShild}
is designed to mitigate COOP attacks. 
Chapter~\ref{C++ Bad Forward Indirect Calls} gives an overview of bad C++
forward edge calls and its security implications.
Chapter~\ref{chapter:TypeShild Overview} contains an high level overview of \textit{TypeShild}.
Chapter~\ref{chapter:Design} gives an overview of the techniques
used in \textit{TypeShild}.
Chapter~\ref{chapter:Implementation} presents briefly the implementation
details of our tool.
The \textit{TypeShild} implementation is evaluated and discussed in
Chapter~\ref{chapter:Evaluation} and Chapter~\ref{chapter:Discussion}, respectively.
Chapter~\ref{chapter:Related_Work} surveys related work.
Finally, Chapter~\ref{chapter:Future_Work} highlights several future steps and
Chapter~\ref{chapter:Conclusion} concludes this thesis.


