\chapter{Introduction}
\label{chapter:Introduction}
Control-Flow Integrity (CFI)~\cite{abadi:cfi, abadi:cfi2} is one of the most used techniques
to secure program execution flows against advanced Code-Reuse Attacks (CRAs).

Advanced CRAs such as COOP~\cite{schuster:coop}


Proposal: Name for our tool: 

CCTypeMapper (Caller Calle Type Mapper) \\ 
 
\textbf{TypeShild} \\ 
i have selected this name for our tool.
 
TypeProtector \\ 

CCTS (caller/calle type securer or shilder) \\ 

TypeFlower \\ 

you can also make your sugestion here. \\ 

Proposal: \\ 

Citation example:~\cite{latex}. 

The introduction should answer this questions:
\\

1.What is the problem? 
\\

Specify The Problem statement. 
\\
2-3 sentences.

2.What are the current solutions?
\\
talk about TypeArmor~\cite{veen:typearmor}....\\

3.Where the solutions lack?
\\

4.What is your idea?
\\

5.Contributions.
\\

In summary, we make the following contributions:

\begin{enumerate}
 \item We did this 
 
 \item We did this 
 
 \item We did this.
 
 The rest of the MA is organized as follows.
 
\end{enumerate}

\section{Motivation}
here coms the Motivation.
\\
2-3 sentences.

\section{Research Goals}
here coms the Motivation.
\\
2-3 sentences.

\section{Outline}
here coms the Motivation.
\\
2-3 sentences.

example: 

The remainder of this thesis is organized as follows. Chapter 2 provides a profound
background regarding VMs, VMI, and modern rootkits. We relate our work to previous
research in Chapter 3. The design and architecture of WhiteRabbit is discussed in
Chapter 4. This chapter comprises assumptions and necessary means that are required
to meet the goals previously stated in Section 1.2. The WhiteRabbit prototype implementation
is discussed and evaluated in Chapter 5 and Chapter 6, respectively. Finally,
we provide an outlook concerning future work in Chapter 7 and conclude this thesis
with a brief recapitulation in Chapter 8.


