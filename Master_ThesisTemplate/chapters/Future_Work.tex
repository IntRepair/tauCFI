\chapter{Future Work}
\label{chapter:Future_Work}
We see at least five different possible ways of future venues of research, which range from improving the structural matching between ground truth and binary based data to expanding our schema alltogether by incorporating aliasing and tracking memory operations.\\

\textbf{Improving the structural matching capability} is the most important further venue of research, as we need a reliable way to match a ground truth against the resulting binary. This is important, because it is a preqrequisite to the ability to generate reliable measurements and reduces the current uncertainy (we rely on the number of calltargets per callsite to match callsites and furthermore assume that the order within ground truth and binary is the same).\\

\textbf{Finding a better suited callsite analysis} would present itself as another important possibility, as we have still relatively high - up to16\% - number of underestimated callsites. However, this venue should only be attempted after significant improvements to the structural matching of callsites.\\

\textbf{Devising a patching schema} that is based on Dyninst functionality, which allows annotation of calltargets so they can hold at least 4 bytes of arbitrary data. This is required to hold the type data that we generate using our classification. Keeping the runtime overhead of said patching schema low should be the second goal of this venue after satisfying stability.\\

\textbf{Expanding our schema to return values} is another viable venue of further work, as we were not able to reliably reduce the number of problematic classification regarding the return values of functions to managable levels. Should one attempt this, it should be noted that the responsibilities of callsites and calltargets are reversed in this case: The callsite requires return value wideness, while the calltarget needs to provide it.\\

\textbf{Introducing pointer/memory analysis} to distinguish simple 32/64bit values and actual addresses to even further restrict the possible number of calltargets per callsite. This would require more precise dataflow analysis, as in calculating value possibilities for registers at each instruction.
