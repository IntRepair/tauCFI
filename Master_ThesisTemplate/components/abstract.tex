% Abstract for the TUM report document
% Included by MAIN.TEX


\clearemptydoublepage
\phantomsection
\addcontentsline{toc}{chapter}{Abstract}	

\vspace*{2cm}
\begin{center}
{\Large \bf Abstract}
\end{center}
\vspace{1cm}

High security, high performance and high availability 
applications such as the Firefox and Chrome web browsers 
are implemented in C++ for modularity, performance and 
compatibility to name just few reasons.
Virtual functions, which facilitate late binding,
are a key ingredient in facilitating run-time polymorphism
in C++ because it allows and object to use general (its own) 
or specific functions (inherited) contained in the class hierarchy.
However, because of the specific implementation of late binding,
which performs no verification in order to check where an indirect call site 
(virtual object dispatch through virtual pointers (vptrs)) is allow to
call inside the class hierarchy, this opens a large attack surface which
was successfully exploited by the COOP attack.
Since manipulation (changing or inserting new vptrs) violates the 
programmer initial pointer semantics and allows an attacker to
redirect the control flow of the program as he desires, vptrs corruption
has serious security consequences similar to those of other 
data-only corruption vulnerabilities.
Despite the alarmingly high number of vptr corruption
vulnerabilities, the vptr corruption problem has not
been sufficiently addressed by the researchers.

In this paper, we present \textit{TypeShild}, a run-time vptr corruption
detection tool. It is based on executable instrumentation at load time
and uses a novel run-time type and function parameter counter technique
in order to overcome the limitations of current approaches and efficiently
verify dynamic dispatching during run-time.
In particular, \textit{TypeShild} can be automatically and easily used
in conjunction with legacy applications or where source code is missing.
It achieves higher caller/caller matching (precision) and with reasonable
run-time overhead.
We have applied \textit{TypeShild} to real life software such as
web servers, JavaScript engines, FTP servers and large-scale software
including Chrome and Firefox browsers, and were able to efficiently
and with low performance overhead to protect this applications from 
vptr corruptions vulnerabilities.
Our evaluation revealed that \textit{TypeShild} imposes up
to X\% and X\% overhead for performance-intensive
benchmarks on the Chrome and Firefox browsers, respectively.


\todo[inline]{@Matthias: add final \% values at the end, when the evaluation is done.}