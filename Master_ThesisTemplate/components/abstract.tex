% Abstract for the TUM report document
% Included by MAIN.TEX


\clearemptydoublepage
\phantomsection
\addcontentsline{toc}{chapter}{Abstract}	

\vspace*{2cm}
\begin{center}
{\Large \bf Abstract}
\end{center}
\vspace{1cm}

An abstracts abstracts the thesis!

This is just an example that shows how long the abstract should be
and which parts an abstract should contain. \\


Many applications such as the Chrome and Firefox
browsers are largely implemented in C++ for its perfor-
mance and modularity. Type casting, which converts one
type of an object to another, plays an essential role in en-
abling polymorphism in C++ because it allows a program
to utilize certain general or specific implementations in
the class hierarchies. However, if not correctly used, it
may return unsafe and incorrectly casted values, leading
to so-called bad-casting or type-confusion vulnerabili-
ties. Since a bad-casted pointer violates a programmer’s
intended pointer semantics and enables an attacker to
corrupt memory, bad-casting has critical security implica-
tions similar to those of other memory corruption vulner-
abilities. Despite the increasing number of bad-casting
vulnerabilities, the bad-casting detection problem has not
been addressed by the security community.

In this paper, we present C A V ER , a runtime bad-casting
detection tool. It performs program instrumentation
at compile time and uses a new runtime type tracing
mechanism—the type hierarchy table—to overcome the
limitation of existing approaches and efficiently verify
type casting dynamically. In particular, C A V ER can be
easily and automatically adopted to target applications,
achieves broader detection coverage, and incurs reason-
able runtime overhead. We have applied C A V ER to large-
scale software including Chrome and Firefox browsers,
and discovered 11 previously unknown security vulnera-
bilities: nine in GNU libstdc++ and two in Firefox, all
of which have been confirmed and subsequently fixed by
vendors. Our evaluation showed that C A V ER imposes up
to 7.6\% and 64.6\% overhead for performance-intensive
benchmarks on the Chromium and Firefox browsers, re-
spectively.