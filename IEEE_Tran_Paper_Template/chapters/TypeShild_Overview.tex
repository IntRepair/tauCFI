\section{Overview}
\label{chapter:TypeShild Overview}

% \textbf{Adversary Model and Assumptions.}
\subsection{Adversary Model and Assumptions}
\label{Adversary Model}

We largely use the same threat model and the same basic assumptions as described in the TypeArmor 
paper \cite{veen:typearmor}, meaning that our attacker has read and write access to the data 
sections of the attacked binary.  We also assume that the protected binary does not contain 
self modifying code, handcrafted assembly or any kind of obfuscation. We also consider pages 
to be either writable or executable but not both at the same time. We assume 
that our attacker has the ability to execute a memory corruption to hijack the programs 
control flow and that a solution for backward CFI is in place.

\subsection{Invariants for Targets and Callsites}
% \textbf{Invariants for Targets and Callsites.}
\label{Invariants for Targets and Callsites}
Advanced code reuse attacks change the calltargets that are invoked within indirect 
callsites. As standard CFI solutions can hardly restrict these, TypeArmor proposed using two base invariants:
\textit{1)} indirect callsites provide a number of parameters (\textit{i.e.,} possibly overestimated compared to source), and 
\textit{2)} calltargets require a minimum number of parameters (\textit{i.e.,} possibly underestimated compared to source).
The idea is that a callsite might only call functions that do not require more parameters than provided by the callsite. 
To compute the necessary information, TypeArmor uses a modified version of forward liveness analysis for call-targets and 
backward reaching definitions analysis for callsites.

\subsection{\textsc{TypeShield} Impact on COOP}
% \textbf{\textsc{TypeShield} Impact on COOP.}
\label{TypeShild Impact on COOP}
\begin{figure}[h!]
\centering
\resizebox{0.25\textwidth}{!}{
\begin{tikzpicture}

\draw[thick] (0,0) rectangle node[anchor=center]  (320)  {\LARGE{(32, 0)}}  (2,-1);

\draw[thick] (3,2) rectangle node[anchor=center]  (640)  {\LARGE{(64, 0)}}  (5,1);

\draw[thick] (0,3) rectangle node[anchor=center]  (3232) {\LARGE{(32, 32)}} (2,4);

\draw[thick] (3,5) rectangle node[anchor=center]  (6432) {\LARGE{(64, 32)}} (5,6);
\draw[thick] (-1,5) rectangle node[anchor=center] (3264) {\LARGE{(32, 64)}} (-3,6);

\draw[thick] (0,7) rectangle node[anchor=center]  (6464) {\LARGE{(64, 64)}} (2,8);

  \draw[draw, -triangle 45, thick] (3264.north) -- (6464.south);
  \draw[draw, -triangle 45, thick] (6432.north) -- (6464.south);
  \draw[draw, -triangle 45, thick] (3232.north) -- (3264.south);
  \draw[draw, -triangle 45, thick] (3232.north) -- (6432.south);
  
  \draw[draw, -triangle 45, thick] (640.north) -- (6432.south);
  \draw[draw, -triangle 45, thick] (320.north) -- (3232.south);
  \draw[draw, -triangle 45, thick] (320.north) -- (640.south);
  
\end{tikzpicture}
}
\caption{Example for the wideness based schema when only using a parameter wideness of 64-, 32- and 0-bit and only two parameters.}
\label{fig:lattice3264}
\end{figure}
The problem with relying solely on the parameter count is that a callsite can use any call-target as long as the parameter count 
requirement is fulfilled, even if the parameter types do not match (\textit{i.e.,} imagine 8-bit values provided but 64-bit values required). 
Therefore, we extend the classification schema to the parameter types:
\textit{1)} indirect callsites provides a maximum wideness to each parameter (possibly overestimated compared to source), and 
\textit{2)} calltargets require a minimum wideness for each parameter (possibly underestimated compared to source).

The basic idea stays the same, the provision must be no lower than the requirement. However, the approach is more fine-grained 
applying to the wideness of each parameter. The result is that we split the buckets of TypeArmor up into smaller ones as shown 
in the limited example depicted in Figure \ref{fig:lattice3264}.
% There we can see that while in a parameter-count oriented schema a callsite classified as (32,32) would be able to 
% call functions classified as (64,0), however in our parameter wideness oriented schema that is not possible.
